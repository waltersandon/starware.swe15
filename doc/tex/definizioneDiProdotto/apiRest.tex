\subsection{Introduzione}

    L'interfaccia \mgls{rest} è utilizzabile solo da client o utenti che hanno effettuato l'accesso
    attraverso la procedura descritta sopra.

    \subsubsection{Autenticazione}

        Per autenticarsi attraverso il servizio è necessario visitare uno dei seguenti \mgls{uri}:

        \begin{itemize}
            \item \texttt{/auth/google}
            \item \texttt{/auth/facebook}
            \item \texttt{/auth/windows}
        \end{itemize}

        Per scollegarsi è necessario visitare l'\mgls{uri} \texttt{auth/logout}.

        La procedura completa per l'autenticazione è la seguente:

        \begin{enumerate}
            \item L'ospite visita uno degli \mgls{uri} forniti per autenticarsi
            \item Il server reindirizza l'utente alla pagina del provider dove potersi autenticare
            \item L'ospite segue i passi dettati dal provider per autenticarsi
            \item Il provider reindirizza l'ospite al server
            \item Il server verifica che l'ospite si sia effettivamente autenticato con il servizio
            \item Se l'ospite si è autenticato correttamente viene reindirizzato alla pagina principale
                altrimenti viene reindirizzato alla pagina di login per cominciare un'altra procedura di
                autenticazione
        \end{enumerate}

        E' possibile connettere più account diversi all'utente connesso eseguendo la procedura
        di autenticazione appena descritta mentre si è autenticati nell'applicazione.

    \subsubsection{Errori}

        Nel caso avvenga un errore durante una qualsiasi delle richieste all'\mgls{api}, il server risponderà
        con un errore nel formato \mgls{json} seguente:

        \begin{lstlisting}[basicstyle={\ttfamily}]
{
    "error": <descrizione testuale dell'errore>,
    "data": <eventuali dati aggiuntivi sull'errore>
}
        \end{lstlisting}

    \subsubsection{Struttura}

        Di seguito vengono specificati i seguenti attributi:

        \begin{description}[style=multiline,leftmargin=3cm]
            \item[Ruolo minimo:] Il ruolo minimo necessario dell'utente per poter accedere
                alla risorsa
            \item[Descrizione:] Una breve descrizione dei valori ritornati e/o delle
                operazioni compiute
            \item[Precondizioni:] Eventuali precondizioni per l'eseguzione della richiesta
            \item[Argomenti:] Se presente, specifica il formato del corpo
                della richiesta JSON
            \item[Attributi:] Se presente, specifica il formato del corpo
                della risposta JSON dal server
        \end{description}

\subsection{Utenti}

    Un utente rappresenta una qualsiasi persona autenticata con il sistema.

    \begin{center}
        \begin{tabular}{ | l | l | l | } 
        \hline
            Attributo & Tipo & Descrizione \\
        \hline
            \texttt{fullName} & \texttt{string} & 
                Il nome completo non necessariamente univoco dell'utente \\
            \texttt{role} & \texttt{string} & 
                La stringa identificativa del ruolo dell'utente \\
        \hline
        \end{tabular}
    \end{center}

    \subsubsection{Visualizzazione elenco utenti}

        \begin{tabular}{|l|l|} 
            \hline
            \textbf{URL} & \texttt{/users} \\ \hline
            \textbf{Metodo} & \texttt{GET} \\ \hline
            \textbf{Descrizione} & 
                Ritorna la lista degli utenti all'interno del sistema \\ \hline
            \textbf{Dati} & \\ \hline
            \textbf{Risposta} & \
                \begin{lstlisting}[basicstyle={\ttfamily}]
[
    {
        "fullName": string,
        "role": string
    }
]
                \end{lstlisting} \\ \hline
            \textbf{Note} & \\ \hline
        \end{tabular}

    \subsubsection{Visualizzazione dati personali}

        \begin{tabular}{|l|l|} 
            \hline
            \textbf{URL} & \texttt{/users/me} \\ \hline
            \textbf{Metodo} & \texttt{GET} \\ \hline
            \textbf{Descrizione} & 
                Ritorna i dati dell'utente autenticato \\ \hline
            \textbf{Dati} & \\ \hline
            \textbf{Risposta} & \
                \begin{lstlisting}[basicstyle={\ttfamily}]
{
"fullName": string,
"role": string
}
                \end{lstlisting} \\ \hline
            \textbf{Note} & \\ \hline
        \end{tabular}

    \subsubsection{Modifica dati personali}

        \begin{tabular}{|l|l|} 
            \hline
            \textbf{URL} & \texttt{/users/me} \\ \hline
            \textbf{Metodo} & \texttt{POST} \\ \hline
            \textbf{Descrizione} & 
                Modifica dei dati personali da parte dell'utente
                autenticato nel sistema \\ \hline
            \textbf{Dati} & \
                \begin{lstlisting}[basicstyle={\ttfamily}]
{
"fullName": string
}
                \end{lstlisting} \\ \hline
            \textbf{Risposta} & \\ \hline
            \textbf{Note} & \\ \hline
        \end{tabular}

    \subsubsection{Disattivazione utente}

        \begin{figure}
            \begin{tabular}{|l|l|} 
                \hline
                \textbf{URL} & \texttt{/users/:id} \\ \hline
                \textbf{Metodo} & \texttt{DELETE} \\ \hline
                \textbf{Ruolo} & Amministratore \\ \hline
                \textbf{Descrizione} & 
                    Disattiva un utente imponento che esso possa
                    accedere o compiere alcun tipo di azione \\ \hline
                \textbf{Dati} & \\ \hline
                \textbf{Risposta} & \\ \hline
                \textbf{Note} 
                    L'utente che si intende disattivare deve essere sempre
                    di un ruolo gerarchicamente più basso rispetto al proprio.
                    Ad esempio un amministratore potrà solo disattivare studenti
                    e docenti, mentre un proprietario potrà disattivare anche 
                    altri amministratori. & \\ \hline
            \end{tabular}
        \end{figure}

\subsection{Account}

    \par Un account rappresenta una particolare connessione con un account di un 
    provider esterno. Esso rappresenta semplicementa una corrispondenza tra
    un particolare esterno e un utente all'interno dell'applicazione.

    \par Un utente può avere più account collegati e ne può quindi eliminare
    a propria discrezione fintanto che ne abbia almeno uno con cui accedere.

    \begin{center}
        \begin{tabular}{ | l | l | l | } 
        \hline
            Attributo & Tipo & Descrizione \\
        \hline
            \texttt{provider} & \texttt{string} & 
                Il nome identificativo del provider dell'account \\
            \texttt{email} & \texttt{string} & 
                Email, eventualmente non verificata, utilizzata esclusivamente
                per l'indentificazione da parte dell'utente dell'account \\
        \hline
        \end{tabular}
    \end{center}

    \subsubsection{Lettura elenco account}

        \begin{tabular}{|l|l|} 
            \hline
            \textbf{URL} & \texttt{/accounts} \\ \hline
            \textbf{Metodo} & \texttt{GET} \\ \hline
            \textbf{Descrizione} & 
                Ritorna la lista degli account dell'utente connesso \\ \hline
            \textbf{Dati} & \\ \hline
            \textbf{Risposta} & \
                \begin{lstlisting}[basicstyle={\ttfamily}]
[
{
    "provider": string,
    "email": string
}
]
                \end{lstlisting} \\ \hline
            \textbf{Note} & \\ \hline
        \end{tabular}

    \subsubsection{Disconnessione account}

        \begin{figure}
            \begin{tabular}{|l|l|} 
                \hline
                \textbf{URL} & \texttt{/accounts/:id} \\ \hline
                \textbf{Metodo} & \texttt{DELETE} \\ \hline
                \textbf{Descrizione} & 
                    Disassocia un account con un particolare utente \\ \hline
                \textbf{Dati} & \\ \hline
                \textbf{Risposta} & \\ \hline
                \textbf{Note} & \\ \hline
            \end{tabular}
        \end{figure}

\subsection{Domande}

    \par Una domanda può essere di vari tipi:
    \begin{itemize}
        \item A risposta multiple
        \item Vero o falso
    \end{itemize}

    \par Gli attributi ad essa associati possono essere diversi in base
    alla tipologia.

    \begin{figure}[H]
        \begin{tabular}{ | l | l | l | } 
        \hline
            Attributo & Tipo & Descrizione \\
        \hline
            \texttt{body} & \texttt{string} & 
                Il testo della domanda in formato QML \\
            \texttt{author} & \texttt{string} & 
                Riferimento all'autore della domanda \\
            \texttt{isCorrect} & \texttt{string} & 
                Riferimento all'autore della domanda \\
        \hline
        \end{tabular}
        \caption {Attributi domanda vero o falso}
    \end{figure}

    \begin{figure}[H]
        \begin{tabular}{ | l | l | l | } 
        \hline
            Attributo & Tipo & Descrizione \\
        \hline
            \texttt{body} & \texttt{string} & 
                Il testo della domanda in formato QML \\
            \texttt{author} & \texttt{object} & 
                Utente autore della domanda \\
            \texttt{options} & \texttt{object} & 
                Lista delle opzioni \\
        \hline
        \end{tabular}
        \caption {Attributi domanda a risposta multipla}
    \end{figure}

    \begin{figure}[H]
        \begin{tabular}{ | l | l | l | } 
        \hline
            Attributo & Tipo & Descrizione \\
        \hline
            \texttt{statement} & \texttt{string} & 
                Testo dell'opzione \\
            \texttt{body} & \texttt{string} & 
                Il testo della domanda in formato QML \\
        \hline
        \end{tabular}
        \caption {Attributi opzione per domanda a risposta multipla}
    \end{figure}

    \subsubsection{GET /questions}

        \begin{tabular}{|l|l|} 
            \hline
            \textbf{URL} & \texttt{/accounts} \\ \hline
            \textbf{Metodo} & \texttt{GET} \\ \hline
            \textbf{Descrizione} & 
                Ritorna la lista degli account dell'utente connesso \\ \hline
            \textbf{Dati} & \\ \hline
            \textbf{Risposta} & \
                \begin{lstlisting}[basicstyle={\ttfamily}]
[
{
    "provider": string,
    "email": string
}
]
                \end{lstlisting} \\ \hline
            \textbf{Note} & \\ \hline
        \end{tabular}

% Questionari

    % GET /questionnaires
    % POST /questionnaires
    % GET /questionnaires/:id
    % PUT /questionnaires/:id
    % DELETE /questionnaires/:id (?)

% Domande

    % GET /questions
    % POST /questions
    % GET /questions/:id
    % PUT /questions/:id
    % DELETE /questions/:id (?)

% Argomenti

    % GET /tags

% Risposte

    % POST /answers
    % POST /answers

% Risposte a questionari
    
    % POST /submissions
