\subsection{Introduzione}

    \par L'interfaccia \mgls{rest} è utilizzabile solo da client o utenti che hanno effettuato l'accesso
    attraverso la procedura descritta sotto.
    
    \subsubsection{Riferimenti ad oggetti}
        \par I riferimenti ad altri oggetti sono stringhe che rappresentano l'attributo \texttt{\_id} della risorsa. 
        
        \par Nella documentazione seguente vengono identificate con il tipo \texttt{ref} per distinguerle da stringhe normali.

    \subsubsection{Errori}

        \par Nel caso avvenga un errore durante una qualsiasi delle richieste all'\mgls{api}, il server risponderà
        con un errore HTTP e corpo vuoto.
        
    \subsubsection{Sicurezza}
    
        \par Le password e tutti gli altri dati, inclusi
            quelli di sessione sono passati in chiaro, poiché si assume che 
            utilizzato un protocollo sicuro, come \mgls{https}, per la 
            comunicazione con l'\mgls{api}, che quindi deve appoggiarsi ad un
            server che fornisca i certificati e garanzie adeguate.

    \subsubsection{Struttura}

        Di seguito vengono specificati i seguenti attributi:

        \begin{itemize}
            \item \textbf{URI:} L'\mgls{uri} della risorsa specificando il prefisso
                ":" davanti ai parametri
            \item \textbf{Metodo:} Il metodo \mgls{http} utilizzato
            \item \textbf{Parametri:} Eventuali parametri specificati nella \mgls{query string}
                dell'\mgls{url}
            \item \textbf{Descrizione:} Una breve descrizione dei valori ritornati e/o delle
                operazioni compiute
            \item \textbf{Permessi:} Descrizione testuale dei permessi necessari
                per accedere alla risorsa
            \item \textbf{Richiesta JSON:} Formato della richiesta \mgls{json}
            \item \textbf{Risposta JSON:} Formato della risposta \mgls{json}
            \item \textbf{Note:} Note aggiuntive
        \end{itemize}

\subsection{Autenticazione}

    L'utente può effettuare l'autenticazione facendo una richiesta \texttt{POST} sull'\mgls{uri} \texttt{/session} il quale ritornerà come cookie l'ID della sessione assegnata all'utente.
    
    Successive richieste all'\mgls{api} potranno essere completate fornendo il valori della sessione attraverso il cookie impostato.

    \subsubsection{Login}

        \begin{table}[H]
            \begin{center}
                \begin{tabular}{p{0.3\textwidth} p{0.6\textwidth}}
                    \toprule
                    \textbf{URI} & \texttt{/session} \\ \midrule
                    \textbf{Metodo} & \texttt{POST} \\ \midrule
                    \textbf{Parametri} & \\ \midrule
                    \textbf{Descrizione} & Autenticazione dell'utente \\ \midrule
                    \textbf{Permessi} & Ospite \\ \midrule
                    \textbf{Richiesta JSON} & \
                        \begin{lstlisting}[basicstyle={\ttfamily}]
{
    "userName": string,
    "password": string,
}
                        \end{lstlisting}
                        \\ \midrule
                    \textbf{Risposta JSON} & \
                        \begin{lstlisting}[basicstyle={\ttfamily}]
{
    "_id": ref,
    "userName": string,
    "fullName": string,
    "role": ref
}
                        \end{lstlisting}
                        \\ \midrule
                    \textbf{Note} & \\
                    \bottomrule
                \end{tabular}
                \caption{API REST: \texttt{POST /session}}
            \end{center}
        \end{table}

    \subsubsection{Logout}

        \begin{table}[H]
            \begin{center}
                \begin{tabular}{p{0.3\textwidth} p{0.6\textwidth}}
                    \toprule
                    \textbf{URI} & \texttt{/session} \\ \midrule
                    \textbf{Metodo} & \texttt{DELETE} \\ \midrule
                    \textbf{Parametri} & \\ \midrule
                    \textbf{Descrizione} & Disconnessione dell'utente \\ \midrule
                    \textbf{Permessi} & Studente, Docente, Amministratore, Proprietario  \\ \midrule
                    \textbf{Richiesta JSON} & \\ \midrule
                    \textbf{Risposta JSON} & \\ \midrule
                    \textbf{Note} & \\
                    \bottomrule
                \end{tabular}
                \caption{API REST: \texttt{DELETE /session}}
            \end{center}
        \end{table}

\subsection{Utenti}

    Un utente rappresenta una qualsiasi persona autenticata con il sistema.

    \subsubsection{Visualizzazione elenco utenti}

        \begin{table}[H]
            \begin{center}
                \begin{tabular}{p{0.3\textwidth} p{0.6\textwidth}}
                    \toprule
                    \textbf{URI} & \texttt{/users} \\ \midrule
                    \textbf{Metodo} & \texttt{GET} \\ \midrule
                    \textbf{Parametri} & \
                        \begin{itemize}
                            \item \texttt{userName}: filtra i risultati in base all'username
                            \item \texttt{fullName}: filtra i risultati in base al nome completo
                        \end{itemize} 
                        \\ \midrule
                    \textbf{Descrizione} & Restituisce l'elenco degli utenti \\ \midrule
                    \textbf{Permessi} & Studente, Docente, Amministratore, Proprietario  \\ \midrule
                    \textbf{Richiesta JSON} &
                        \begin{lstlisting}[basicstyle={\ttfamily}]
{
    "userName": string,
    "fullName": string,
}
                        \end{lstlisting}
                     \\ \midrule
                    \textbf{Risposta JSON} & \
                        \begin{lstlisting}[basicstyle={\ttfamily}]
[
    {
        "_id": ref,
        "fullName": string,
        "userName": string,
        "role": ref
    }
]
                        \end{lstlisting}
                        \\ \midrule
                    \textbf{Note} & \\
                    \bottomrule
                \end{tabular}
                \caption{API REST: \texttt{GET /users}}
            \end{center}
        \end{table}

    \subsubsection{Visualizzazione dati utente}

        \begin{table}[H]
            \begin{center}
                \begin{tabular}{p{0.3\textwidth} p{0.6\textwidth}}
                    \toprule
                    \textbf{URI} & \texttt{/users/:id} \\ \midrule
                    \textbf{Metodo} & \texttt{GET} \\ \midrule
                    \textbf{Parametri} & \\ \midrule
                    \textbf{Descrizione} & Restituisce i dati di un singolo utente \\ \midrule
                    \textbf{Permessi} & Studente, Docente, Amministratore, Proprietario \\ \midrule
                    \textbf{Richiesta JSON} & \\ \midrule
                    \textbf{Risposta JSON} & \
                        \begin{lstlisting}[basicstyle={\ttfamily}]
{
    "_id": ref,
    "fullName": string,
    "userName": string,
    "role": ref
}
                        \end{lstlisting}
                        \\ \midrule
                    \textbf{Note} & \\
                    \bottomrule
                \end{tabular}
                \caption{API REST: \texttt{GET /users/:id}}
            \end{center}
        \end{table}

    \subsubsection{Visualizzazione profilo personale}

        \begin{table}[H]
            \begin{center}
                \begin{tabular}{p{0.3\textwidth} p{0.6\textwidth}}
                    \toprule
                    \textbf{URI} & \texttt{/users/me} \\ \midrule
                    \textbf{Metodo} & \texttt{GET} \\ \midrule
                    \textbf{Parametri} & \\ \midrule
                    \textbf{Descrizione} & Restituisce i dati dell'utente autenticato \\ \midrule
                    \textbf{Permessi} & Studente, Docente, Amministratore, Proprietario  \\ \midrule
                    \textbf{Richiesta JSON} & \\ \midrule
                    \textbf{Risposta JSON} & \
                        \begin{lstlisting}[basicstyle={\ttfamily}]
{
    "_id": ref,
    "fullName": string,
    "userName": string,
    "role": ref
}
                        \end{lstlisting}
                        \\ \midrule
                    \textbf{Note} & \\
                    \bottomrule
                \end{tabular}
                \caption{API REST: \texttt{GET /users/me}}
            \end{center}
        \end{table}

    \subsubsection{Registrazione utente}

        \begin{table}[H]
            \begin{center}
                \begin{tabular}{p{0.3\textwidth} p{0.6\textwidth}}
                    \toprule
                    \textbf{URI} & \texttt{/users} \\ \midrule
                    \textbf{Metodo} & \texttt{POST} \\ \midrule
                    \textbf{Parametri} & \\ \midrule
                    \textbf{Descrizione} & Registrazione utente \\ \midrule
                    \textbf{Permessi} & Studente, Docente, Amministratore, Proprietario  \\ \midrule
                    \textbf{Richiesta JSON} & \
                        \begin{lstlisting}[basicstyle={\ttfamily}]
{
    "fullName": string,
    "userName": string,
    "password": string
}
                        \end{lstlisting}
                        \\ \midrule
                    \textbf{Risposta JSON} & \
                        \begin{lstlisting}[basicstyle={\ttfamily}]
{
    "_id": string,
    "userName": string,
    "fullName": string,
    "role": ref
}
                        \end{lstlisting}
                        \\ \midrule
                    \textbf{Note} & \\
                    \bottomrule
                \end{tabular}
                \caption{API REST: \texttt{POST /users}}
            \end{center}
        \end{table}

    \subsubsection{Modifica profilo personale}

        \begin{table}[H]
            \begin{center}
                \begin{tabular}{p{0.3\textwidth} p{0.6\textwidth}}
                    \toprule
                    \textbf{URI} & \texttt{/users/me} \\ \midrule
                    \textbf{Metodo} & \texttt{POST} \\ \midrule
                    \textbf{Parametri} & \\ \midrule
                    \textbf{Descrizione} & Modifica dati profilo dell'utente autenticato \\ \midrule
                    \textbf{Permessi} & Studente, Docente, Amministratore, Proprietario  \\ \midrule
                    \textbf{Richiesta JSON} & \
                        \begin{lstlisting}[basicstyle={\ttfamily}]
{
    "fullName": string,
    "userName": string
}
                        \end{lstlisting}
                        \\ \midrule
                    \textbf{Risposta JSON} & \\ \midrule
                    \textbf{Note} & \\
                    \bottomrule
                \end{tabular}
                \caption{API REST: \texttt{POST /users/me}}
            \end{center}
        \end{table}

    \subsubsection{Cambio password}

        \begin{table}[H]
            \begin{center}
                \begin{tabular}{p{0.3\textwidth} p{0.6\textwidth}}
                    \toprule
                    \textbf{URI} & \texttt{/users/me} \\ \midrule
                    \textbf{Metodo} & \texttt{POST} \\ \midrule
                    \textbf{Parametri} & \\ \midrule
                    \textbf{Descrizione} & Modifica password dell'utente autenticato \\ \midrule
                    \textbf{Permessi} & Studente, Docente, Amministratore, Proprietario  \\ \midrule
                    \textbf{Richiesta JSON} & \
                        \begin{lstlisting}[basicstyle={\ttfamily}]
{
    "oldPassword": string,
    "newPassword": string
}
                        \end{lstlisting}
                        \\ \midrule
                    \textbf{Risposta JSON} & \\ \midrule
                    \textbf{Note} & \\
                    \bottomrule
                \end{tabular}
                \caption{API REST: \texttt{POST /users/me}}
            \end{center}
        \end{table}

    \subsubsection{Cambio ruolo utente}

        \begin{table}[H]
            \begin{center}
                \begin{tabular}{p{0.3\textwidth} p{0.6\textwidth}}
                    \toprule
                    \textbf{URI} & \texttt{/users/:id} \\ \midrule
                    \textbf{Metodo} & \texttt{POST} \\ \midrule
                    \textbf{Parametri} & \\ \midrule
                    \textbf{Descrizione} & Cambio del ruolo dell'utente \\ \midrule
                    \textbf{Permessi} & Amministratore, Proprietario  \\ \midrule
                    \textbf{Richiesta JSON} & \
                        \begin{lstlisting}[basicstyle={\ttfamily}]
{
    "role": ref
}
                        \end{lstlisting}
                        \\ \midrule
                    \textbf{Risposta JSON} & \\ \midrule
                    \textbf{Note} & Il chiamante può modificare solo il ruolo di un utente di ruolo
                        inferiore al proprio \\
                    \bottomrule
                \end{tabular}
                \caption{API REST: \texttt{POST /users/:id}}
            \end{center}
        \end{table}

    \subsubsection{Rimozione utente}

        \begin{table}[H]
            \begin{center}
                \begin{tabular}{p{0.3\textwidth} p{0.6\textwidth}}
                    \toprule
                    \textbf{URI} & \texttt{/users/:id} \\ \midrule
                    \textbf{Metodo} & \texttt{DELETE} \\ \midrule
                    \textbf{Parametri} & \\ \midrule
                    \textbf{Descrizione} & Rimozione utente \\ \midrule
                    \textbf{Permessi} & Amministratore, Proprietario  \\ \midrule
                    \textbf{Richiesta JSON} & \\ \midrule
                    \textbf{Risposta JSON} & \\ \midrule
                    \textbf{Note} & Il chiamante può rimuovere solo un utente di ruolo
                        inferiore al proprio \\
                    \bottomrule
                \end{tabular}
                \caption{API REST: \texttt{DELETE /users/:id}}
            \end{center}
        \end{table}

\subsection{Domande}

    \subsubsection{Visualizzazione elenco domande}

        \begin{table}[H]
            \begin{center}
                \begin{tabular}{p{0.3\textwidth} p{0.6\textwidth}}
                    \toprule
                    \textbf{URI} & \texttt{/questions} \\ \midrule
                    \textbf{Metodo} & \texttt{GET} \\ \midrule
                    \textbf{Parametri} & \
                        \begin{itemize}
                            \item \textbf{keywords}: filtro per parole chiave
                            \item \textbf{tags}: filtro per argomenti separati "|" 
                            \item \textbf{author}: filtro per autore
                        \end{itemize}
                        \\ \midrule
                    \textbf{Descrizione} & Visualizzazione dell'elenco delle domande \\ \midrule
                    \textbf{Permessi} & Ospite \\ \midrule
                    \textbf{Richiesta JSON} & \\ \midrule
                    \textbf{Risposta JSON} & \
                        \begin{lstlisting}[basicstyle={\ttfamily}]
[
    {
        "_id": ref,
        "body": string,
        "author": ref
        "tags": [ref]
    }
]
                        \end{lstlisting}
                        \\ \midrule
                    \textbf{Note} & \\
                    \bottomrule
                \end{tabular}
                \caption{API REST: \texttt{GET /questions}}
            \end{center}
        \end{table}

    \subsubsection{Visualizzazione domanda}

        \begin{table}[H]
            \begin{center}
                \begin{tabular}{p{0.3\textwidth} p{0.6\textwidth}}
                    \toprule
                    \textbf{URI} & \texttt{/questions/:id} \\ \midrule
                    \textbf{Metodo} & \texttt{GET} \\ \midrule
                    \textbf{Parametri} & \\ \midrule
                    \textbf{Descrizione} & Visualizzazione domanda specifica \\ \midrule
                    \textbf{Permessi} & Ospite \\ \midrule
                    \textbf{Richiesta JSON} & \\ \midrule
                    \textbf{Risposta JSON} & \
                        \begin{lstlisting}[basicstyle={\ttfamily}]
{
    "_id": ref,
    "body": string,
    "author": ref
    "tags": [ref]
}
                        \end{lstlisting}
                        \\ \midrule
                    \textbf{Note} & \\
                    \bottomrule
                \end{tabular}
                \caption{API REST: \texttt{GET /questions/:id}}
            \end{center}
        \end{table}

    \subsubsection{Creazione domanda}

        \begin{table}[H]
            \begin{center}
                \begin{tabular}{p{0.3\textwidth} p{0.6\textwidth}}
                    \toprule
                    \textbf{URI} & \texttt{/questions} \\ \midrule
                    \textbf{Metodo} & \texttt{POST} \\ \midrule
                    \textbf{Parametri} & \\ \midrule
                    \textbf{Descrizione} & Creazione nuova domanda \\ \midrule
                    \textbf{Permessi} & Docente, Amministratore, Proprietario  \\ \midrule
                    \textbf{Richiesta JSON} & \
                        \begin{lstlisting}[basicstyle={\ttfamily}]
{
    "body": string,
    "tags": [ref]
}
                        \end{lstlisting}
                        \\ \midrule
                    \textbf{Risposta JSON} & \
                        \begin{lstlisting}[basicstyle={\ttfamily}]
{
    "_id": ref,
    "body": string,
    "author": ref,
    "tags": [ref]
}
                        \end{lstlisting}
                        \\ \midrule
                    \textbf{Note} & \\
                    \bottomrule
                \end{tabular}
                \caption{API REST: \texttt{POST /questions}}
            \end{center}
        \end{table}

    \subsubsection{Modifica domanda}

        \begin{table}[H]
            \begin{center}
                \begin{tabular}{p{0.3\textwidth} p{0.6\textwidth}}
                    \toprule
                    \textbf{URI} & \texttt{/questions/:id} \\ \midrule
                    \textbf{Metodo} & \texttt{PUT} \\ \midrule
                    \textbf{Parametri} & \\ \midrule
                    \textbf{Descrizione} & Modifica domanda \\ \midrule
                    \textbf{Permessi} & Docente, Amministratore, Proprietario  \\ \midrule
                    \textbf{Richiesta JSON} & \
                        \begin{lstlisting}[basicstyle={\ttfamily}]
{
    "body": string,
    "tags": [ref]
}
                        \end{lstlisting}
                        \\ \midrule
                    \textbf{Risposta JSON} & \\ \midrule
                    \textbf{Note} & La domanda può essere modificata solo dal proprio autore \\
                    \bottomrule
                \end{tabular}
                \caption{API REST: \texttt{PUT /questions/:id}}
            \end{center}
        \end{table}

    \subsubsection{Rimozione domanda}

        \begin{table}[H]
            \begin{center}
                \begin{tabular}{p{0.3\textwidth} p{0.6\textwidth}}
                    \toprule
                    \textbf{URI} & \texttt{/questions/:id} \\ \midrule
                    \textbf{Metodo} & \texttt{DELETE} \\ \midrule
                    \textbf{Parametri} & \\ \midrule
                    \textbf{Descrizione} & Rimozione domanda \\ \midrule
                    \textbf{Permessi} & Docente, Amministratore, Proprietario  \\ \midrule
                    \textbf{Richiesta JSON} & \\ \midrule
                    \textbf{Risposta JSON} & \\ \midrule
                    \textbf{Note} & La domanda può essere eliminata solo dal proprio autore
                        o da un utente con ruolo superiore; Il chiamante può eliminare la domanda solo se non è associata ad alcun questionario  \\
                    \bottomrule
                \end{tabular}
                \caption{API REST: \texttt{DELETE /questions/:id}}
            \end{center}
        \end{table}

\subsection{Questionari}

    \subsubsection{Visualizzazione elenco questionari}

        \begin{table}[H]
            \begin{center}
                \begin{tabular}{p{0.3\textwidth} p{0.6\textwidth}}
                    \toprule
                    \textbf{URI} & \texttt{/questionnaires} \\ \midrule
                    \textbf{Metodo} & \texttt{GET} \\ \midrule
                    \textbf{Parametri} & \
                        \begin{itemize}
                            \item \textbf{title}: filtro per titolo
                            \item \textbf{tags}: filtro per argomenti separati "|" 
                            \item \textbf{author}: filtro per autore del questionario
                        \end{itemize}
                        \\ \midrule
                    \textbf{Descrizione} & Visualizzazione elenco questionari \\ \midrule
                    \textbf{Permessi} & Ospite \\ \midrule
                    \textbf{Richiesta JSON} & \\ \midrule
                    \textbf{Risposta JSON} & \
                        \begin{lstlisting}[basicstyle={\ttfamily}]
[
    {
        "_id": ref,
        "title": string,
        "author": ref,
        "tags": [ref],
        "questions": [ref] 
    }
]
                        \end{lstlisting}
                        \\ \midrule
                    \textbf{Note} & \\
                    \bottomrule
                \end{tabular}
                \caption{API REST: \texttt{GET /questionnaires}}
            \end{center}
        \end{table}

    \subsubsection{Visualizzazione questionario}

        \begin{table}[H]
            \begin{center}
                \begin{tabular}{p{0.3\textwidth} p{0.6\textwidth}}
                    \toprule
                    \textbf{URI} & \texttt{/questionnaires/:id} \\ \midrule
                    \textbf{Metodo} & \texttt{GET} \\ \midrule
                    \textbf{Parametri} & \\ \midrule
                    \textbf{Descrizione} & Visualizzazione questionario specifico \\ \midrule
                    \textbf{Permessi} & Ospite \\ \midrule
                    \textbf{Richiesta JSON} & \\ \midrule
                    \textbf{Risposta JSON} & \
                        \begin{lstlisting}[basicstyle={\ttfamily}]
{
    "_id": ref,
    "title": string,
    "author": ref,
    "tags": [ref],
    "questions": [ref] 
}
                        \end{lstlisting}
                        \\ \midrule
                    \textbf{Note} & \\
                    \bottomrule
                \end{tabular}
                \caption{API REST: \texttt{GET /questionnaires/:id}}
            \end{center}
        \end{table}

    \subsubsection{Creazione questionario}

        \begin{table}[H]
            \begin{center}
                \begin{tabular}{p{0.3\textwidth} p{0.6\textwidth}}
                    \toprule
                    \textbf{URI} & \texttt{/questionnaires} \\ \midrule
                    \textbf{Metodo} & \texttt{POST} \\ \midrule
                    \textbf{Parametri} & \\ \midrule
                    \textbf{Descrizione} & Creazione nuovo questionario \\ \midrule
                    \textbf{Permessi} & Docente, Amministratore, Proprietario  \\ \midrule
                    \textbf{Richiesta JSON} & \
                        \begin{lstlisting}[basicstyle={\ttfamily}]
{
    "title": string,
    "tags": [ref],
    "questions": [ref] 
}
                        \end{lstlisting}
                        \\ \midrule
                    \textbf{Risposta JSON} & \
                        \begin{lstlisting}[basicstyle={\ttfamily}]
{
    "_id": ref,
    "title": string,
    "author": ref,
    "tags": [ref]
    "questions": [ref] 
}
                        \end{lstlisting}
                        \\ \midrule
                    \textbf{Note} & \\
                    \bottomrule
                \end{tabular}
                \caption{API REST: \texttt{POST /questionnaires/:id}}
            \end{center}
        \end{table}

    \subsubsection{Modifica questionario}

        \begin{table}[H]
            \begin{center}
                \begin{tabular}{p{0.3\textwidth} p{0.6\textwidth}}
                    \toprule
                    \textbf{URI} & \texttt{/questionnaires/:id} \\ \midrule
                    \textbf{Metodo} & \texttt{PUT} \\ \midrule
                    \textbf{Parametri} & \\ \midrule
                    \textbf{Descrizione} & Modifica questionario \\ \midrule
                    \textbf{Permessi} & Docente, Amministratore, Proprietario  \\ \midrule
                    \textbf{Richiesta JSON} & \
                        \begin{lstlisting}[basicstyle={\ttfamily}]
{
    "title": string,
    "tags": [ref],
    "questions": [ref] 
}
                        \end{lstlisting}
                        \\ \midrule
                    \textbf{Risposta JSON} & \\ \midrule
                    \textbf{Note} & Il questionario può essere modificato solo dal proprio autore \\
                    \bottomrule
                \end{tabular}
                \caption{API REST: \texttt{PUT /questionnaires/:id}}
            \end{center}
        \end{table}

    \subsubsection{Rimozione questionario}

        \begin{table}[H]
            \begin{center}
                \begin{tabular}{p{0.3\textwidth} p{0.6\textwidth}}
                    \toprule
                    \textbf{URI} & \texttt{/questionnaires/:id} \\ \midrule
                    \textbf{Metodo} & \texttt{DELETE} \\ \midrule
                    \textbf{Parametri} & \\ \midrule
                    \textbf{Descrizione} & Rimozione questionario \\ \midrule
                    \textbf{Permessi} & Docente, Amministratore, Proprietario  \\ \midrule
                    \textbf{Richiesta JSON} & \\ \midrule
                    \textbf{Risposta JSON} & \\ \midrule
                    \textbf{Note} & Il questionario può essere eliminato solo dal proprio autore
                        o da un utente con ruolo superiore \\
                    \bottomrule
                \end{tabular}
                \caption{API REST: \texttt{DELETE /questionnaires/:id}}
            \end{center}
        \end{table}
        
\subsection{Risposte}

    \subsubsection{Visualizzazione elenco risposte}
    
        \begin{table}[H]
            \begin{center}
                \begin{tabular}{p{0.3\textwidth} p{0.6\textwidth}}
                    \toprule
                    \textbf{URI} & \texttt{/answers} \\ \midrule
                    \textbf{Metodo} & \texttt{GET} \\ \midrule
                    \textbf{Parametri} & \
                        \begin{itemize}
                            \item \textbf{authors}: filtro per autori separati da "|"
                            \item \textbf{questions}: filtro per domande separate da "|" 
                            \item \textbf{questionnaires}: filtro per questionari separati da "|" 
                        \end{itemize}
                        \\ \midrule
                    \textbf{Descrizione} & Visualizzazione elenco delle risposte \\ \midrule
                    \textbf{Permessi} & Ospite \\ \midrule
                    \textbf{Richiesta JSON} & \\ \midrule
                    \textbf{Risposta JSON} & \
                        \begin{lstlisting}[basicstyle={\ttfamily}]
[
    {
        "_id": ref,
        "question": ref,
        "questionnaire": ref,
        "author": ref | null,
        "score": integer
    }
]
                        \end{lstlisting}
                        \\ \midrule
                    \textbf{Note} & \\
                    \bottomrule
                \end{tabular}
                \caption{API REST: \texttt{GET /answers}}
            \end{center}
        \end{table}
        
    \subsubsection{Visualizzazione risposta}
    
        \begin{table}[H]
            \begin{center}
                \begin{tabular}{p{0.3\textwidth} p{0.6\textwidth}}
                    \toprule
                    \textbf{URI} & \texttt{/answers/:id} \\ \midrule
                    \textbf{Metodo} & \texttt{GET} \\ \midrule
                    \textbf{Parametri} & \\ \midrule
                    \textbf{Descrizione} & Visualizzazione risposta singola \\ \midrule
                    \textbf{Permessi} & Ospite \\ \midrule
                    \textbf{Richiesta JSON} & \\ \midrule
                    \textbf{Risposta JSON} & \
                        \begin{lstlisting}[basicstyle={\ttfamily}]
{
    "_id": ref,
    "question": ref,
    "questionnaire": ref,
    "author": ref | null,
    "score": integer
}
                        \end{lstlisting}
                        \\ \midrule
                    \textbf{Note} & \\
                    \bottomrule
                \end{tabular}
                \caption{API REST: \texttt{GET /answers/:id}}
            \end{center}
        \end{table}
        
    \subsubsection{Creazione risposta}
    
        \begin{table}[H]
            \begin{center}
                \begin{tabular}{p{0.3\textwidth} p{0.6\textwidth}}
                    \toprule
                    \textbf{URI} & \texttt{/answers} \\ \midrule
                    \textbf{Metodo} & \texttt{POST} \\ \midrule
                    \textbf{Parametri} & \\ \midrule
                    \textbf{Descrizione} & Creazione nuova risposta \\ \midrule
                    \textbf{Permessi} & Ospite \\ \midrule
                    \textbf{Richiesta JSON} & \
                        \begin{lstlisting}[basicstyle={\ttfamily}]
{
    "question": ref,
    "questionnaire": ref,
    "score": integer
}
                        \end{lstlisting}
                        \\ \midrule
                    \textbf{Risposta JSON} & \\ \midrule
                    \textbf{Note} & \\
                    \bottomrule
                \end{tabular}
                \caption{API REST: \texttt{POST /answers}}
            \end{center}
        \end{table}

\subsection{Argomenti}

    \subsubsection{Visualizzazione elenco argomenti}

        \begin{table}[H]
            \begin{center}
                \begin{tabular}{p{0.3\textwidth} p{0.6\textwidth}}
                    \toprule
                    \textbf{URI} & \texttt{/tags} \\ \midrule
                    \textbf{Metodo} & \texttt{GET} \\ \midrule
                    \textbf{Parametri} & \
                        \begin{itemize}
                            \item \textbf{keywords}: filtro per parole chiave
                        \end{itemize}
                        \\ \midrule
                    \textbf{Descrizione} & Visualizzazione elenco argomenti \\ \midrule
                    \textbf{Permessi} & Ospite \\ \midrule
                    \textbf{Richiesta JSON} & \\ \midrule
                    \textbf{Risposta JSON} & \
                        \begin{lstlisting}[basicstyle={\ttfamily}]
[
    {
        "_id": ref,
        "name": string,
        "description": string
    }
]
                        \end{lstlisting}
                        \\ \midrule
                    \textbf{Note} & \\
                    \bottomrule
                \end{tabular}
                \caption{API REST: \texttt{GET /tags}}
            \end{center}
        \end{table}

    \subsubsection{Visualizzazione argomento}

        \begin{table}[H]
            \begin{center}
                \begin{tabular}{p{0.3\textwidth} p{0.6\textwidth}}
                    \toprule
                    \textbf{URI} & \texttt{/tags/:id} \\ \midrule
                    \textbf{Metodo} & \texttt{GET} \\ \midrule
                    \textbf{Parametri} & \\ \midrule
                    \textbf{Descrizione} & Visualizzazione argomento specifico \\ \midrule
                    \textbf{Permessi} & Ospite, \\ \midrule
                    \textbf{Richiesta JSON} & \\ \midrule
                    \textbf{Risposta JSON} & \
                        \begin{lstlisting}[basicstyle={\ttfamily}]
{
    "_id": ref,
    "name": string,
    "description": string
}
                        \end{lstlisting}
                        \\ \midrule
                    \textbf{Note} & \\
                    \bottomrule
                \end{tabular}
                \caption{API REST: \texttt{GET /tags/:id}}
            \end{center}
        \end{table}

    \subsubsection{Creazione argomento}
    
    \begin{table}[H]
        \begin{center}
            \begin{tabular}{p{0.3\textwidth} p{0.6\textwidth}}
                \toprule
                \textbf{URI} & \texttt{/tags} \\ \midrule
                \textbf{Metodo} & \texttt{POST} \\ \midrule
                \textbf{Parametri} & \\ \midrule
                \textbf{Descrizione} & Creazione nuovo argomento \\ \midrule
                \textbf{Permessi} & Docente, Amministratore, Proprietario  \\ \midrule
                \textbf{Richiesta JSON} & \
                    \begin{lstlisting}[basicstyle={\ttfamily}]  
{
    "name": string,
    "description": string
}
                    \end{lstlisting}
                \\ \midrule
                \textbf{Risposta JSON} & \
                    \begin{lstlisting}[basicstyle={\ttfamily}]
{
    "_id": ref,
    "name": string,
    "description": string
}
                    \end{lstlisting}
                    \\ \midrule
                \textbf{Note} & \\
                \bottomrule
            \end{tabular}
            \caption{API REST: \texttt{POST /tags}}
        \end{center}
    \end{table}
    
    \subsubsection{Modifica argomento}
    
    \begin{table}[H]
        \begin{center}
            \begin{tabular}{p{0.3\textwidth} p{0.6\textwidth}}
                \toprule
                \textbf{URI} & \texttt{/tags/:id} \\ \midrule
                \textbf{Metodo} & \texttt{PUT} \\ \midrule
                \textbf{Parametri} & \\ \midrule
                \textbf{Descrizione} & Modifica argomento \\ \midrule
                \textbf{Permessi} & Docente, Amministratore, Proprietario  \\ \midrule
                \textbf{Richiesta JSON} &\
                    \begin{lstlisting}[basicstyle={\ttfamily}]
{
    "name": string,
    "description": string
}
                    \end{lstlisting}
                \\ \midrule
                \textbf{Risposta JSON} & \\ \midrule
                \textbf{Note} & \\
                \bottomrule
            \end{tabular}
            \caption{API REST: \texttt{PUT /tags}}
        \end{center}
    \end{table}
    
        \subsubsection{Elimina argomento}
        
        \begin{table}[H]
            \begin{center}
                \begin{tabular}{p{0.3\textwidth} p{0.6\textwidth}}
                    \toprule
                    \textbf{URI} & \texttt{/tags/:id} \\ \midrule
                    \textbf{Metodo} & \texttt{DELETE} \\ \midrule
                    \textbf{Parametri} & \\ \midrule
                    \textbf{Descrizione} & Elimina argomento \\ \midrule
                    \textbf{Permessi} & Docente, Amministratore, Proprietario  \\ \midrule
                    \textbf{Richiesta JSON} & \\ \midrule
                    \textbf{Risposta JSON} & \\ \midrule
                    \textbf{Note} & Il chiamante può eliminare un argomento solo se non ci sono domande associate allo stesso \\ 
                    \bottomrule
                \end{tabular}
                \caption{API REST: \texttt{PUT /tags}}
            \end{center}
        \end{table}
    

\subsection{Ruoli}

    \subsubsection{Visualizzazione elenco ruoli}

        \begin{table}[H]
            \begin{center}
                \begin{tabular}{p{0.3\textwidth} p{0.6\textwidth}}
                    \toprule
                    \textbf{URI} & \texttt{/roles} \\ \midrule
                    \textbf{Metodo} & \texttt{GET} \\ \midrule
                    \textbf{Parametri} & \
                        \begin{itemize}
                            \item \textbf{keywords}: filtro per parole chiave
                        \end{itemize}
                        \\ \midrule
                    \textbf{Descrizione} & Visualizzazione elenco ruoli \\ \midrule
                    \textbf{Permessi} & Amministratore, Proprietario  \\ \midrule
                    \textbf{Richiesta JSON} & \\ \midrule
                    \textbf{Risposta JSON} & \
                        \begin{lstlisting}[basicstyle={\ttfamily}]
[
    {
        "_id": ref,
        "name": string
    }
]
                        \end{lstlisting}
                        \\ \midrule
                    \textbf{Note} & \\
                    \bottomrule
                \end{tabular}
                \caption{API REST: \texttt{GET /roles}}
            \end{center}
        \end{table}

    \subsubsection{Visualizzazione ruolo}

        \begin{table}[H]
            \begin{center}
                \begin{tabular}{p{0.3\textwidth} p{0.6\textwidth}}
                    \toprule
                    \textbf{URI} & \texttt{/roles/:id} \\ \midrule
                    \textbf{Metodo} & \texttt{GET} \\ \midrule
                    \textbf{Parametri} & \\ \midrule
                    \textbf{Descrizione} & Visualizzazione ruolo singolo \\ \midrule
                    \textbf{Permessi} & Studente, Docente, Amministratore, Proprietario  \\ \midrule
                    \textbf{Richiesta JSON} & \\ \midrule
                    \textbf{Risposta JSON} & \
                        \begin{lstlisting}[basicstyle={\ttfamily}]
{
    "_id": ref,
    "name": string
}
                        \end{lstlisting}
                        \\ \midrule
                    \textbf{Note} & \\
                    \bottomrule
                \end{tabular}
                \caption{API REST: \texttt{GET /roles/:id}}
            \end{center}
        \end{table}
