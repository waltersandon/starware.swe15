\section{Specifica componenti server}
Package base per le funzionalità del server\begin{center}
	\begin{figure}[H]
		\centering \includegraphics[angle=90, height=20cm]{../img/diagrammiClassi/server.png}
		\caption{Diagramma package - server}
	\end{figure}
\end{center}\subsection{server::app}
Package che ha il compito di fornire i parametri di configurazione e avviare il web server di Quizzipedia\begin{center}
	\begin{figure}[H]
		\centering \includegraphics[scale=4, max width=\textwidth, max height=\myheight]{../img/diagrammiClassi/server/app.png}
		\caption{Diagramma package - server::app}
	\end{figure}
\end{center}\hypertarget{server::app::App}{}
\subsubsection[App]{server::app::App}
\begin{figure}[H]
	\centering
	\begin{tikzpicture}
	\umlclass{server::app::App} {--app : Express\\--config : Configuration}{+App(config : Configuration)\\+start() : void\\+config() : Express\\+getApp() : Express\\+setApp(app : Express) : void}
	\end{tikzpicture}
	\caption{Diagramma classe - server::app::App}
\end{figure}\begin{description}
\item[Descrizione] \hfill \\
Classe che si occupa di avviare il server e di invocare i middleware
\item[Utilizzo] \hfill \\
Viene utilizzata per avviare il server dell'applicazione
\item[Relazioni con altre classi] \hfill \\
\vspace{-7mm}
\begin{description}
	\item[\hyperlink{server::app::Configuration}{server::app::Configuration}] \hfill \\
	Relazione uscente, campo dati che rappresenta un riferimento a Configuration
\end{description}

\item[Attributi] \hfill \\
\vspace{-7mm}
\begin{itemize}
	\item app : Express $\rightarrow$ campo dati che rappresenta l'applicazione Express
	\item config : Configuration $\rightarrow$ campo dati che rappresenta un riferimento a Configuration
\end{itemize}

\item[Metodi] \hfill \\
\vspace{-7mm}
\begin{itemize}
	\item App(config : Configuration) $\rightarrow$ costruttore della classe\begin{itemize}
		\item config $\rightarrow$ configurazione per l'inizializzazione dell'applicazione
	\end{itemize}
	
	\item start() : void $\rightarrow$ metodo che avvia il server. Non ritorna il controllo fino a che il server è in funzione
	\item config() : Express $\rightarrow$ metodo che configura i parametri del server sulla base dell'oggetto di configurazione
	\item getApp() : Express $\rightarrow$ metodo che restituisce il valore del campo dati app
	\item setApp(app : Express) : void $\rightarrow$ metodo che cambia il valore del campo dati app\begin{itemize}
		\item app $\rightarrow$ campo dati che rappresenta l'applicazione Express
	\end{itemize}
	
\end{itemize}

\end{description}

\vspace{0.5cm}
\hypertarget{server::app::Configuration}{}
\subsubsection[Configuration]{server::app::Configuration}
\begin{figure}[H]
	\centering
	\begin{tikzpicture}
	\umlclass{server::app::Configuration} {--environment : String\\--serverHost : String\\--serverPort : Number\\--serverStaticPath : String\\--dbHost : String\\--dbName : String\\--dbUser : String\\--dbPassword : String\\--dbUri : String\\--dbPort : number}{+getEnvironment() : String\\+getServerHost() : String\\+getServerPort() : Number\\+getServerStaticPath() : String\\+getDbHost() : String\\+getDbName() : String\\+getDbUser() : String\\+getDbPassword() : String\\+getDbUri() : String\\+getDbPort() : number\\+setEnvironment(environment : String) : void\\+setServerHost(serverHost : String) : void\\+setServerPort(serverPort : Number) : void\\+setServerStaticPath(serverStaticPath : String) : void\\+setDbHost(dbHost : String) : void\\+setDbName(dbName : String) : void\\+setDbUser(dbUser : String) : void\\+setDbPassword(dbPassword : String) : void\\+setDbUri(dbUri : String) : void\\+setDbPort(dbPort : number) : void}
	\end{tikzpicture}
	\caption{Diagramma classe - server::app::Configuration}
\end{figure}\begin{description}
\item[Descrizione] \hfill \\
Classe che contiene tutti i parametri di configurazione necessari al server e all'applicazione Quizzipedia per funzionare
\item[Utilizzo] \hfill \\
Viene utilizzata per definire le configurazioni dell'applicazione, passando un oggetto di questa classe al costruttore della classe App
\item[Relazioni con altre classi] \hfill \\
\vspace{-7mm}
\begin{description}
	\item[\hyperlink{server::app::App}{server::app::App}] \hfill \\
	Relazione entrante, campo dati che rappresenta un riferimento a Configuration
\end{description}

\item[Attributi] \hfill \\
\vspace{-7mm}
\begin{itemize}
	\item environment : String $\rightarrow$ variabile d'ambiente che informa se l'applicazione deve essere eseguita in modalità development o production
	\item serverHost : String $\rightarrow$ campo dati che identifica l'Indirizzo IP dell'host
	\item serverPort : Number $\rightarrow$ campo dati che identifica la porta su cui il server deve mettersi in ascolto
	\item serverStaticPath : String $\rightarrow$ campo dati che identifica il percorso della cartella che il server utilizza per fornire file statici
	\item dbHost : String $\rightarrow$ campo dati che identifica l'host del database
	\item dbName : String $\rightarrow$ campo dati che identifica il nome del database dell'applicazione
	\item dbUser : String $\rightarrow$ campo dati che identifica l'Username per connettersi al database
	\item dbPassword : String $\rightarrow$ campo dati che identifica la password per connettersi al database
	\item dbUri : String $\rightarrow$ campo dati che identifica l'Uri di connessione al database
	\item dbPort : number $\rightarrow$ campo dati che identifica la porta su cui il server di mongodb deve mettersi in ascolto
\end{itemize}

\item[Metodi] \hfill \\
\vspace{-7mm}
\begin{itemize}
	\item getEnvironment() : String $\rightarrow$ metodo che restituisce il valore del campo dati environment
	\item getServerHost() : String $\rightarrow$ metodo che restituisce il valore del campo dati serverHost
	\item getServerPort() : Number $\rightarrow$ metodo che restituisce il valore del campo dati serverPort
	\item getServerStaticPath() : String $\rightarrow$ metodo che restituisce il valore del campo dati serverStaticPath
	\item getDbHost() : String $\rightarrow$ metodo che restituisce il valore del campo dati dbHost
	\item getDbName() : String $\rightarrow$ metodo che restituisce il valore del campo dati dbName
	\item getDbUser() : String $\rightarrow$ metodo che restituisce il valore del campo dati dbUser
	\item getDbPassword() : String $\rightarrow$ metodo che restituisce il valore del campo dati dbPassword
	\item getDbUri() : String $\rightarrow$ metodo che restituisce il valore del campo dati dbUri
	\item getDbPort() : number $\rightarrow$ metodo che restituisce il valore del campo dati dbPort
	\item setEnvironment(environment : String) : void $\rightarrow$ metodo che cambia il valore del campo dati environment\begin{itemize}
		\item environment $\rightarrow$ variabile d'ambiente che informa se l'applicazione deve essere eseguita in modalità development o production
	\end{itemize}
	
	\item setServerHost(serverHost : String) : void $\rightarrow$ metodo che cambia il valore del campo dati serverHost\begin{itemize}
		\item serverHost $\rightarrow$ campo dati che identifica l'Indirizzo IP dell'host
	\end{itemize}
	
	\item setServerPort(serverPort : Number) : void $\rightarrow$ metodo che cambia il valore del campo dati serverPort\begin{itemize}
		\item serverPort $\rightarrow$ campo dati che identifica la porta su cui il server deve mettersi in ascolto
	\end{itemize}
	
	\item setServerStaticPath(serverStaticPath : String) : void $\rightarrow$ metodo che cambia il valore del campo dati serverStaticPath\begin{itemize}
		\item serverStaticPath $\rightarrow$ campo dati che identifica il percorso della cartella che il server utilizza per fornire file statici
	\end{itemize}
	
	\item setDbHost(dbHost : String) : void $\rightarrow$ metodo che cambia il valore del campo dati dbHost\begin{itemize}
		\item dbHost $\rightarrow$ campo dati che identifica l'host del database
	\end{itemize}
	
	\item setDbName(dbName : String) : void $\rightarrow$ metodo che cambia il valore del campo dati dbName\begin{itemize}
		\item dbName $\rightarrow$ campo dati che identifica il nome del database dell'applicazione
	\end{itemize}
	
	\item setDbUser(dbUser : String) : void $\rightarrow$ metodo che cambia il valore del campo dati dbUser\begin{itemize}
		\item dbUser $\rightarrow$ campo dati che identifica l'Username per connettersi al database
	\end{itemize}
	
	\item setDbPassword(dbPassword : String) : void $\rightarrow$ metodo che cambia il valore del campo dati dbPassword\begin{itemize}
		\item dbPassword $\rightarrow$ campo dati che identifica la password per connettersi al database
	\end{itemize}
	
	\item setDbUri(dbUri : String) : void $\rightarrow$ metodo che cambia il valore del campo dati dbUri\begin{itemize}
		\item dbUri $\rightarrow$ campo dati che identifica l'Uri di connessione al database
	\end{itemize}
	
	\item setDbPort(dbPort : number) : void $\rightarrow$ metodo che cambia il valore del campo dati dbPort\begin{itemize}
		\item dbPort $\rightarrow$ campo dati che identifica la porta su cui il server di mongodb deve mettersi in ascolto
	\end{itemize}
	
\end{itemize}

\end{description}

\vspace{0.5cm}
\subsection{server::express}
Express è un framework minimale, basato sul design pattern architetturale MVC per creare applicazioni web con Node.js. Express offre funzionalità che semplificano e aumentano le potenzialità di Node.js, fornendo una migliore implementazione del sistema di routing, incrementando
le funzioni di richiesta e risposta estendendole per una maggior flessibilità, integrando nuovi middleware, ed agevolando la realizzazione delle viste.
Express non limita l’utente nella scelta del linguaggio di templating, lo aiuta a gestire le route, le request e le view\begin{center}
	\begin{figure}[H]
		\centering \includegraphics[scale=4, max width=\textwidth, max height=\myheight]{../img/diagrammiClassi/server/expressApp.png}
		\caption{Diagramma package - server::express}
	\end{figure}
\end{center}\subsection{server::middleware}
Package che si occupa di ricevere richieste, richiamare il servizio adatto e restituire le risposte\begin{center}
	\begin{figure}[H]
		\centering \includegraphics[scale=4, max width=\textwidth, max height=\myheight]{../img/diagrammiClassi/server/middleware.png}
		\caption{Diagramma package - server::middleware}
	\end{figure}
\end{center}\hypertarget{server::middleware::Router}{}
\subsubsection[Router]{server::middleware::Router}
\begin{figure}[H]
	\centering
	\begin{tikzpicture}
	\umlclass{server::middleware::Router} {--router : express.Router\\--sessionService : SessionService\\--userService : UserService\\--questionService : QuestionService\\--questionnaireService : QuestionnaireService\\--tagService : TagService\\--roleService : RoleService\\--answerService : AnswerService}{+Router(auth : Authorization, error : ErrorHandler)}
	\end{tikzpicture}
	\caption{Diagramma classe - server::middleware::Router}
\end{figure}\begin{description}
\item[Descrizione] \hfill \\
Classe che si occupa di instradare le richieste verso le relative richieste
\item[Utilizzo] \hfill \\
Si occupa di smistare la richiesta in base all’URI ricevuto e ad invocare l’opportuno servizio
\item[Relazioni con altre classi] \hfill \\
\vspace{-7mm}
\begin{description}
	\item[\hyperlink{server::service::SessionService}{server::service::SessionService}] \hfill \\
	Relazione uscente, campo dati che rappresenta un oggetto SessionService
	\item[\hyperlink{server::service::UserService}{server::service::UserService}] \hfill \\
	Relazione uscente, campo dati che rappresenta un oggetto UserService
	\item[\hyperlink{server::service::QuestionService}{server::service::QuestionService}] \hfill \\
	Relazione uscente, campo dati che rappresenta un oggetto QuestionService
	\item[\hyperlink{server::service::QuestionnaireService}{server::service::QuestionnaireService}] \hfill \\
	Relazione uscente, campo dati che rappresenta un oggetto QuestionnaireService
	\item[\hyperlink{server::service::TagService}{server::service::TagService}] \hfill \\
	Relazione uscente, campo dati che rappresenta un oggetto TagService
	\item[\hyperlink{server::service::RoleService}{server::service::RoleService}] \hfill \\
	Relazione uscente, campo dati che rappresenta un oggetto RoleService
	\item[\hyperlink{server::middleware::Loader}{server::middleware::Loader}] \hfill \\
	Relazione entrante, campo dati che rappresenta un riferimento al Middleware Router per gestire il reindirizzamento delle richieste
	\item[\hyperlink{server::service::AnswerService}{server::service::AnswerService}] \hfill \\
	Relazione uscente, campo dati che rappresenta un oggetto Answer Service
\end{description}

\item[Attributi] \hfill \\
\vspace{-7mm}
\begin{itemize}
	\item router : express.Router $\rightarrow$ campo dati che rappresenta un oggetto Router di Express
	\item sessionService : SessionService $\rightarrow$ campo dati che rappresenta un oggetto SessionService
	\item userService : UserService $\rightarrow$ campo dati che rappresenta un oggetto UserService
	\item questionService : QuestionService $\rightarrow$ campo dati che rappresenta un oggetto QuestionService
	\item questionnaireService : QuestionnaireService $\rightarrow$ campo dati che rappresenta un oggetto QuestionnaireService
	\item tagService : TagService $\rightarrow$ campo dati che rappresenta un oggetto TagService
	\item roleService : RoleService $\rightarrow$ campo dati che rappresenta un oggetto RoleService
	\item answerService : AnswerService $\rightarrow$ campo dati che rappresenta un oggetto Answer Service
\end{itemize}

\item[Metodi] \hfill \\
\vspace{-7mm}
\begin{itemize}
	\item Router(auth : Authorization, error : ErrorHandler) $\rightarrow$ costruttore della classe\begin{itemize}
		\item auth $\rightarrow$ istanza dell'oggetto Authorization
		\item error $\rightarrow$ istanza del gestore degli errori
	\end{itemize}
	
\end{itemize}

\end{description}

\vspace{0.5cm}
\hypertarget{server::middleware::Authorization}{}
\subsubsection[Authorization]{server::middleware::Authorization}
\begin{figure}[H]
	\centering
	\begin{tikzpicture}
	\umlclass{server::middleware::Authorization} {}{+requireRole(name : String) : Function\\+Authorization()}
	\end{tikzpicture}
	\caption{Diagramma classe - server::middleware::Authorization}
\end{figure}\begin{description}
\item[Descrizione] \hfill \\
Classe che si occupa dell’autorizzazione delle richieste
\item[Utilizzo] \hfill \\
Viene utilizzata per verificare i permessi dell'utente per ogni richiesta
\item[Relazioni con altre classi] \hfill \\
\vspace{-7mm}
\begin{description}
	\item[\hyperlink{server::middleware::Loader}{server::middleware::Loader}] \hfill \\
	Relazione entrante, campo dati che rappresenta un riferimento al Middleware Authorization, utilizzato per l'autorizzazione delle richieste
\end{description}

\item[Metodi] \hfill \\
\vspace{-7mm}
\begin{itemize}
	\item requireRole(name : String) : Function $\rightarrow$ metodo che verifica che l’utente autenticato sia almeno di ruolo specificato, richiamando il successivo middleware in caso affermativo, rispondendo con un errore altrimenti\begin{itemize}
		\item name $\rightarrow$ il nome del ruolo minimo richiesto
	\end{itemize}
	
	\item Authorization() $\rightarrow$ costruttore della classe
\end{itemize}

\end{description}

\vspace{0.5cm}
\hypertarget{server::middleware::Loader}{}
\subsubsection[Loader]{server::middleware::Loader}
\begin{figure}[H]
	\centering
	\begin{tikzpicture}
	\umlclass{server::middleware::Loader} {--authorization : Authorization\\--error : ErrorHandler\\--router : Router}{+Loader(app : App)}
	\end{tikzpicture}
	\caption{Diagramma classe - server::middleware::Loader}
\end{figure}\begin{description}
\item[Descrizione] \hfill \\
Classe utilizzata per istanziare tutti i middleware dell'applicazione
\item[Utilizzo] \hfill \\
Viene utilizzato per istanziare in modo nascosto all’applicazione tutti i middleware presenti nel componente server::middleware
\item[Relazioni con altre classi] \hfill \\
\vspace{-7mm}
\begin{description}
	\item[\hyperlink{server::middleware::Authorization}{server::middleware::Authorization}] \hfill \\
	Relazione uscente, campo dati che rappresenta un riferimento al Middleware Authorization, utilizzato per l'autorizzazione delle richieste
	\item[\hyperlink{server::middleware::Router}{server::middleware::Router}] \hfill \\
	Relazione uscente, campo dati che rappresenta un riferimento al Middleware Router per gestire il reindirizzamento delle richieste
	\item[\hyperlink{server::middleware::ErrorHandler}{server::middleware::ErrorHandler}] \hfill \\
	Relazione uscente, campo dati che rappresenta un riferimento al Middleware ErrorHandler, che si occupa di inoltrare le risposte d'errore al client
\end{description}

\item[Attributi] \hfill \\
\vspace{-7mm}
\begin{itemize}
	\item authorization : Authorization $\rightarrow$ campo dati che rappresenta un riferimento al Middleware Authorization, utilizzato per l'autorizzazione delle richieste
	\item error : ErrorHandler $\rightarrow$ campo dati che rappresenta un riferimento al Middleware ErrorHandler, che si occupa di inoltrare le risposte d'errore al client
	\item router : Router $\rightarrow$ campo dati che rappresenta un riferimento al Middleware Router per gestire il reindirizzamento delle richieste
\end{itemize}

\item[Metodi] \hfill \\
\vspace{-7mm}
\begin{itemize}
	\item Loader(app : App) $\rightarrow$ costruttore della classe\begin{itemize}
		\item app $\rightarrow$ applicazione su cui configurare i middleware
	\end{itemize}
	
\end{itemize}

\end{description}

\vspace{0.5cm}
\hypertarget{server::middleware::ErrorHandler}{}
\subsubsection[ErrorHandler]{server::middleware::ErrorHandler}
\begin{figure}[H]
	\centering
	\begin{tikzpicture}
	\umlclass{server::middleware::ErrorHandler} {}{+handler(req : Request, res : Response, next : Function, err : Error) : void\\+ErrorHandler()}
	\end{tikzpicture}
	\caption{Diagramma classe - server::middleware::ErrorHandler}
\end{figure}\begin{description}
\item[Descrizione] \hfill \\
Classe che gestisce gli errori generati nei controllers restituendo al client la risposta contenente il codice dell'errore verificatosi
\item[Utilizzo] \hfill \\
Questo middleware viene utilizzato per ultimo nella catena di gestione delle richieste di Express, in modo da gestire tutti gli errori generati precedentemente
\item[Relazioni con altre classi] \hfill \\
\vspace{-7mm}
\begin{description}
	\item[\hyperlink{server::middleware::Loader}{server::middleware::Loader}] \hfill \\
	Relazione entrante, campo dati che rappresenta un riferimento al Middleware ErrorHandler, che si occupa di inoltrare le risposte d'errore al client
\end{description}

\item[Metodi] \hfill \\
\vspace{-7mm}
\begin{itemize}
	\item handler(req : Request, res : Response, next : Function, err : Error) : void $\rightarrow$ metodo che gestisce l'errore generato dalla richiesta e da la relativa risposta con il codice dell'errore al client\begin{itemize}
		\item req $\rightarrow$ questo oggetto rappresenta la richiesta arrivata al server che il metodo deve gestire	
		\item res $\rightarrow$ questo oggetto rappresenta la risposta che il server dovrà inviare al termine dell'elaborazione	
		\item next $\rightarrow$ questo parametro rappresenta la callback che il metodo dovrà chiamare al termine dell’elaborazione	
		\item err $\rightarrow$ questo parametro rappresenta l'oggetto  dell'errore
	\end{itemize}
	
	\item ErrorHandler() $\rightarrow$ costruttore della classe
\end{itemize}

\end{description}

\vspace{0.5cm}
\subsection{server::data}
Package contenente le componenti che gestiscono i dati utilizzati dall'applicazione e l'interfacciamento con il database\begin{center}
	\begin{figure}[H]
		\centering \includegraphics[scale=4, max width=\textwidth, max height=\myheight]{../img/diagrammiClassi/server/data.png}
		\caption{Diagramma package - server::data}
	\end{figure}
\end{center}\hypertarget{server::data::Tag}{}
\subsubsection[Tag]{server::data::Tag}
\begin{figure}[H]
	\centering
	\begin{tikzpicture}
	\umlclass{server::data::Tag} {--name : String\\--description : String}{+getName() : String\\+getDescription() : String\\+setName(name : String) : void\\+setDescription(description : String) : void}
	\end{tikzpicture}
	\caption{Diagramma classe - server::data::Tag}
\end{figure}\begin{description}
\item[Descrizione] \hfill \\
Classe che rappresenta una String contenente l'argomento da assegnare alle domande o ai questionari
\item[Relazioni con altre classi] \hfill \\
\vspace{-7mm}
\begin{description}
	\item[\hyperlink{server::data::Question}{server::data::Question}] \hfill \\
	Relazione entrante, campo dati che rappresenta l'insieme di riferimenti agli argomenti associati alla domanda
	\item[\hyperlink{server::data::Questionnaire}{server::data::Questionnaire}] \hfill \\
	Relazione entrante, campo dati che rappresenta l'insieme di riferimenti agli argomenti associati al questionario
\end{description}

\item[Attributi] \hfill \\
\vspace{-7mm}
\begin{itemize}
	\item name : String $\rightarrow$ campo dati contenente le parole dell'argomento separate da '-'
	\item description : String $\rightarrow$ campo dati che rappresenta una breve descrizione dell'attributo
\end{itemize}

\item[Metodi] \hfill \\
\vspace{-7mm}
\begin{itemize}
	\item getName() : String $\rightarrow$ restituisce il nome identificativo dell'argomento
	\item getDescription() : String $\rightarrow$ restituisce la descrizione dell'argomento
	\item setName(name : String) : void $\rightarrow$ modifica il nome identificativo dell'argomento\begin{itemize}
		\item name $\rightarrow$ campo dati contenente le parole dell'argomento separate da '-'
	\end{itemize}
	
	\item setDescription(description : String) : void $\rightarrow$ imposta la descrizione dell'argomento\begin{itemize}
		\item description $\rightarrow$ campo dati che rappresenta una breve descrizione dell'attributo
	\end{itemize}
	
\end{itemize}

\end{description}

\vspace{0.5cm}
\hypertarget{server::data::Questionnaire}{}
\subsubsection[Questionnaire]{server::data::Questionnaire}
\begin{figure}[H]
	\centering
	\begin{tikzpicture}
	\umlclass{server::data::Questionnaire} {--title : String\\--tags : Tag []\\--author : User\\--questions : Question []}{+getTitle() : String\\+getTags() : Tag []\\+getAuthor() : User\\+getQuestions() : Question []\\+setTitle(title : String) : void\\+setTags(tags : Tag []) : void\\+setQuestions(questions : Question []) : void}
	\end{tikzpicture}
	\caption{Diagramma classe - server::data::Questionnaire}
\end{figure}\begin{description}
\item[Descrizione] \hfill \\
Classe che rappresenta un questionario
\item[Relazioni con altre classi] \hfill \\
\vspace{-7mm}
\begin{description}
	\item[\hyperlink{server::data::Question}{server::data::Question}] \hfill \\
	Relazione uscente, campo dati che rappresenta l'insieme di riferimenti alle domande del questionario
	\item[\hyperlink{server::data::Tag}{server::data::Tag}] \hfill \\
	Relazione uscente, campo dati che rappresenta l'insieme di riferimenti agli argomenti associati al questionario
	\item[\hyperlink{server::data::User}{server::data::User}] \hfill \\
	Relazione uscente, campo dati che rappresenta il riferimento all'autore del questionario
	\item[\hyperlink{server::data::Answer}{server::data::Answer}] \hfill \\
	Relazione entrante, campo dati che rappresenta il questionario della domanda a cui la risposta fa riferimento
\end{description}

\item[Attributi] \hfill \\
\vspace{-7mm}
\begin{itemize}
	\item title : String $\rightarrow$ campo dati che rappresenta il titolo del questionario
	\item tags : Tag [] $\rightarrow$ campo dati che rappresenta l'insieme di riferimenti agli argomenti associati al questionario
	\item author : User $\rightarrow$ campo dati che rappresenta il riferimento all'autore del questionario
	\item questions : Question [] $\rightarrow$ campo dati che rappresenta l'insieme di riferimenti alle domande del questionario
\end{itemize}

\item[Metodi] \hfill \\
\vspace{-7mm}
\begin{itemize}
	\item getTitle() : String $\rightarrow$ restituisce il titolo del questionario
	\item getTags() : Tag [] $\rightarrow$ restituisce la lista dei riferimenti agli argomenti del questionario
	\item getAuthor() : User $\rightarrow$ restituisce il riferimento all'autore del questionario
	\item getQuestions() : Question [] $\rightarrow$ restituisce la lista dei riferimenti alle domande del questionario
	\item setTitle(title : String) : void $\rightarrow$ imposta il titolo del questionario\begin{itemize}
		\item title $\rightarrow$ campo dati che rappresenta il titolo del questionario
	\end{itemize}
	
	\item setTags(tags : Tag []) : void $\rightarrow$ imposta la lista dei riferimenti agli argomenti del questionario\begin{itemize}
		\item tags $\rightarrow$ campo dati che rappresenta l'insieme di riferimenti agli argomenti associati al questionario
	\end{itemize}
	
	\item setQuestions(questions : Question []) : void $\rightarrow$ imposta la lista dei riferimenti alle domande del questionario\begin{itemize}
		\item questions $\rightarrow$ campo dati che rappresenta l'insieme di riferimenti alle domande del questionario
	\end{itemize}
	
\end{itemize}

\end{description}

\vspace{0.5cm}
\hypertarget{server::data::Question}{}
\subsubsection[Question]{server::data::Question}
\begin{figure}[H]
	\centering
	\begin{tikzpicture}
	\umlclass{server::data::Question} {--body : String\\--author : User\\--tags : Tag []}{+getBody() : String\\+getAuthor() : User\\+getTags() : Tag []\\+setBody(body : String) : void\\+setTags(tags : Tag []) : void}
	\end{tikzpicture}
	\caption{Diagramma classe - server::data::Question}
\end{figure}\begin{description}
\item[Descrizione] \hfill \\
Classe base comune a tutti i tipi di domanda
\item[Relazioni con altre classi] \hfill \\
\vspace{-7mm}
\begin{description}
	\item[\hyperlink{server::data::Tag}{server::data::Tag}] \hfill \\
	Relazione uscente, campo dati che rappresenta l'insieme di riferimenti agli argomenti associati alla domanda
	\item[\hyperlink{server::data::User}{server::data::User}] \hfill \\
	Relazione uscente, campo dati che rappresenta il riferimento al docente che ha creato la domanda
	\item[\hyperlink{server::data::Questionnaire}{server::data::Questionnaire}] \hfill \\
	Relazione entrante, campo dati che rappresenta l'insieme di riferimenti alle domande del questionario
	\item[\hyperlink{server::data::Answer}{server::data::Answer}] \hfill \\
	Relazione entrante, campo dati che rappresenta la domanda a cui si riferisce la risposta
\end{description}

\item[Attributi] \hfill \\
\vspace{-7mm}
\begin{itemize}
	\item body : String $\rightarrow$ campo dati che rappresenta il QML del corpo della domanda
	\item author : User $\rightarrow$ campo dati che rappresenta il riferimento al docente che ha creato la domanda
	\item tags : Tag [] $\rightarrow$ campo dati che rappresenta l'insieme di riferimenti agli argomenti associati alla domanda
\end{itemize}

\item[Metodi] \hfill \\
\vspace{-7mm}
\begin{itemize}
	\item getBody() : String $\rightarrow$ restituisce il corpo della domanda in formato QML
	\item getAuthor() : User $\rightarrow$ restituisce il riferimento all'autore della domanda
	\item getTags() : Tag [] $\rightarrow$ restituisce la lista dei riferimenti agli argomenti della domanda
	\item setBody(body : String) : void $\rightarrow$ imposta il corto della domanda in formato QML\begin{itemize}
		\item body $\rightarrow$ campo dati che rappresenta il QML del corpo della domanda
	\end{itemize}
	
	\item setTags(tags : Tag []) : void $\rightarrow$ imposta la lista dei riferimenti agli argomenti della domanda\begin{itemize}
		\item tags $\rightarrow$ campo dati che rappresenta l'insieme di riferimenti agli argomenti associati alla domanda
	\end{itemize}
	
\end{itemize}

\end{description}

\vspace{0.5cm}
\hypertarget{server::data::Role}{}
\subsubsection[Role]{server::data::Role}
\begin{figure}[H]
	\centering
	\begin{tikzpicture}
	\umlclass{server::data::Role} {--name : String}{+greaterThan(role : Role) : boolean\\+equalTo(role : Role) : boolean\\+getName() : String}
	\end{tikzpicture}
	\caption{Diagramma classe - server::data::Role}
\end{figure}\begin{description}
\item[Descrizione] \hfill \\
Classe che rappresenta un ruolo all'interno dell'applicazione
\item[Relazioni con altre classi] \hfill \\
\vspace{-7mm}
\begin{description}
	\item[\hyperlink{server::data::User}{server::data::User}] \hfill \\
	Relazione entrante, campo dati che rappresenta il riferimento al Role dell'utente nell'applicazione
\end{description}

\item[Attributi] \hfill \\
\vspace{-7mm}
\begin{itemize}
	\item name : String $\rightarrow$ campo dati che identifica la tipologia di utente
\end{itemize}

\item[Metodi] \hfill \\
\vspace{-7mm}
\begin{itemize}
	\item greaterThan(role : Role) : boolean $\rightarrow$ metodo che controlla se il ruolo è superiore (e non uguale) a quello passato\begin{itemize}
		\item role $\rightarrow$ l'altro ruolo da comparare
	\end{itemize}
	
	\item equalTo(role : Role) : boolean $\rightarrow$ metodo che controlla se il ruolo è uguale a quello passato\begin{itemize}
		\item role $\rightarrow$ l'altro ruolo da comparare
	\end{itemize}
	
	\item getName() : String $\rightarrow$ restituisce il nome identificativo del ruolo
\end{itemize}

\end{description}

\vspace{0.5cm}
\hypertarget{server::data::User}{}
\subsubsection[User]{server::data::User}
\begin{figure}[H]
	\centering
	\begin{tikzpicture}
	\umlclass{server::data::User} {--fullName : String\\--role : Role\\--userName : String\\--password : String\\--isActive : boolean}{+hasPassword(rawPassword : String) : boolean\\+getFullName() : String\\+getRole() : Role\\+getUserName() : String\\+getIsActive() : boolean\\+setFullName(fullName : String) : void\\+setRole(role : Role) : void\\+setUserName(userName : String) : void\\+setPassword(password : String) : void\\+setIsActive(isActive : boolean) : void}
	\end{tikzpicture}
	\caption{Diagramma classe - server::data::User}
\end{figure}\begin{description}
\item[Descrizione] \hfill \\
Classe che rappresenta un utente dell'applicazione
\item[Relazioni con altre classi] \hfill \\
\vspace{-7mm}
\begin{description}
	\item[\hyperlink{server::data::Question}{server::data::Question}] \hfill \\
	Relazione entrante, campo dati che rappresenta il riferimento al docente che ha creato la domanda
	\item[\hyperlink{server::data::Questionnaire}{server::data::Questionnaire}] \hfill \\
	Relazione entrante, campo dati che rappresenta il riferimento all'autore del questionario
	\item[\hyperlink{server::data::Role}{server::data::Role}] \hfill \\
	Relazione uscente, campo dati che rappresenta il riferimento al Role dell'utente nell'applicazione
	\item[\hyperlink{server::data::Answer}{server::data::Answer}] \hfill \\
	Relazione entrante, campo dati che rappresenta l'autore della risposta
\end{description}

\item[Attributi] \hfill \\
\vspace{-7mm}
\begin{itemize}
	\item fullName : String $\rightarrow$ campo dati che rappresenta il nome e cognome dell'utente
	\item role : Role $\rightarrow$ campo dati che rappresenta il riferimento al Role dell'utente nell'applicazione
	\item userName : String $\rightarrow$ campo dati che rappresenta l'Username univoco con la quale, in combinazione con la password, l'utente accede al sistema
	\item password : String $\rightarrow$ campo dati contenente l'hash della password utilizzata per l'accesso
	\item isActive : boolean $\rightarrow$ campo dati che definisce se l'utente è attivo all'interno del sistema o se è stato disattivato
\end{itemize}

\item[Metodi] \hfill \\
\vspace{-7mm}
\begin{itemize}
	\item hasPassword(rawPassword : String) : boolean $\rightarrow$ metodo che controlla se la password dell'utente è uguale a quella passata\begin{itemize}
		\item rawPassword $\rightarrow$ la password in chiaro
	\end{itemize}
	
	\item getFullName() : String $\rightarrow$ restituisce la stringa rappresentante il nome e cognome (ed eventuale secondo nome) dell'utente
	\item getRole() : Role $\rightarrow$ restituisce il riferimento al ruolo dell'utente
	\item getUserName() : String $\rightarrow$ restituisce il nome identificativo dell'utente utilizzato per l'accesso al sistema
	\item getIsActive() : boolean $\rightarrow$ restituisce un booleano che rappresenta lo stato dell'utente: false significa che l'utente non è abilitato (a tutti gli effetti è rimosso dal sistema), altrimenti l'utente è abilitato
	\item setFullName(fullName : String) : void $\rightarrow$ imposta il nome completo (nome, secondo nome, cognome) dell'utente\begin{itemize}
		\item fullName $\rightarrow$ campo dati che rappresenta il nome e cognome dell'utente
	\end{itemize}
	
	\item setRole(role : Role) : void $\rightarrow$ imposta il riferimento al ruolo dell'utente\begin{itemize}
		\item role $\rightarrow$ campo dati che rappresenta il riferimento al Role dell'utente nell'applicazione
	\end{itemize}
	
	\item setUserName(userName : String) : void $\rightarrow$ imposta il nome identificativo dell'utente\begin{itemize}
		\item userName $\rightarrow$ campo dati che rappresenta l'Username univoco con la quale, in combinazione con la password, l'utente accede al sistema
	\end{itemize}
	
	\item setPassword(password : String) : void $\rightarrow$ imposta la password dell'utente\begin{itemize}
		\item password $\rightarrow$ campo dati contenente l'hash della password utilizzata per l'accesso
	\end{itemize}
	
	\item setIsActive(isActive : boolean) : void $\rightarrow$ imposta lo stato di attività dell'utente (true = abilitato, false = disabilitato)\begin{itemize}
		\item isActive $\rightarrow$ campo dati che definisce se l'utente è attivo all'interno del sistema o se è stato disattivato
	\end{itemize}
	
\end{itemize}

\end{description}

\vspace{0.5cm}
\hypertarget{server::data::Answer}{}
\subsubsection[Answer]{server::data::Answer}
\begin{figure}[H]
	\centering
	\begin{tikzpicture}
	\umlclass{server::data::Answer} {--question : Question\\--questionnaire : Questionnaire\\--author : User\\--score : Number}{+getQuestion() : Question\\+getQuestionnaire() : Questionnaire\\+getAuthor() : User\\+getScore() : Number}
	\end{tikzpicture}
	\caption{Diagramma classe - server::data::Answer}
\end{figure}\begin{description}
\item[Descrizione] \hfill \\
Classe che rappresenta una risposta ad una domanda di un questionario all'interno dell'applicazione
\item[Relazioni con altre classi] \hfill \\
\vspace{-7mm}
\begin{description}
	\item[\hyperlink{server::data::User}{server::data::User}] \hfill \\
	Relazione uscente, campo dati che rappresenta l'autore della risposta
	\item[\hyperlink{server::data::Question}{server::data::Question}] \hfill \\
	Relazione uscente, campo dati che rappresenta la domanda a cui si riferisce la risposta
	\item[\hyperlink{server::data::Questionnaire}{server::data::Questionnaire}] \hfill \\
	Relazione uscente, campo dati che rappresenta il questionario della domanda a cui la risposta fa riferimento
\end{description}

\item[Attributi] \hfill \\
\vspace{-7mm}
\begin{itemize}
	\item question : Question $\rightarrow$ campo dati che rappresenta la domanda a cui si riferisce la risposta
	\item questionnaire : Questionnaire $\rightarrow$ campo dati che rappresenta il questionario della domanda a cui la risposta fa riferimento
	\item author : User $\rightarrow$ campo dati che rappresenta l'autore della risposta
	\item score : Number $\rightarrow$ campo dati che rappresenta il punteggio ottenuto per la risposta (valore razionale tra 0.0 e 1.0 compresi)
\end{itemize}

\item[Metodi] \hfill \\
\vspace{-7mm}
\begin{itemize}
	\item getQuestion() : Question $\rightarrow$ restituisce il riferimento alla domanda alla quale la risposta è stata data all'interno del questionario
	\item getQuestionnaire() : Questionnaire $\rightarrow$ ritorna il riferimento al questionario all'interno della quale si trova la domanda a cui questa risposta è stata data
	\item getAuthor() : User $\rightarrow$ restituisce il riferimento all'autore della risposta
	\item getScore() : Number $\rightarrow$ ritorna il punteggio normalizzato ottenuto da questa risposta (valore razionale tra 0.0 e 1.0 compresi)
\end{itemize}

\end{description}

\vspace{0.5cm}
\subsection{server::service}
Package contenente le classi che implementano tutti i servizi offerti dall'applicazione lato server\begin{center}
	\begin{figure}[H]
		\centering \includegraphics[scale=4, max width=\textwidth, max height=\myheight]{../img/diagrammiClassi/server/service.png}
		\caption{Diagramma package - server::service}
	\end{figure}
\end{center}\hypertarget{server::service::UserService}{}
\subsubsection[UserService]{server::service::UserService}
\begin{figure}[H]
	\centering
	\begin{tikzpicture}
	\umlclass{server::service::UserService} {}{+getByID(req : Request, res : Response, next : Function) : void\\+getMe(req : Request, res : Response, next : Function) : void\\+get(req : Request, res : Response, next : Function) : void\\+new(req : Request, res : Response, next : Function) : void\\+modify(req : Request, res : Response, next : Function) : void\\+delete(req : Request, res : Response, next : Function) : void\\+modifyMe(req : Request, res : Response, next : Function) : void\\+UserService()}
	\end{tikzpicture}
	\caption{Diagramma classe - server::service::UserService}
\end{figure}\begin{description}
\item[Descrizione] \hfill \\
Classe che si occupa della operazioni di inserimento, modifica e rimozione di account utenti, sfruttando la classe server::data::User per accedere ai dati persistenti nel database.
\item[Utilizzo] \hfill \\
Fornisce i dati personali degli utenti a chi ne ha il permesso di accesso ed esegue operazioni di aggiunta, modifica, rimozione e cambio di ruolo per gli utenti del sistema.
\item[Relazioni con altre classi] \hfill \\
\vspace{-7mm}
\begin{description}
	\item[\hyperlink{server::middleware::Router}{server::middleware::Router}] \hfill \\
	Relazione entrante, campo dati che rappresenta un oggetto UserService
\end{description}

\item[Metodi] \hfill \\
\vspace{-7mm}
\begin{itemize}
	\item getByID(req : Request, res : Response, next : Function) : void $\rightarrow$ metodo che ritorna al client un Json contenente l'utente specifico identificato nella richiesta http\begin{itemize}
		\item req $\rightarrow$ questo oggetto rappresenta la richiesta arrivata al server che il metodo deve gestire
		\item res $\rightarrow$ questo oggetto rappresenta la risposta che il server dovrà inviare al termine dell'elaborazione
		\item next $\rightarrow$ questo parametro rappresenta la callback che il metodo dovrà chiamare al termine dell’elaborazione
	\end{itemize}
	
	\item getMe(req : Request, res : Response, next : Function) : void $\rightarrow$ metodo che restituisce al client un Json con i dati relativi all'utente loggato\begin{itemize}
		\item req $\rightarrow$ questo oggetto rappresenta la richiesta arrivata al server che il metodo deve gestire
		\item res $\rightarrow$ questo oggetto rappresenta la risposta che il server dovrà inviare al termine dell'elaborazione
		\item next $\rightarrow$ questo parametro rappresenta la callback che il metodo dovrà chiamare al termine dell’elaborazione
	\end{itemize}
	
	\item get(req : Request, res : Response, next : Function) : void $\rightarrow$ metodo che invia al client la lista degli utenti attraverso un Json\begin{itemize}
		\item req $\rightarrow$ questo oggetto rappresenta la richiesta arrivata al server che il metodo deve gestire
		\item res $\rightarrow$ questo oggetto rappresenta la risposta che il server dovrà inviare al termine dell'elaborazione
		\item next $\rightarrow$ questo parametro rappresenta la callback che il metodo dovrà chiamare al termine dell’elaborazione
	\end{itemize}
	
	\item new(req : Request, res : Response, next : Function) : void $\rightarrow$ metodo che aggiunge un nuovo utente al database\begin{itemize}
		\item req $\rightarrow$ questo oggetto rappresenta la richiesta arrivata al server che il metodo deve gestire
		\item res $\rightarrow$ questo oggetto rappresenta la risposta che il server dovrà inviare al termine dell'elaborazione
		\item next $\rightarrow$ questo parametro rappresenta la callback che il metodo dovrà chiamare al termine dell’elaborazione
	\end{itemize}
	
	\item modify(req : Request, res : Response, next : Function) : void $\rightarrow$ metodo che modifica i dati dell'utente specificato nella richiesta http\begin{itemize}
		\item req $\rightarrow$ questo oggetto rappresenta la richiesta arrivata al server che il metodo deve gestire
		\item res $\rightarrow$ questo oggetto rappresenta la risposta che il server dovrà inviare al termine dell'elaborazione
		\item next $\rightarrow$ questo parametro rappresenta la callback che il metodo dovrà chiamare al termine dell’elaborazione
	\end{itemize}
	
	\item delete(req : Request, res : Response, next : Function) : void $\rightarrow$ metodo che elimina un utente dal database\begin{itemize}
		\item req $\rightarrow$ questo oggetto rappresenta la richiesta arrivata al server che il metodo deve gestire
		\item res $\rightarrow$ questo oggetto rappresenta la risposta che il server dovrà inviare al termine dell'elaborazione
		\item next $\rightarrow$ questo parametro rappresenta la callback che il metodo dovrà chiamare al termine dell’elaborazione
	\end{itemize}
	
	\item modifyMe(req : Request, res : Response, next : Function) : void $\rightarrow$ metodo che modifica i dati dell'utente connesso al sistema se presente\begin{itemize}
		\item req $\rightarrow$ questo oggetto rappresenta la richiesta arrivata al server che il metodo deve gestire
		\item res $\rightarrow$ questo oggetto rappresenta la risposta che il server dovrà inviare al termine dell'elaborazione
		\item next $\rightarrow$ questo parametro rappresenta la callback che il metodo dovrà chiamare al termine dell’elaborazione
	\end{itemize}
	
	\item UserService() $\rightarrow$ costruttore della classe
\end{itemize}

\end{description}

\vspace{0.5cm}
\hypertarget{server::service::QuestionnaireService}{}
\subsubsection[QuestionnaireService]{server::service::QuestionnaireService}
\begin{figure}[H]
	\centering
	\begin{tikzpicture}
	\umlclass{server::service::QuestionnaireService} {}{+getByID(req : Request, res : Response, next : Function) : void\\+get(req : Request, res : Response, next : Function) : void\\+new(req : Request, res : Response, next : Function) : void\\+modify(req : Request, res : Response, next : Function) : void\\+delete(req : Request, res : Response, next : Function) : void\\+QuestionnaireService()}
	\end{tikzpicture}
	\caption{Diagramma classe - server::service::QuestionnaireService}
\end{figure}\begin{description}
\item[Descrizione] \hfill \\
Classe che si occupa di gestire questionari, sfruttando la classe server::data::Questionnaire per accedere ai dati persistenti nel database.
\item[Utilizzo] \hfill \\
Offre metodi per restituire questionari. Permette inoltre ad un docente di effettuare l'inserimento, la modifica, l'eliminazione di questionari
\item[Relazioni con altre classi] \hfill \\
\vspace{-7mm}
\begin{description}
	\item[\hyperlink{server::middleware::Router}{server::middleware::Router}] \hfill \\
	Relazione entrante, campo dati che rappresenta un oggetto QuestionnaireService
\end{description}

\item[Metodi] \hfill \\
\vspace{-7mm}
\begin{itemize}
	\item getByID(req : Request, res : Response, next : Function) : void $\rightarrow$ metodo che ritorna al client un Json contenente il questionario specifico richiesto identificato nella richiesta http\begin{itemize}
		\item req $\rightarrow$ questo oggetto rappresenta la richiesta arrivata al server che il metodo deve gestire
		\item res $\rightarrow$ questo oggetto rappresenta la risposta che il server dovrà inviare al termine dell'elaborazione
		\item next $\rightarrow$ questo parametro rappresenta la callback che il metodo dovrà chiamare al termine dell’elaborazione
	\end{itemize}
	
	\item get(req : Request, res : Response, next : Function) : void $\rightarrow$ metodo che invia al client una lista di questionari attraverso un Json\begin{itemize}
		\item req $\rightarrow$ questo oggetto rappresenta la richiesta arrivata al server che il metodo deve gestire
		\item res $\rightarrow$ questo oggetto rappresenta la risposta che il server dovrà inviare al termine dell'elaborazione
		\item next $\rightarrow$ questo parametro rappresenta la callback che il metodo dovrà chiamare al termine dell’elaborazione
	\end{itemize}
	
	\item new(req : Request, res : Response, next : Function) : void $\rightarrow$ metodo che aggiunge un nuovo questionario al database\begin{itemize}
		\item req $\rightarrow$ questo oggetto rappresenta la richiesta arrivata al server che il metodo deve gestire
		\item res $\rightarrow$ questo oggetto rappresenta la risposta che il server dovrà inviare al termine dell'elaborazione
		\item next $\rightarrow$ questo parametro rappresenta la callback che il metodo dovrà chiamare al termine dell’elaborazione
	\end{itemize}
	
	\item modify(req : Request, res : Response, next : Function) : void $\rightarrow$ metodo che modifica un questionario specificato nella richiesta http\begin{itemize}
		\item req $\rightarrow$ questo oggetto rappresenta la richiesta arrivata al server che il metodo deve gestire
		\item res $\rightarrow$ questo oggetto rappresenta la risposta che il server dovrà inviare al termine dell'elaborazione
		\item next $\rightarrow$ questo parametro rappresenta la callback che il metodo dovrà chiamare al termine dell’elaborazione
	\end{itemize}
	
	\item delete(req : Request, res : Response, next : Function) : void $\rightarrow$ metodo che cancella un questionario specifico dal database\begin{itemize}
		\item req $\rightarrow$ questo oggetto rappresenta la richiesta arrivata al server che il metodo deve gestire
		\item res $\rightarrow$ questo oggetto rappresenta la risposta che il server dovrà inviare al termine dell'elaborazione
		\item next $\rightarrow$ questo parametro rappresenta la callback che il metodo dovrà chiamare al termine dell’elaborazione
	\end{itemize}
	
	\item QuestionnaireService() $\rightarrow$ costruttore della classe
\end{itemize}

\end{description}

\vspace{0.5cm}
\hypertarget{server::service::QuestionService}{}
\subsubsection[QuestionService]{server::service::QuestionService}
\begin{figure}[H]
	\centering
	\begin{tikzpicture}
	\umlclass{server::service::QuestionService} {}{+get(req : Request, res : Response, next : Function) : void\\+getByID(req : Request, res : Response, next : Function) : void\\+new(req : Request, res : Response, next : Function) : void\\+modify(req : Request, res : Response, next : Function) : void\\+delete(req : Request, res : Response, next : Function) : void\\+QuestionService()}
	\end{tikzpicture}
	\caption{Diagramma classe - server::service::QuestionService}
\end{figure}\begin{description}
\item[Descrizione] \hfill \\
Classe che si occupa di gestire domande, sfruttando la classe server::data::Question per accedere ai dati persistenti nel database
\item[Utilizzo] \hfill \\
Offre metodi per restituire le domande. Permette inoltre ad un docente di effettuare l'inserimento, la modifica, l'eliminazione di domande
\item[Relazioni con altre classi] \hfill \\
\vspace{-7mm}
\begin{description}
	\item[\hyperlink{server::middleware::Router}{server::middleware::Router}] \hfill \\
	Relazione entrante, campo dati che rappresenta un oggetto QuestionService
\end{description}

\item[Metodi] \hfill \\
\vspace{-7mm}
\begin{itemize}
	\item get(req : Request, res : Response, next : Function) : void $\rightarrow$ metodo che invia al client una lista di domande attraverso un Json\begin{itemize}
		\item req $\rightarrow$ questo oggetto rappresenta la richiesta arrivata al server che il metodo deve gestire
		\item res $\rightarrow$ questo oggetto rappresenta la risposta che il server dovrà inviare al termine dell'elaborazione
		\item next $\rightarrow$ questo parametro rappresenta la callback che il metodo dovrà chiamare al termine dell’elaborazione
	\end{itemize}
	
	\item getByID(req : Request, res : Response, next : Function) : void $\rightarrow$ metodo che ritorna al client un Json contenente la domanda specifica identificata nella richiesta http\begin{itemize}
		\item req $\rightarrow$ questo oggetto rappresenta la richiesta arrivata al server che il metodo deve gestire
		\item res $\rightarrow$ questo oggetto rappresenta la risposta che il server dovrà inviare al termine dell'elaborazione
		\item next $\rightarrow$ questo parametro rappresenta la callback che il metodo dovrà chiamare al termine dell’elaborazione
	\end{itemize}
	
	\item new(req : Request, res : Response, next : Function) : void $\rightarrow$ metodo che aggiunge una nuova domanda al database\begin{itemize}
		\item req $\rightarrow$ questo oggetto rappresenta la richiesta arrivata al server che il metodo deve gestire
		\item res $\rightarrow$ questo oggetto rappresenta la risposta che il server dovrà inviare al termine dell'elaborazione
		\item next $\rightarrow$ questo parametro rappresenta la callback che il metodo dovrà chiamare al termine dell’elaborazione
	\end{itemize}
	
	\item modify(req : Request, res : Response, next : Function) : void $\rightarrow$ metodo che modifica una domanda specificato nella richiesta http\begin{itemize}
		\item req $\rightarrow$ questo oggetto rappresenta la richiesta arrivata al server che il metodo deve gestire
		\item res $\rightarrow$ questo oggetto rappresenta la risposta che il server dovrà inviare al termine dell'elaborazione
		\item next $\rightarrow$ questo parametro rappresenta la callback che il metodo dovrà chiamare al termine dell’elaborazione
	\end{itemize}
	
	\item delete(req : Request, res : Response, next : Function) : void $\rightarrow$ metodo che elimina una domanda selezionata dal database\begin{itemize}
		\item req $\rightarrow$ questo oggetto rappresenta la richiesta arrivata al server che il metodo deve gestire
		\item res $\rightarrow$ questo oggetto rappresenta la risposta che il server dovrà inviare al termine dell'elaborazione
		\item next $\rightarrow$ questo parametro rappresenta la callback che il metodo dovrà chiamare al termine dell’elaborazione
	\end{itemize}
	
	\item QuestionService() $\rightarrow$ costruttore della classe
\end{itemize}

\end{description}

\vspace{0.5cm}
\hypertarget{server::service::TagService}{}
\subsubsection[TagService]{server::service::TagService}
\begin{figure}[H]
	\centering
	\begin{tikzpicture}
	\umlclass{server::service::TagService} {}{+get(req : Request, res : Response, next : Function) : void\\+getByID(req : Request, res : Response, next : Function) : void\\+new(req : Request, res : Response, next : Function) : void\\+modify(req : Request, res : Response, next : Function) : void\\+delete(req : Request, res : Response, next : Function) : void\\+TagService()}
	\end{tikzpicture}
	\caption{Diagramma classe - server::service::TagService}
\end{figure}\begin{description}
\item[Descrizione] \hfill \\
Classe che si occupa di gestire gli argomenti, sfruttando la classe server::data::Tag per accedere ai dati persistenti nel database
\item[Utilizzo] \hfill \\
Offre metodi per restituire gli argomenti presenti. Permette inoltre ad un docente di effettuare l'inserimento, la modifica, l'eliminazione di argomenti
\item[Relazioni con altre classi] \hfill \\
\vspace{-7mm}
\begin{description}
	\item[\hyperlink{server::middleware::Router}{server::middleware::Router}] \hfill \\
	Relazione entrante, campo dati che rappresenta un oggetto TagService
\end{description}

\item[Metodi] \hfill \\
\vspace{-7mm}
\begin{itemize}
	\item get(req : Request, res : Response, next : Function) : void $\rightarrow$ metodo che invia al client la lista degli argomenti attraverso un Json\begin{itemize}
		\item req $\rightarrow$ questo oggetto rappresenta la richiesta arrivata al server che il metodo deve gestire
		\item res $\rightarrow$ questo oggetto rappresenta la risposta che il server dovrà inviare al termine dell'elaborazione
		\item next $\rightarrow$ questo parametro rappresenta la callback che il metodo dovrà chiamare al termine dell’elaborazione
	\end{itemize}
	
	\item getByID(req : Request, res : Response, next : Function) : void $\rightarrow$ metodo che ritorna al client un Json contenente l'argomento specifico identificato nella richiesta http\begin{itemize}
		\item req $\rightarrow$ questo oggetto rappresenta la richiesta arrivata al server che il metodo deve gestire
		\item res $\rightarrow$ questo oggetto rappresenta la risposta che il server dovrà inviare al termine dell'elaborazione
		\item next $\rightarrow$ questo parametro rappresenta la callback che il metodo dovrà chiamare al termine dell’elaborazione
	\end{itemize}
	
	\item new(req : Request, res : Response, next : Function) : void $\rightarrow$ metodo che aggiunge un nuovo argomento al database\begin{itemize}
		\item req $\rightarrow$ questo oggetto rappresenta la richiesta arrivata al server che il metodo deve gestire
		\item res $\rightarrow$ questo oggetto rappresenta la risposta che il server dovrà inviare al termine dell'elaborazione
		\item next $\rightarrow$ questo parametro rappresenta la callback che il metodo dovrà chiamare al termine dell’elaborazione
	\end{itemize}
	
	\item modify(req : Request, res : Response, next : Function) : void $\rightarrow$ metodo che modifica un argomento specificato nella richiesta http\begin{itemize}
		\item req $\rightarrow$ questo oggetto rappresenta la richiesta arrivata al server che il metodo deve gestire
		\item res $\rightarrow$ questo oggetto rappresenta la risposta che il server dovrà inviare al termine dell'elaborazione
		\item next $\rightarrow$ questo parametro rappresenta la callback che il metodo dovrà chiamare al termine dell’elaborazione
	\end{itemize}
	
	\item delete(req : Request, res : Response, next : Function) : void $\rightarrow$ metodo che elimina un argomento specifico dal database\begin{itemize}
		\item req $\rightarrow$ questo oggetto rappresenta la richiesta arrivata al server che il metodo deve gestire
		\item res $\rightarrow$ questo oggetto rappresenta la risposta che il server dovrà inviare al termine dell'elaborazione
		\item next $\rightarrow$ questo parametro rappresenta la callback che il metodo dovrà chiamare al termine dell’elaborazione
	\end{itemize}
	
	\item TagService() $\rightarrow$ costruttore della classe
\end{itemize}

\end{description}

\vspace{0.5cm}
\hypertarget{server::service::SessionService}{}
\subsubsection[SessionService]{server::service::SessionService}
\begin{figure}[H]
	\centering
	\begin{tikzpicture}
	\umlclass{server::service::SessionService} {}{+new(req : Request, res : Response, next : Function) : void\\+delete(req : Request, res : Response, next : Function) : void\\+SessionService()}
	\end{tikzpicture}
	\caption{Diagramma classe - server::service::SessionService}
\end{figure}\begin{description}
\item[Descrizione] \hfill \\
Classe che si occupa della gestione della sessione dell'utente, sfruttando la classe server::data::User per accedere ai dati persistenti nel database
\item[Utilizzo] \hfill \\
Viene utilizzata per gestire il login e logout dell'utente
\item[Relazioni con altre classi] \hfill \\
\vspace{-7mm}
\begin{description}
	\item[\hyperlink{server::middleware::Router}{server::middleware::Router}] \hfill \\
	Relazione entrante, campo dati che rappresenta un oggetto SessionService
\end{description}

\item[Metodi] \hfill \\
\vspace{-7mm}
\begin{itemize}
	\item new(req : Request, res : Response, next : Function) : void $\rightarrow$ metodo che crea una sessione con una volta che l'utente effettua l'accesso all'applicazione\begin{itemize}
		\item req $\rightarrow$ questo oggetto rappresenta la richiesta arrivata al server che il metodo deve gestire
		\item res $\rightarrow$ questo oggetto rappresenta la risposta che il server dovrà inviare al termine dell'elaborazione
		\item next $\rightarrow$ questo parametro rappresenta la callback che il metodo dovrà chiamare al termine dell’elaborazione
	\end{itemize}
	
	\item delete(req : Request, res : Response, next : Function) : void $\rightarrow$ metodo che elimina la sessione dell'utente dal database\begin{itemize}
		\item req $\rightarrow$ questo oggetto rappresenta la richiesta arrivata al server che il metodo deve gestire
		\item res $\rightarrow$ questo oggetto rappresenta la risposta che il server dovrà inviare al termine dell'elaborazione
		\item next $\rightarrow$ questo parametro rappresenta la callback che il metodo dovrà chiamare al termine dell’elaborazione
	\end{itemize}
	
	\item SessionService() $\rightarrow$ costruttore della classe
\end{itemize}

\end{description}

\vspace{0.5cm}
\hypertarget{server::service::RoleService}{}
\subsubsection[RoleService]{server::service::RoleService}
\begin{figure}[H]
	\centering
	\begin{tikzpicture}
	\umlclass{server::service::RoleService} {}{+get(req : Request, res : Response, next : Function) : void\\+getByID(req : Request, res : Response, next : Function) : void\\+RoleService()}
	\end{tikzpicture}
	\caption{Diagramma classe - server::service::RoleService}
\end{figure}\begin{description}
\item[Descrizione] \hfill \\
Classe che rappresenta il servizio per la lettura dei ruoli utente, sfruttando la classe server::data::Role per accedere ai dati persistenti nel database
\item[Utilizzo] \hfill \\
Viene utilizzata per fornire un punto d'accesso per l'elenco di tutti i ruoli dell'applicazione e la lettura di un singolo ruolo
\item[Relazioni con altre classi] \hfill \\
\vspace{-7mm}
\begin{description}
	\item[\hyperlink{server::middleware::Router}{server::middleware::Router}] \hfill \\
	Relazione entrante, campo dati che rappresenta un oggetto RoleService
\end{description}

\item[Metodi] \hfill \\
\vspace{-7mm}
\begin{itemize}
	\item get(req : Request, res : Response, next : Function) : void $\rightarrow$ metodo che invia al client la lista di tutti i ruoli assumibili dagli utenti dell'applicazione attraverso un Json\begin{itemize}
		\item req $\rightarrow$ questo oggetto rappresenta la richiesta arrivata al server che il metodo deve gestire
		\item res $\rightarrow$ questo oggetto rappresenta la risposta che il server dovrà inviare al termine dell'elaborazione
		\item next $\rightarrow$ questo parametro rappresenta la callback che il metodo dovrà chiamare al termine dell’elaborazione
	\end{itemize}
	
	\item getByID(req : Request, res : Response, next : Function) : void $\rightarrow$ metodo che ritorna al client un Json contenente il ruolo specificato nella richiesta http\begin{itemize}
		\item req $\rightarrow$ questo oggetto rappresenta la richiesta arrivata al server che il metodo deve gestire
		\item res $\rightarrow$ questo oggetto rappresenta la risposta che il server dovrà inviare al termine dell'elaborazione
		\item next $\rightarrow$ questo parametro rappresenta la callback che il metodo dovrà chiamare al termine dell’elaborazione
	\end{itemize}
	
	\item RoleService() $\rightarrow$ costruttore della classe
\end{itemize}

\end{description}

\vspace{0.5cm}
\hypertarget{server::service::AnswerService}{}
\subsubsection[AnswerService]{server::service::AnswerService}
\begin{figure}[H]
	\centering
	\begin{tikzpicture}
	\umlclass{server::service::AnswerService} {}{+get(req : Request, res : Response, next : Function) : void\\+getByID(req : Request, res : Response, next : Function) : void\\+new(req : Request, res : Response, next : Function) : void\\+AnswerService()}
	\end{tikzpicture}
	\caption{Diagramma classe - server::service::AnswerService}
\end{figure}\begin{description}
\item[Descrizione] \hfill \\
Classe che si occupa della operazioni di inserimento e visualizzazione di risposte a domande dei questionari, sfruttando la classe server::data::Answer per accedere ai dati persistenti nel database.
\item[Utilizzo] \hfill \\
Fornisce i punteggi delle risposte date alle domande dei questionari a chi ne ha il permesso di accesso ed esegue operazioni di aggiunta e visualizzazione.
\item[Relazioni con altre classi] \hfill \\
\vspace{-7mm}
\begin{description}
	\item[\hyperlink{server::middleware::Router}{server::middleware::Router}] \hfill \\
	Relazione entrante, campo dati che rappresenta un oggetto Answer Service
\end{description}

\item[Metodi] \hfill \\
\vspace{-7mm}
\begin{itemize}
	\item get(req : Request, res : Response, next : Function) : void $\rightarrow$ metodo che invia al client la lista delle risposte in formato JSON in base alle impostazioni di filtraggio impostate\begin{itemize}
		\item req $\rightarrow$ questo oggetto rappresenta la richiesta arrivata al server che il metodo deve gestire
		\item res $\rightarrow$ questo oggetto rappresenta la risposta che il server dovrà inviare al termine dell'elaborazione
		\item next $\rightarrow$ questo parametro rappresenta la callback che il metodo dovrà chiamare al termine dell’elaborazione
	\end{itemize}
	
	\item getByID(req : Request, res : Response, next : Function) : void $\rightarrow$ metodo che ritorna al client un oggetto JSON contenente i dati della risposta identificata nella richiesta http\begin{itemize}
		\item req $\rightarrow$ questo oggetto rappresenta la richiesta arrivata al server che il metodo deve gestire
		\item res $\rightarrow$ questo oggetto rappresenta la risposta che il server dovrà inviare al termine dell'elaborazione
		\item next $\rightarrow$ questo parametro rappresenta la callback che il metodo dovrà chiamare al termine dell’elaborazione
	\end{itemize}
	
	\item new(req : Request, res : Response, next : Function) : void $\rightarrow$ metodo che aggiunge una nuova risposta nel database\begin{itemize}
		\item req $\rightarrow$ questo oggetto rappresenta la richiesta arrivata al server che il metodo deve gestire
		\item res $\rightarrow$ questo oggetto rappresenta la risposta che il server dovrà inviare al termine dell'elaborazione
		\item next $\rightarrow$ questo parametro rappresenta la callback che il metodo dovrà chiamare al termine dell’elaborazione
	\end{itemize}
	
	\item AnswerService() $\rightarrow$ costruttore della classe
\end{itemize}

\end{description}

\vspace{0.5cm}
\subsection{server::validator}
Package contenente tutte le classi "validator" che hanno il compito di controllare che alcuni campi di alcuni model siano validi; esempi includono la password dell'utente, la lunghezza del nome utente o il QML di una domanda\begin{center}
	\begin{figure}[H]
		\centering \includegraphics[scale=4, max width=\textwidth, max height=\myheight]{../img/diagrammiClassi/server/validator.png}
		\caption{Diagramma package - server::validator}
	\end{figure}
\end{center}\hypertarget{server::validator::UserCheck}{}
\subsubsection[UserCheck]{server::validator::UserCheck}
\begin{figure}[H]
	\centering
	\begin{tikzpicture}
	\umlclass{server::validator::UserCheck} {}{+checkFullName(fullName : String) : boolean\\+checkPassword(psw : String) : boolean\\+checkUserName(userName : String) : boolean\\+UserCheck()}
	\end{tikzpicture}
	\caption{Diagramma classe - server::validator::UserCheck}
\end{figure}\begin{description}
\item[Descrizione] \hfill \\
Classe contenente tutte le funzioni di controllo della validità dei campi del model User
\item[Utilizzo] \hfill \\
Viene utilizzata dai service per effettuare controlli sul model User
\item[Metodi] \hfill \\
\vspace{-7mm}
\begin{itemize}
	\item checkFullName(fullName : String) : boolean $\rightarrow$ metodo che controlla che il nome completo passato contenga almeno 2 caratteri\begin{itemize}
		\item fullName $\rightarrow$ il nome completo da controllare
	\end{itemize}
	
	\item checkPassword(psw : String) : boolean $\rightarrow$ metodo che controlla che la password passata contenga almeno 6 caratteri\begin{itemize}
		\item psw $\rightarrow$ la password, ovviamente in chiaro (non l'hash), da controllare
	\end{itemize}
	
	\item checkUserName(userName : String) : boolean $\rightarrow$ metodo che controlla che l'username passato contenga almeno 6 caratteri\begin{itemize}
		\item userName $\rightarrow$ l'username da controllare
	\end{itemize}
	
	\item UserCheck() $\rightarrow$ costruttore della classe
\end{itemize}

\end{description}

\vspace{0.5cm}
\hypertarget{server::validator::QuestionnaireCheck}{}
\subsubsection[QuestionnaireCheck]{server::validator::QuestionnaireCheck}
\begin{figure}[H]
	\centering
	\begin{tikzpicture}
	\umlclass{server::validator::QuestionnaireCheck} {}{+checkQuestions(questions : Question []) : boolean\\+checkTitle(title : String) : boolean\\+checkTags(tags : Tag []) : boolean\\+QuestionnaireCheck()}
	\end{tikzpicture}
	\caption{Diagramma classe - server::validator::QuestionnaireCheck}
\end{figure}\begin{description}
\item[Descrizione] \hfill \\
Classe contenente tutte le funzioni di controllo della validità dei campi del model Questionnaire
\item[Utilizzo] \hfill \\
Viene utilizzata dai service per effettuare controlli sul model User
\item[Metodi] \hfill \\
\vspace{-7mm}
\begin{itemize}
	\item checkQuestions(questions : Question []) : boolean $\rightarrow$ metodo che controlla che la lista/array di domande passate non sia vuota e che non contenga domande duplicate\begin{itemize}
		\item questions $\rightarrow$ l'elenco delle domande del questionario
	\end{itemize}
	
	\item checkTitle(title : String) : boolean $\rightarrow$ metodo che controlla che il titolo del questionario non sia vuoto\begin{itemize}
		\item title $\rightarrow$ il titolo del questionario da controllare
	\end{itemize}
	
	\item checkTags(tags : Tag []) : boolean $\rightarrow$ metodo che controlla che la lista/array di argomenti passati non sia vuota\begin{itemize}
		\item tags $\rightarrow$ elenco degli argomenti	
	\end{itemize}
	
	\item QuestionnaireCheck() $\rightarrow$ costruttore della classe
\end{itemize}

\end{description}

\vspace{0.5cm}
\hypertarget{server::validator::TagCheck}{}
\subsubsection[TagCheck]{server::validator::TagCheck}
\begin{figure}[H]
	\centering
	\begin{tikzpicture}
	\umlclass{server::validator::TagCheck} {}{+checkName(name : String) : boolean\\+TagCheck()}
	\end{tikzpicture}
	\caption{Diagramma classe - server::validator::TagCheck}
\end{figure}\begin{description}
\item[Descrizione] \hfill \\
Classe contenente tutte le funzioni di controllo della validità dei campi del model Tag
\item[Utilizzo] \hfill \\
Viene utilizzata dai service per effettuare controlli sul model Tag
\item[Metodi] \hfill \\
\vspace{-7mm}
\begin{itemize}
	\item checkName(name : String) : boolean $\rightarrow$ metodo che controlla che il nome dell'argomento non sia vuoto\begin{itemize}
		\item name $\rightarrow$ il nome dell'argomento da controllare
	\end{itemize}
	
	\item TagCheck() $\rightarrow$ costruttore della classe
\end{itemize}

\end{description}

\vspace{0.5cm}
\hypertarget{server::validator::QuestionCheck}{}
\subsubsection[QuestionCheck]{server::validator::QuestionCheck}
\begin{figure}[H]
	\centering
	\begin{tikzpicture}
	\umlclass{server::validator::QuestionCheck} {}{+checkQML(qml : String) : boolean\\+checkTags(tags : Tag []) : boolean\\+QuestionCheck()}
	\end{tikzpicture}
	\caption{Diagramma classe - server::validator::QuestionCheck}
\end{figure}\begin{description}
\item[Descrizione] \hfill \\
Classe contenente tutte le funzioni di controllo della validità dei campi del model Question
\item[Utilizzo] \hfill \\
Viene utilizzata dai service per effettuare controlli sul model Question
\item[Metodi] \hfill \\
\vspace{-7mm}
\begin{itemize}
	\item checkQML(qml : String) : boolean $\rightarrow$ metodo che controlla che la stringa di QML (generalmente quella dell'attributo body del model Question) sia QML valido\begin{itemize}
		\item qml $\rightarrow$ la stringa in formato QML da controllare
	\end{itemize}
	
	\item checkTags(tags : Tag []) : boolean $\rightarrow$ metodo che controlla che la lista/array di argomenti passati non sia vuota\begin{itemize}
		\item tags $\rightarrow$ elenco degli argomenti
	\end{itemize}
	
	\item QuestionCheck() $\rightarrow$ costruttore della classe
\end{itemize}

\end{description}

\vspace{0.5cm}
\hypertarget{server::validator::AnswerCheck}{}
\subsubsection[AnswerCheck]{server::validator::AnswerCheck}
\begin{figure}[H]
	\centering
	\begin{tikzpicture}
	\umlclass{server::validator::AnswerCheck} {}{+AnswareCheck()\\+checkScore(score : double) : boolean}
	\end{tikzpicture}
	\caption{Diagramma classe - server::validator::AnswerCheck}
\end{figure}\begin{description}
\item[Descrizione] \hfill \\
Classe contenente tutte le funzioni di controllo della validità dei campi del model Answare
\item[Utilizzo] \hfill \\
Viene utilizzata dai service per effettuare controlli sul model Answare
\item[Metodi] \hfill \\
\vspace{-7mm}
\begin{itemize}
	\item AnswareCheck() $\rightarrow$ costruttore della classe
	\item checkScore(score : double) : boolean $\rightarrow$ metodo che controlla se il punteggio passato sia compreso tra 0 e 1 compresi\begin{itemize}
		\item score $\rightarrow$ Punteggio da controllare della domanda
	\end{itemize}
	
\end{itemize}

\end{description}