%pdflatex -synctex=1 -interaction=nonstopmode %.tex|makeglossaries %|pdflatex -synctex=1 -interaction=nonstopmode %.tex|pdflatex -synctex=1 -interaction=nonstopmode %.tex
\makeglossary

\newglossaryentry{responsabile} {
	name=responsabile,
	description={è il responsabile della gestione, pianificazione e realizzazione del progetto},
	plural=Responsabili
}

\newglossaryentry{verificatore} {
	name=verificatore,
	description={è il responsabile dell'attività di verifica},
	plural=verificatori
}

\newglossaryentry{programmatore} {
	name=programmatore,
	description={è responsabile delle attività di codifica miranti alla realizzazione del prodotto e delle componenti di ausilio necessarie per l'esecuzione delle prove di verifica e validazione},
	plural=programmatori
}

\newglossaryentry{progettista} {
	name=progettista,
	description={è responsabile delle attività di progettazione},
	plural=Progettisti
}

\newglossaryentry{analista} {
	name=analista,
	description={è responsabile delle attività di analisi. },
	plural=Analisti
}

\newglossaryentry{amministratore} {
	name=amministratore,
	description={è responsabile dell'efficienza e dell'operatività dell'ambiente di sviluppo; si occupa della redazione e attuazione di piani e procedure di gestione della qualità; inoltre gestisce l'archivio della documentazione del progetto},
	plural=Amministratori
}

\newglossaryentry{revisione dei requisiti} {
	name=revisione dei requisiti,
	description={è una revisione formale che determina l'accesso del gruppo al progetto didattico e la concordanza con il cliente di una visione condivisa del prodotto atteso}
}

\newglossaryentry{revisione di accettazione} {
	name=revisione di accettazione,
	description={è una revisione formale per l'accertamento del soddisfacimento di tutti i requisiti e il completamento del progetto}
}

\newglossaryentry{revisione di progettazione} {
	name=revisione di progettazione,
	description={è una revisione di progresso che accerta la realizzabilità del prodotto e informa il cliente sulle caratteristiche del prodotto}
}

\newglossaryentry{revisione di qualifica} {
	name=revisione di qualifica,
	description={è una revisione di progresso che approva l'esito finale delle verifiche e attiva la fase di \mgls{validazione}}
}

\newglossaryentry{analisi} {
	name=analisi,
	description={è il periodo di preparazione e produzioni di documenti che precede la \RR}
}

\newglossaryentry{analisi di dettaglio} {
	name=analisi di dettaglio,
	description={è il periodo di correzione e revisione dei documenti che segue la \RR{}}
}

\newglossaryentry{progettazione} {
	name=progettazione architetturale,
	description={è il periodo che precede la \RP}
}

\newglossaryentry{progettazione di dettaglio} {
	name=progettazione di dettaglio,
	description={è il periodo che precede la \RP}
}

\newglossaryentry{codifica} {
	name=codifica,
	description={è il periodo che intercorre tra \RP\ e la \RQ}
}

\newglossaryentry{validazione} {
	name=validazione,
	description={è il periodo che intercorre tra la \RQ\ e la \RA}
}

\newglossaryentry{repository} {
	name=repository,
	description={è dove i file sono memorizzati, spesso su un server}
}

\newglossaryentry{ruolo} {
	name=ruolo,
	description={una delle figure professionali che una persona fisica interpreta nel corso del progetto. I ruoli sono: \mgls{responsabile}, \mgls{amministratore}, \mgls{analista}, \mgls{progettista}, \mgls{programmatore} e \mgls{verificatore}},
    plural=ruoli
}

\newglossaryentry{svg} {
	name=SVG,
	description={è un formato per la visualizzazione di oggetti in grafica vettoriale. Per maggiori informazioni si veda \href{https://it.wikipedia.org/wiki/Scalable_Vector_Graphics}{qui}}
}

\newglossaryentry{png} {
	name=PNG,
	description={abbreviazione di Portable Network Graphics, è un formato di file per memorizzare immagini. Per ulteriori informazioni si veda \href{http://it.wikipedia.org/wiki/Portable_Network_Graphics}{qui}}
}

\newglossaryentry{pdf} {
	name=PDF,
	description={è un formato di file basato su un linguaggio di descrizione di pagina sviluppato da Adobe Systems nel 1993 per rappresentare documenti in modo indipendente dall’hardware e dal software utilizzati per generarli o per visualizzarli. Per ulteriori informazioni si veda \href{http://it.wikipedia.org/wiki/Portable_Document_Format}{qui}}
}

\newglossaryentry{uml} {
	name=UML,
	description={è un linguaggio di modellazione e specifica basato sul paradigma object-oriented. Per ulteriori informazioni si veda \href{http://it.wikipedia.org/wiki/Unified_Modeling_Language}{qui}}
}

\newglossaryentry{walkthrough} {
	name=walkthrough,
    description={consiste nella lettura di un documento o codice cercando errori ed anomalie senza un'idea precisa di quali tipi di errori sarà possibile trovare}
}

\newglossaryentry{lista di controllo} {
	name=lista di controllo,
	description={è un elenco di cose da fare per eseguire una determinata attività}
}

\newglossaryentry{inspection} {
	name=inspection,
	description={è la lettura mirata di un documento o codice cercando errori specifici}
}

\newglossaryentry{milestone} {
	name=milestone,
	description={momento saliente nello sviluppo di un prodotto software per la quale devono essere pronti documenti e/o funzionalità}
}

\newglossaryentry{ticket} {
	name=ticket,
	description={rappresenta un compito nell'organizzazione e distribuzione del lavoro all'interno del progetto},
	plural=tickets
}

\newglossaryentry{commit} {
	name=commit,
	description={è la copia di modifiche fatte su file locali verso la \mgls{repository} remota. Esso rappresenta anche un particolare stato della \mgls{repository} nel tempo}
}

\newglossaryentry{versionamento} {
	name=versionamento,
	description={è la gestione delle versioni multiple di un insieme di informazioni. Per maggiori informazioni si veda \href{http://it. wikipedia.org/wiki/Controllo_versione}{qui}}
}

\newglossaryentry{task} {
	name=task,
	description={è un compito secondo la definizione dello standard ISO/IEC 12207},
	plural=tasks
}

\newglossaryentry{attivita} {
	name=attivita,
	description={è un insieme di \mgls{task}}
}

\newglossaryentry{redattore} {
	name=redattore,
	description={colui che redige un documento},
	plural=redattori
}

\newglossaryentry{proponente} {
	name=proponente,
	description={colui che ha proposto al \mgls{committente} un capitolato d'appalto}
}

\newglossaryentry{committente} {
	name=committente,
	description={colui che assegna un compito. In questo caso è il Professor Tullio Vardanega}
}

\newglossaryentry{quality assurance} {
	name=quality assurance,
	description={è l'insieme delle \mgls{attivita} volte a garantire il soddisfacimento degli obiettivi della qualità}
}

\newglossaryentry{telegram} {
	name=Telegram,
	description={è un servizio di messaggistica istantanea utilizzato dal gruppo per comunicazioni interne. Per maggiori informazioni si veda \href{https://it.wikipedia.org/wiki/Telegram_(software)}{qui}}
}

\newglossaryentry{browser} {
	name=browser,
	description={è un'applicazione per il recupero, la presentazione e la navigazione di risorse web}
}

\newglossaryentry{google drive} {
	name=Google Drive,
	description={è un servizio di memorizzazione e sincronizzazione online introdotto da Google il 24 aprile 2012. Per maggiori informazioni si veda \href{https://it.wikipedia.org/wiki/Google_Drive}{qui}}
}

\newglossaryentry{skype} {
	name=Skype,
	description={è un software proprietario freeware di messaggistica istantanea e VoIP. Per maggiori informazioni si veda \href{https://it.wikipedia.org/wiki/Skype}{qui}}
}

\newglossaryentry{gantt} {
	name=Gantt,
	description={è un diagramma di supporto alla gestione dei progetti}
}

\newglossaryentry{projectlibre} {
	name=ProjectLibre,
	description={è un software di gestione progettuale}
}

\newglossaryentry{pert} {
	name=PERT,
	description={è uno strumento volto alla programmazione delle attività che compongono il progetto e, più in generale, alla gestione degli aspetti temporali di quest'ultimo}
}

\newglossaryentry{subtask} {
	name=subtask,
	description={è un \mgls{task} compreso all'interno di un altro \mgls{task}. La totalità di tutti i subtasks costituisce un intero \mgls{task}}
	plural=subtasks
}

\newglossaryentry{ticketing} {
	name=ticketing,
	description={procedura con la quale il \RE assegna un \mgls{task}}
}

\newglossaryentry{git} {
	name=git,
	description={è un sistema software di controllo di versione distribuito}
}

\newglossaryentry{quizzipedia} {
	name=Quizzipedia,
	description={è il nome del prodotto software richiesto dal capitolato d'appalto scelto}
}

\newglossaryentry{schierabile} {
	name=schierabile,
	description={è la capacità di rilasciare al cliente, con relativa installazione e messa in funzione o esercizio, di una applicazione o di un sistema software tipicamente all'interno di un sistema informatico aziendale}
}

\newglossaryentry{cross-platform} {
	name=cross-platform,
	description={può essere riferito ad un linguaggio di programmazione, ad un'applicazione software o ad un dispositivo hardware che funziona su più di un sistema}
}

\newglossaryentry{qml} {
	name=QML,
	description={è un "Domain Specific Language" richiesto dal capitolato d'appalto per la definizione delle domande all'interno del sistema}
}

\newglossaryentry{tomcat} {
	name=Tomcat,
	description={è un application server nella forma di contenitore servlet \mgls{open-source} sviluppato dalla Apache Software Foundation. Per maggiori informazioni si veda \href{https://it.wikipedia.org/wiki/Apache_Tomcat}{qui}}
}

\newglossaryentry{java} {
	name=Java,
	description={è un linguaggio di programmazione orientato agli oggetti, specificatamente progettato per essere il più possibile indipendente dalla piattaforma di esecuzione. Per maggiori informazioni si veda \href{https://it.wikipedia.org/wiki/Java_(linguaggio_di_programmazione)}{qui}}
}

\newglossaryentry{node.js} {
	name=Node.js,
	description={è una piattaforma lato server avvolto attorno al linguaggio \mgls{javascript} per la creazione di applicazioni scalabili, utilizzando un modello event-driven. Ovvero si eseguono azioni solo al verificarsi di un evento }
}

\newglossaryentry{javascript} {
	name=JavaScript,
	description={linguaggio di scripting orientato agli oggetti comunemente usato nella programmazione Web}
}

\newglossaryentry{postgresql} {
	name=PostgreSQL,
	description={e un sistema di gestione di basi di dati \mgls{open-source} usato per applicazioni che richiedono caratteristiche molto complesse }
}

\newglossaryentry{mongodb} {
	name=MongoDB,
	description={è un sistema gestionale di basi di dati \mgls{nosql} orientato ai documenti, adatto per ambienti che hanno la necessità d'immagazzinare grosse quantità di dati, e dove il linguaggio utilizzato per la gestione dei dati è \mgls{javascript}}
}

\newglossaryentry{html5} {
	name=HTML5,
	description={linguaggio di markup per la strutturazione delle pagine web}
}

\newglossaryentry{css3} {
	name=CSS3,
	description={è un linguaggio di programmazione web utilizzato per descrivere l'aspetto e la formattazione di un sito web al browser lato client}
}

\newglossaryentry{xml} {
	name=XML,
	description={è un meta-linguaggio che fornisce un insieme standard di regole sintattiche per modellare la struttura di documenti e dati. Questo insieme di regole, definiscono le modalità secondo cui è possibile crearsi un proprio linguaggio di markup}
}

\newglossaryentry{nosql} {
	name=NoSQL,
	description={come dice anche il termine questi database non sono basati su SQL, ovvero su uno schema relazionale. I database relazionali sono infatti ottimi quando esistono delle relazioni tra i dati che salviamo, ma sono poco performanti nel caso sia necessario salvare una grande quantità di dati, magari usando la scalabilità orizzontale, quando cioè si utilizzano più server dove salvare questi dati e non solamente incrementando la potenza di un singolo server}
}

\newglossaryentry{scala} {
	name=Scala,
	description={è un linguaggio di programmazione di tipo general-purpose multi-paradigma studiato per integrare le caratteristiche e funzionalità dei linguaggi orientati agli oggetti e dei linguaggi funzionali. Per maggiori informazioni si veda \href{https://it.wikipedia.org/wiki/Scala_(linguaggio_di_programmazione)}{qui}}
}

\newglossaryentry{akka} {
	name=Akka,
	description={è un toolkit di strumenti per la costruzione di applicazioni con elevata concorrenza di dati che necessitano di un sistema resiliente per l'invio e la ricezione di messaggi}
}

\newglossaryentry{ble} {
	name=BLE,
	description={Bluetooth Low Energy,  pur mantenendo un range di comunicazione simile a quello classico, fornisce un  consumo energetico dei device notevolmente ridotto}
}

\newglossaryentry{mqtt} {
	name=MQTT,
	description={è un protocollo di messaggistica leggero posizionato in cima a TCP/IP, disegnato per le situazioni in cui è richiesto un basso impatto e dove la banda è limitata }
}

\newglossaryentry{aws} {
	name=AWS,
	description={è un insieme di servizi di elaborazione remoti, detti anche servizi web, che costituiscono una piattaforma di\mgls{cloud} computing offerto da Amazon. Per maggiori informazioni si veda \href{https://aws.amazon.com/it/}{qui}}
}

\newglossaryentry{heroku} {
	name=Heroku,
	description={è una  delle prime \mgls{cloud} Platform-as-a-Service (PaaS) che supportava solo Java come linguaggio di programmazione, oggi giorno ha aggiunto molti altri linguaggi come \mGls{scala}, Phyton, etc. Per maggiori informazioni si veda \href{https://www.heroku.com}{qui}}
}

\newglossaryentry{github} {
	name=Github,
	description={è un servizio che offre gratuitamente l'hosting di un sever per \mgls{git}. Per maggiori informazioni si veda \href{https://github.com}{qui}}
}

\newglossaryentry{template} {
	name=template,
	description={traducibile in italiano come modello, indica o un programma o un documento idealizzato come un documento semi compilato cartaceo che ha degli spazi bianchi che saranno successivamente riempiti}
	plural=templates
}

\newglossaryentry{dropbox} {
	name=Dropbox,
	description={è un software di \mgls{cloud} storage multi piattaforma, che offre un servizio di file hosting e sincronizzazione automatica di file tramite web}
}

\newglossaryentry{open-source} {
	name=open-source,
	description={un software di cui gli autori ovvero i detentori dei diritti rendono pubblico il codice sorgente, favorendone il libero studio e permettendo a programmatori indipendenti di apportarvi modifiche ed estensioni. Questo è realizzato tramite apposite licenze d'uso}
}

\newglossaryentry{project management} {
	name=project management,
	description={si intende l'insieme di \mgls{attivita} aziendali, svolte tipicamente da una figura dedicata e specializzata detta responsabile, volte all'analisi, progettazione, pianificazione e realizzazione degli obiettivi di un progetto, gestendolo in tutte le sue caratteristiche e fasi evolutive, nel rispetto di precisi vincoli come i tempi, costi, risorse, scopi, qualità}
}

\newglossaryentry{linux} {
	name=Linux,
	description={è una famiglia di sistemi operativi open-source di tipo Unix-like, rilasciati sotto varie possibili distribuzioni, aventi la caratteristica comune di utilizzare come nucleo il kernel Linux}
}

\newglossaryentry{windows} {
	name=Windows,
	description={è una famiglia di ambienti operativi e sistemi operativi dedicati ai personal computer, alle workstation, ai server e agli \mgls{smartphone}. Il sistema operativo si chiama così per via della sua interfaccia di programmazione di un'applicazione a finestre}
}

\newglossaryentry{mac os} {
	name=Mac OS,
	description={il sistema operativo di Apple dedicato dedicati ai personal computer Macintosh, alle workstation, ai server e agli \mgls{smartphone}  }
}


\newglossaryentry{schedule variance} {
	name=schedule variance,
	description={ogni deviazione alle baseline di un progetto misurata confrontando costo preventivato di programma di lavoro con il costo preventivato del lavoro svolto. Indica quindi al cliente se il progetto sta procedendo nei tempi stabiliti}
}

\newglossaryentry{cost variance} {
	name=cost variance,
	description={indica al management aziendale se il valore del costo realmente maturato è maggiore, uguale o minore rispetto al costo pianificato}
}


\newglossaryentry{merge} {
	name=merge,
	description={comando \mGls{git} per unire due \mglspl{branch}. Per maggiori informazioni si veda \href{https://git-scm.com/docs/}{qui}}
}

\newglossaryentry{slack} {
	name=slack,
	description={intervallo di tempo entro cui un evento deve avvenire nel rispetto dei vincoli logici e imposti dal reticolo di pianificazione senza compromettere la durata complessiva del progetto}
}

\newglossaryentry{baseline} {
	name=baseline,
	description={di progetto costituisce il punto di riferimento rispetto al quale calcolare gli scostamenti delle principali variabili implicate nella gestione di un progetto}
}

\newglossaryentry{asana} {
	name=Asana,
	description={è un software che permette a dei team di monitorare il loro lavoro e tener traccia dei risultati tramite l'assegnazione di \mgls{task} a componenti specifiche del gruppo di lavoro}
}

\newglossaryentry{deadline} {
	name=deadline,
	description={la deadline di un \mgls{task} è un indicazione dell'urgenza del \mgls{task}; rappresenta un un punto su una linea temporale ideale. Data di scadenza o termine entro il quale deve essere completato un compito assegnato}
}

\newglossaryentry{revert} {
	name=revert,
	description={per annullare le ultime modifiche effettuate al \mgls{repository}. Per maggiori informazioni si veda \href{https://git-scm.com/docs/}{qui}}
}

\newglossaryentry{backup} {
	name=backup,
	description={si indica la replicazione, su un qualunque supporto di memorizzazione, di materiale informativo archiviato nella memoria di massa, al fine di prevenire la perdita definitiva dei dati in caso di eventi malevoli accidentali o intenzionali}
}

\newglossaryentry{push} {
	name=push,
	description={comando \mGls{git} per inviare modifiche di un documento site in un host locale al \mgls{repository} remoto. Per maggiori informazioni si veda \href{https://git-scm.com/docs/}{qui}}
}

\newglossaryentry{teamwork} {
	name=teamwork,
	description={è un software che permette a dei team di monitorare il loro lavoro e tener traccia dei risultati tramite l'assegnazione di \mgls{task} a componenti specifiche del gruppo di lavoro}
}

\newglossaryentry{evento} {
	name=evento,
	description={\mgls{teamwork}, attraverso il suo calendario, offre la possibilità di impostare eventi in giorni stabiliti dagli utenti. Questi eventi verranno periodicamente segnalati come promemoria a tutte le persone invitate agli stessi}
}

\newglossaryentry{etichetta} {
	name=etichetta,
	description={è un controllo grafico che mostra informazioni testuali all'interno di un form},
	plural=etichette
}

\newglossaryentry{bug} {
	name=bug,
	description={identifica un errore nella scrittura di un programma software che ne causa un comportamento imprevisto o comunque diverso da quello specificato dal produttore}
	plural=bugs
}

\newglossaryentry{software} {
	name=software,
	description={è un termine generico che definisce programmi e procedure utilizzati per far eseguire al computer un determinato compito}
}

\newglossaryentry{desktop} {
	name=desktop,
	description={area dello schermo su cui appaiono le icone e le finestre rappresentanti le memorie di massa collegate al computer ed il loro contenuto}
}

\newglossaryentry{draw.io} {
	name=draw.io,
	description={é un software gratuito per la creazione di diagrammi di flusso, di processo, UML, e diagrammi di rete. Per maggiori informazioni si veda \href{https://www.draw.io}{qui}}
}

\newglossaryentry{tracy} {
	name=Tracy,
	description={software \mgls{open-source} per il tracciamento.Per maggiori informazioni si veda \href{http://tracy-tpiga.rhcloud.com/tracy/}{qui}}
}

\newglossaryentry{package} {
	name=package,
	description={è un meccanismo per organizzare classi \mGls{java} in gruppi logici, principalmente allo scopo di definire name space distinti per diversi contesti}
}

\newglossaryentry{stakeholder} {
	name=stakeholder,
	description={si indica genericamente un soggetto o un gruppo di soggetti influenti nei confronti di un'iniziativa economica, sia essa un'azienda o un progetto}
}

\newglossaryentry{branch} {
	name=branch,
	description={quando si vuole creare un nuovo ramo al \mgls{repository} remoto. Per maggiori informazioni si veda \href{https://git-scm.com/docs/}{qui}}
	plural=branches
}

\newglossaryentry{pull} {
	name=pull,
	description={un comando \mgls{git} per poter ricevere nel locale tutte le modifiche fatte nel \mgls{repository} remoto. Per maggiori informazioni si veda \href{https://git-scm.com/docs/}{qui}}
}

\newglossaryentry{underscore} {
	name=underscore,
	description={è un carattere che identifica il trattino basso}
}

\newglossaryentry{spelling} {
	name=spelling,
	description={è l'atto di pronunciare le parole lentamente, separando le singole lettere o le sillabe}
}

\newglossaryentry{cloud} {
	name=cloud,
	description={si indica un sistema di erogazione di risorse informatiche, come l'archiviazione, l'elaborazione o la trasmissione di dati, caratterizzato dalla disponibilità on demand attraverso Internet}
}

\newglossaryentry{smartphone} {
	name=smartphone,
	description={è un telefono cellulare con capacità di calcolo, di memoria e di connessione dati molto più avanzate rispetto ai normali telefoni cellulari}
	plural=smartphones
}

\newglossaryentry{checkbox} {
	name=checkbox,
	description={è un controllo grafico con cui l'utente può effettuare selezioni multiple}
}


\newglossaryentry{makefile} {
	name=makefile,
	description={è usata soprattutto per la compilazione di codice sorgente in codice oggetto, unendo e poi linkando il codice oggetto in programmi eseguibili o in librerie}
}

\newglossaryentry{gulpease} {
	name=Gulpease,
	description={è un indice di leggibilità di un testo tarato sulla lingua italiana. Rispetto ad altri ha il vantaggio di utilizzare la lunghezza delle parole in lettere anziché in sillabe, semplificandone il calcolo automatico}
}

\newglossaryentry{pdca} {
	name=PDCA,
	description={acronimo di "Plan-Do-Check-Act" anche chiamato Ciclo di Daming, è un modello per il miglioramento continuo della qualità}
}

\newglossaryentry{framework} {
	name=framework,
	description={architettura o struttura di supporto su cui un programma può essere creato. In genere è composto da una serie di librerie e strumenti di sviluppo}
}

\newglossaryentry{mobile} {
	name=mobile,
	description={abbreviazione di "mobile phone", nel contesto si intende il mercato dei cellulari e smartphone e le tecnologie specifiche per lo sviluppo su di esse}
}

\newglossaryentry{database} {
	name=database,
	description={archivio di dati collegati tra loro attraverso un modello logico, in genere relazionale; permette la gestione e il reperimento di dati in modo veloce ed efficiente}
}

\newglossaryentry{responsive} {
	name=responsive,
	description={si riferisce ad un design che è ottimizzato per essere visto su un grande insieme di dispositivi sia mobili che desktop}
}

\newglossaryentry{client} {
	name=client,
	description={nell'architettura client-server il client è il componente che richiede un qualsiasi servizio ad un server; si può riferire all'applicazione o all'hardware}
}

\newglossaryentry{server} {
	name=server,
	description={nell'architettura client-server il server è un componente che fornisce un servizio ad un insieme di client; si può riferire all'applicazione o all'hardware}
}

\newglossaryentry{bottom-up} {
	name=bottom-up,
	description={È una strategia di elaborazione dell'informazione e di gestione delle conoscenze, riguardanti principalmente il software e, per estensione, altre teorie umanistiche e teorie dei sistemi. In linea generale, esse sono metodologie adoperate per analizzare situazioni problematiche e costruire ipotesi adeguate alla loro soluzione. La riusabilità del codice è uno dei principali benefici di tale approcio. Per maggiori informazioni si veda \href{https://it.wikipedia.org/wiki/Progettazione_top-down_e_bottom-up}{qui}}
}

\newglossaryentry{stub} {
	name=stub,
	description={nel contesto dei test di unità, sono programmi che simulano l'esecuzione di componenti software per la temporanea sostituzione di essi all'interno di un test software che non riguarda tali componenti}
}

\newglossaryentry{driver} {
    name=driver,
    description={nel contesto dei test di unità è un programma che simula l'utilizzo di altri componenti software in attesa che il vero programma venga sviluppato; in genere viene utilizzato come temporaneo sostituto di un programma che chiama determinate procedure per poter testare le procedure stesse}
}

\newglossaryentry{logger} {
    name=logger,
    description={componente che registra dati sull'esecuzione di un programma o una specifica procedura in modo che il programmatore abbia tali informazioni in caso il programma si comporti in modo inaspettato}
}

\newglossaryentry{tag} {
    name=tag,
    plural=tags,
    description={in git è un segnalibro associato ad un particolare commit utilizzato per marcare uno stato della repository per poi facilitarne il ritrovo}
}

\newglossaryentry{release} {
    name=release,
    plural=releases,
    description={ciascuna nuova versione di un software, messa in commercio o comunque diffusa, contraddistinta da un numero}
}

\newglossaryentry{latex} {
	name=Latex,
	description={è un linguaggio di markup usato per la preparazione di testi basato sul programma di composizione tipografica TEX}
}

\newglossaryentry{Node.js} {
	name=Node.js,
	description={\TODO{}}
}

\newglossaryentry{piattaforma} {
	name=piattaforma,
	description={\TODO{}}
}

\newglossaryentry{C} {
	name=C,
	description={\TODO{}}
}

\newglossaryentry{C++} {
	name=C++,
	description={\TODO{}}
}

\newglossaryentry{Javascript} {
	name=JavaScript,
	description={\TODO{}}
}

\newglossaryentry{V8 JavaScript Engine} {
	name=V8 JavaScript Engine,
	description={\TODO{}}
}

\newglossaryentry{Google Chrome} {
	name=Google Chrome,
	description={\TODO{}}
}

\newglossaryentry{event-driven} {
	name=event-driven,
	description={\TODO{}}
}

\newglossaryentry{event-driver} {
	name=event-driver,
	description={\TODO{}}
}

\newglossaryentry{I/O} {
	name=I/O,
	description={\TODO{}}
}

\newglossaryentry{thread} {
	name=thread,
	description={\TODO{}}
}

\newglossaryentry{multi-thread} {
	name=multi-thread,
	description={\TODO{}}
}

\newglossaryentry{moduli} {
	name=moduli,
	description={\TODO{}}
}

\newglossaryentry{node package manager} {
	name=node package manager,
	description={\TODO{}}
}

\newglossaryentry{MongoDB} {
	name=MongoDB,
	description={\TODO{}}
}

\newglossaryentry{document-oriented} {
	name=document-oriented,
	description={\TODO{}}
}

\newglossaryentry{JSON} {
	name=JSON,
	description={\TODO{}}
}

\newglossaryentry{BSON} {
	name=BSON,
	description={\TODO{}}
}

\newglossaryentry{join} {
	name=join,
	description={\TODO{}}
}

\newglossaryentry{schema} {
	name=schema,
	description={\TODO{}}
}

\newglossaryentry{Map-Reduce} {
	name=Map-Reduce,
	description={\TODO{}}
}

\newglossaryentry{query} {
	name=query,
	description={\TODO{}}
}

\newglossaryentry{Document} {
	name=Document,
	description={\TODO{}}
}

\newglossaryentry{Mongoose} {
	name=Mongoose,
	description={\TODO{}}
}

\newglossaryentry{API} {
	name=API,
	description={\TODO{}}
}

\newglossaryentry{snippet} {
	name=snippet,
	description={\TODO{}}
}

\newglossaryentry{ODM} {
	name=ODM,
	description={\TODO{}}
}

\newglossaryentry{schema-free} {
	name=schema-free,
	description={\TODO{}}
}

\newglossaryentry{sql} {
	name=SQL,
	description={\TODO{}}
}
