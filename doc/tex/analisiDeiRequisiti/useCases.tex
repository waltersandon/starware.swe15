\hypertarget{UC1}{}
\subsection{Caso d'uso UC1: Autenticazione}
	\begin{figure}[H]
		\centering
		\begin{resizedtikzpicture}{\textwidth}
		\umlactor[x=0, y=-1]{Ospite}
		\begin{umlsystem}[x=0, fill=lightgray!20]{Quizzipedia}
			\umlusecase[x=5, y=-3, fill=white, width=4cm, name=192]{\textbf{UC1.2:} Inserimento password}
			\umlassoc{Ospite}{192}
			\umlusecase[x=5, y=1, fill=white, width=4cm, name=191]{\textbf{UC1.1:} Inserimento username}
			\umlassoc{Ospite}{191}
		\end{umlsystem}
		\end{resizedtikzpicture}
		\caption{\textbf{UC1}: Autenticazione}
		\label{UC1}
	\end{figure}
\begin{description}
\item[Attori:] Ospite;
\item[Scopo e descrizione:] L'ospite inserisce le credenziali per accedere al sistema
      \item[Precondizione:] L'ospite non è autenticato e possiede un account all'interno del sistema;

        \item[Flusso principale degli eventi:] \ 
 \begin{enumerate}
          \item L'ospite inserisce l'username (\hyperlink{UC1.1}{UC1.1});
          \item L'ospite inserisce la password (\hyperlink{UC1.2}{UC1.2});

      \end{enumerate}
    \item[Postcondizione:] L’ospite è autenticato nel sistema.
  \end{description}
\hypertarget{UC1.1}{}
\subsection{Caso d'uso UC1.1: Inserimento username}
	\begin{figure}[H]
		\centering
		\begin{resizedtikzpicture}{\textwidth}
		\umlactor[x=0, y=-1]{Ospite}
		\begin{umlsystem}[x=0, fill=lightgray!20]{Quizzipedia}
			\umlusecase[x=5, y=-1, fill=white, width=4cm, name=191]{\textbf{UC1.1:} Inserimento username}
			\umlassoc{Ospite}{191}
			\umlusecase[x=15, y=-1, fill=white, width=4cm, name=193]{\textbf{UC1.3:} Errore username non presente}
			\umlextend{193}{191}
		\end{umlsystem}
		\end{resizedtikzpicture}
		\caption{\textbf{UC1.1}: Inserimento username}
		\label{UC1.1}
	\end{figure}
\begin{description}
\item[Attori:] Ospite;
\item[Scopo e descrizione:] L'ospite inserisce l'username per l'accesso
      \item[Precondizione:] L'ospite non è autenticato e possiede un account all'interno del sistema;

        \item[Flusso principale degli eventi:] \ 
 \begin{enumerate}
          \item L'ospite inserisce l'username per l'accesso;

      \end{enumerate}
    \item[Estensioni:]
      \begin{enumerate}
          \item Se l'username inserito non è presente nel sistema viene visualizzato un messaggio di errore (\hyperlink{UC1.3}{UC1.3});

      \end{enumerate}
    \item[Postcondizione:] L'ospite ha inserito l'username.
  \end{description}
\hypertarget{UC1.2}{}
\subsection{Caso d'uso UC1.2: Inserimento password}
	\begin{figure}[H]
		\centering
		\begin{resizedtikzpicture}{\textwidth}
		\umlactor[x=0, y=-1]{Ospite}
		\begin{umlsystem}[x=0, fill=lightgray!20]{Quizzipedia}
			\umlusecase[x=5, y=-1, fill=white, width=4cm, name=192]{\textbf{UC1.2:} Inserimento password}
			\umlassoc{Ospite}{192}
			\umlusecase[x=15, y=-1, fill=white, width=4cm, name=194]{\textbf{UC1.4:} Errore password errata}
			\umlextend{194}{192}
		\end{umlsystem}
		\end{resizedtikzpicture}
		\caption{\textbf{UC1.2}: Inserimento password}
		\label{UC1.2}
	\end{figure}
\begin{description}
\item[Attori:] Ospite;
\item[Scopo e descrizione:] L'ospite inserisce la sua password per l'accesso
      \item[Precondizione:] L'ospite non è autenticato e possiede un account all'interno del sistema
;

        \item[Flusso principale degli eventi:] \ 
 \begin{enumerate}
          \item L'ospite inserisce la sua password per l'accesso
;

      \end{enumerate}
    \item[Estensioni:]
      \begin{enumerate}
          \item Se la password inserita non è corretta viene visualizzato un messaggio di errore (\hyperlink{UC1.4}{UC1.4});

      \end{enumerate}
    \item[Postcondizione:] L'ospite ha inserito la sua password per l'accesso.
  \end{description}
\hypertarget{UC1.3}{}
\subsection{Caso d'uso UC1.3: Errore username non presente}
	\begin{figure}[H]
		\centering
		\begin{resizedtikzpicture}{\textwidth}
		\umlactor[x=0, y=-1]{Ospite}
		\begin{umlsystem}[x=0, fill=lightgray!20]{Quizzipedia}
			\umlusecase[x=5, y=-1, fill=white, width=4cm, name=193]{\textbf{UC1.3:} Errore username non presente}
			\umlassoc{Ospite}{193}
		\end{umlsystem}
		\end{resizedtikzpicture}
		\caption{\textbf{UC1.3}: Errore username non presente}
		\label{UC1.3}
	\end{figure}
\begin{description}
\item[Attori:] Ospite;
\item[Scopo e descrizione:] L'username inserito non esiste nel sistema
      \item[Precondizione:] L'ospite ha inserito un username che non esiste nel sistema;

        \item[Flusso principale degli eventi:] \ 
 \begin{enumerate}
          \item L'username inserito non esiste nel sistema
;

      \end{enumerate}
    \item[Postcondizione:] Viene visualizzato un messaggio di errore.
  \end{description}
\hypertarget{UC1.4}{}
\subsection{Caso d'uso UC1.4: Errore password errata}
	\begin{figure}[H]
		\centering
		\begin{resizedtikzpicture}{\textwidth}
		\umlactor[x=0, y=-1]{Ospite}
		\begin{umlsystem}[x=0, fill=lightgray!20]{Quizzipedia}
			\umlusecase[x=5, y=-1, fill=white, width=4cm, name=194]{\textbf{UC1.4:} Errore password errata}
			\umlassoc{Ospite}{194}
		\end{umlsystem}
		\end{resizedtikzpicture}
		\caption{\textbf{UC1.4}: Errore password errata}
		\label{UC1.4}
	\end{figure}
\begin{description}
\item[Attori:] Ospite;
\item[Scopo e descrizione:] La password per l'accesso inserita dall'ospite è incorretta
      \item[Precondizione:] L'ospite ha inserito la sua password per l'accesso;

        \item[Flusso principale degli eventi:] \ 
 \begin{enumerate}
          \item La password per l'accesso inserita dall'ospite non corrisponde a quella salvata nel sistema
;

      \end{enumerate}
    \item[Postcondizione:] Viene visualizzato un messaggio d'errore.
  \end{description}
\hypertarget{UC2}{}
\subsection{Caso d'uso UC2: Registrazione}
	\begin{figure}[H]
		\centering
		\begin{resizedtikzpicture}{\textwidth}
		\umlactor[x=0, y=-1]{Ospite}
		\begin{umlsystem}[x=0, fill=lightgray!20]{Quizzipedia}
			\umlusecase[x=5, y=-5, fill=white, width=4cm, name=186]{\textbf{UC2.3:} Inserimento password}
			\umlassoc{Ospite}{186}
			\umlusecase[x=5, y=-1, fill=white, width=4cm, name=185]{\textbf{UC2.2:} Inserimento username}
			\umlassoc{Ospite}{185}
			\umlusecase[x=5, y=3, fill=white, width=4cm, name=184]{\textbf{UC2.1:} Inserimento nome completo}
			\umlassoc{Ospite}{184}
		\end{umlsystem}
		\end{resizedtikzpicture}
		\caption{\textbf{UC2}: Registrazione}
		\label{UC2}
	\end{figure}
\begin{description}
\item[Attori:] Ospite;
\item[Scopo e descrizione:] L'utente inserisce i dati del profilo e le credenziali per l'accesso
      \item[Precondizione:] L'ospite non è autenticato nel sistema e non ha un account presso il sistema;

        \item[Flusso principale degli eventi:] \ 
 \begin{enumerate}
          \item L'utente inserisce il nome completo (\hyperlink{UC2.1}{UC2.1});
          \item L'utente inserisce l'username (\hyperlink{UC2.2}{UC2.2});
          \item L'utente inserisce la password (\hyperlink{UC2.3}{UC2.3});

      \end{enumerate}
    \item[Postcondizione:] L’ospite possiede un account di ruolo studente presso il sistema.
  \end{description}
\hypertarget{UC2.1}{}
\subsection{Caso d'uso UC2.1: Inserimento nome completo}
	\begin{figure}[H]
		\centering
		\begin{resizedtikzpicture}{\textwidth}
		\umlactor[x=0, y=-1]{Ospite}
		\begin{umlsystem}[x=0, fill=lightgray!20]{Quizzipedia}
			\umlusecase[x=5, y=-1, fill=white, width=4cm, name=184]{\textbf{UC2.1:} Inserimento nome completo}
			\umlassoc{Ospite}{184}
			\umlusecase[x=15, y=-1, fill=white, width=4cm, name=187]{\textbf{UC2.4:} Errore nome completo troppo corto}
			\umlextend{187}{184}
		\end{umlsystem}
		\end{resizedtikzpicture}
		\caption{\textbf{UC2.1}: Inserimento nome completo}
		\label{UC2.1}
	\end{figure}
\begin{description}
\item[Attori:] Ospite;
\item[Scopo e descrizione:] L'utente inserisce il nome completo
      \item[Precondizione:] L'ospite non è autenticato nel sistema e non ha un account presso il sistema;

        \item[Flusso principale degli eventi:] \ 
 \begin{enumerate}
          \item L'utente inserisce il nome completo di almeno 2 caratteri;

      \end{enumerate}
    \item[Estensioni:]
      \begin{enumerate}
          \item Se il nome completo è troppo corto viene visualizzato un errore (\hyperlink{UC2.4}{UC2.4});

      \end{enumerate}
    \item[Postcondizione:] L'utente ha specificato il nome completo.
  \end{description}
\hypertarget{UC2.2}{}
\subsection{Caso d'uso UC2.2: Inserimento username}
	\begin{figure}[H]
		\centering
		\begin{resizedtikzpicture}{\textwidth}
		\umlactor[x=0, y=-1]{Ospite}
		\begin{umlsystem}[x=0, fill=lightgray!20]{Quizzipedia}
			\umlusecase[x=5, y=-1, fill=white, width=4cm, name=185]{\textbf{UC2.2:} Inserimento username}
			\umlassoc{Ospite}{185}
			\umlusecase[x=15, y=-3, fill=white, width=4cm, name=190]{\textbf{UC2.7:} Errore username già utilizzato}
			\umlextend{190}{185}
			\umlusecase[x=15, y=1, fill=white, width=4cm, name=188]{\textbf{UC2.5:} Errore username troppo corto}
			\umlextend{188}{185}
		\end{umlsystem}
		\end{resizedtikzpicture}
		\caption{\textbf{UC2.2}: Inserimento username}
		\label{UC2.2}
	\end{figure}
\begin{description}
\item[Attori:] Ospite;
\item[Scopo e descrizione:] L'utente inserisce l'username con la quale effettuare l'autenticazione
      \item[Precondizione:] L'ospite non è autenticato nel sistema e non ha un account presso il sistema;

        \item[Flusso principale degli eventi:] \ 
 \begin{enumerate}
          \item L'utente inserisce l'username, di almeno 6 caratteri, con la quale effettuare l'autenticazione  (\hyperlink{UC2.4}{UC2.4});

      \end{enumerate}
    \item[Estensioni:]
      \begin{enumerate}
          \item Se l'username è troppo corto viene visualizzato un errore (\hyperlink{UC2.5}{UC2.5});
          \item Se l'username è già stato utilizzato viene visualizzato un errore (\hyperlink{UC2.7}{UC2.7});

      \end{enumerate}
    \item[Postcondizione:] L'utente ha specificato l'username.
  \end{description}
\hypertarget{UC2.3}{}
\subsection{Caso d'uso UC2.3: Inserimento password}
	\begin{figure}[H]
		\centering
		\begin{resizedtikzpicture}{\textwidth}
		\umlactor[x=0, y=-1]{Ospite}
		\begin{umlsystem}[x=0, fill=lightgray!20]{Quizzipedia}
			\umlusecase[x=5, y=-1, fill=white, width=4cm, name=186]{\textbf{UC2.3:} Inserimento password}
			\umlassoc{Ospite}{186}
			\umlusecase[x=15, y=-1, fill=white, width=4cm, name=189]{\textbf{UC2.6:} Errore password troppo corta}
			\umlextend{189}{186}
		\end{umlsystem}
		\end{resizedtikzpicture}
		\caption{\textbf{UC2.3}: Inserimento password}
		\label{UC2.3}
	\end{figure}
\begin{description}
\item[Attori:] Ospite;
\item[Scopo e descrizione:] L'utente inserisce la password con la quale effettuare l'autenticazione
      \item[Precondizione:] L'ospite non è autenticato nel sistema e non ha un account presso il sistema;

        \item[Flusso principale degli eventi:] \ 
 \begin{enumerate}
          \item L'utente inserisce la password di almeno 8 caratteri con la quale effettuare l'autenticazione;

      \end{enumerate}
    \item[Estensioni:]
      \begin{enumerate}
          \item Se la password è troppo corta viene visualizzato un errore	 (\hyperlink{UC2.6}{UC2.6});

      \end{enumerate}
    \item[Postcondizione:] L'utente ha specificato la password.
  \end{description}
\hypertarget{UC2.4}{}
\subsection{Caso d'uso UC2.4: Errore nome completo troppo corto}
	\begin{figure}[H]
		\centering
		\begin{resizedtikzpicture}{\textwidth}
		\umlactor[x=0, y=-1]{Ospite}
		\begin{umlsystem}[x=0, fill=lightgray!20]{Quizzipedia}
			\umlusecase[x=5, y=-1, fill=white, width=4cm, name=187]{\textbf{UC2.4:} Errore nome completo troppo corto}
			\umlassoc{Ospite}{187}
		\end{umlsystem}
		\end{resizedtikzpicture}
		\caption{\textbf{UC2.4}: Errore nome completo troppo corto}
		\label{UC2.4}
	\end{figure}
\begin{description}
\item[Attori:] Ospite;
\item[Scopo e descrizione:] Viene visualizzato un messaggio di errore nel caso il nome completo sia inferiore ai 2 caratteri
      \item[Precondizione:] Il nome completo inserito dall'ospite è troppo corto;

        \item[Flusso principale degli eventi:] \ 
 \begin{enumerate}
          \item Viene visualizzato un messaggio di errore nel caso il nome completo sia inferiore ai 2 caratteri;

      \end{enumerate}
    \item[Postcondizione:] Viene visualizzato un errore.
  \end{description}
\hypertarget{UC2.5}{}
\subsection{Caso d'uso UC2.5: Errore username troppo corto}
	\begin{figure}[H]
		\centering
		\begin{resizedtikzpicture}{\textwidth}
		\umlactor[x=0, y=-1]{Ospite}
		\begin{umlsystem}[x=0, fill=lightgray!20]{Quizzipedia}
			\umlusecase[x=5, y=-1, fill=white, width=4cm, name=188]{\textbf{UC2.5:} Errore username troppo corto}
			\umlassoc{Ospite}{188}
		\end{umlsystem}
		\end{resizedtikzpicture}
		\caption{\textbf{UC2.5}: Errore username troppo corto}
		\label{UC2.5}
	\end{figure}
\begin{description}
\item[Attori:] Ospite;
\item[Scopo e descrizione:] Viene visualizzato un messaggio d'errore nel caso in cui l'ospite inserisca uno username inferiore ai 6 caratteri
      \item[Precondizione:] L'username inserito dall'ospite è troppo corto

;

        \item[Flusso principale degli eventi:] \ 
 \begin{enumerate}
          \item Viene visualizzato un messaggio d'errore nel caso in cui l'ospite inserisca uno username inferiore ai 6 caratteri;

      \end{enumerate}
    \item[Postcondizione:] Viene visualizzato un messaggio d'errore.
  \end{description}
\hypertarget{UC2.6}{}
\subsection{Caso d'uso UC2.6: Errore password troppo corta}
	\begin{figure}[H]
		\centering
		\begin{resizedtikzpicture}{\textwidth}
		\umlactor[x=0, y=-1]{Ospite}
		\begin{umlsystem}[x=0, fill=lightgray!20]{Quizzipedia}
			\umlusecase[x=5, y=-1, fill=white, width=4cm, name=189]{\textbf{UC2.6:} Errore password troppo corta}
			\umlassoc{Ospite}{189}
		\end{umlsystem}
		\end{resizedtikzpicture}
		\caption{\textbf{UC2.6}: Errore password troppo corta}
		\label{UC2.6}
	\end{figure}
\begin{description}
\item[Attori:] Ospite;
\item[Scopo e descrizione:] Viene visualizzato un messaggio d'errore nel caso in cui l'ospite inserisca una password inferiore agli 8 caratteri
      \item[Precondizione:] La password inserita dall'ospite è troppo corta
;

        \item[Flusso principale degli eventi:] \ 
 \begin{enumerate}
          \item Viene visualizzato un messaggio d'errore nel caso in cui l'ospite inserisca una password inferiore agli 8 caratteri;

      \end{enumerate}
    \item[Postcondizione:] Viene visualizzato un messaggio d'errore
.
  \end{description}
\hypertarget{UC2.7}{}
\subsection{Caso d'uso UC2.7: Errore username già utilizzato}
	\begin{figure}[H]
		\centering
		\begin{resizedtikzpicture}{\textwidth}
		\umlactor[x=0, y=-1]{Ospite}
		\begin{umlsystem}[x=0, fill=lightgray!20]{Quizzipedia}
			\umlusecase[x=5, y=-1, fill=white, width=4cm, name=190]{\textbf{UC2.7:} Errore username già utilizzato}
			\umlassoc{Ospite}{190}
		\end{umlsystem}
		\end{resizedtikzpicture}
		\caption{\textbf{UC2.7}: Errore username già utilizzato}
		\label{UC2.7}
	\end{figure}
\begin{description}
\item[Attori:] Ospite;
\item[Scopo e descrizione:] Viene visualizzato un messaggio d'errore nel caso in cui l'ospite inserisca un username già utilizzato da qualche altro utente
      \item[Precondizione:] L'ospite ha inserito un username già utilizzato da qualche altro utente;

        \item[Flusso principale degli eventi:] \ 
 \begin{enumerate}
          \item Viene visualizzato un messaggio d'errore nel caso in cui l'ospite inserisca un username già utilizzato da qualche altro utente;

      \end{enumerate}
    \item[Postcondizione:] Viene visualizzato un messaggio d'errore.
  \end{description}
\hypertarget{UC3}{}
\subsection{Caso d'uso UC3: Gestione profilo}
	\begin{figure}[H]
		\centering
		\begin{resizedtikzpicture}{\textwidth}
		\umlactor[x=0, y=-1]{Utente}
		\begin{umlsystem}[x=0, fill=lightgray!20]{Quizzipedia}
			\umlusecase[x=5, y=-5, fill=white, width=4cm, name=195]{\textbf{UC3.3:} Modifica password}
			\umlassoc{Utente}{195}
			\umlusecase[x=5, y=-1, fill=white, width=4cm, name=175]{\textbf{UC3.2:} Modifica username}
			\umlassoc{Utente}{175}
			\umlusecase[x=5, y=3, fill=white, width=4cm, name=41]{\textbf{UC3.1:} Modifica nome completo}
			\umlassoc{Utente}{41}
		\end{umlsystem}
		\end{resizedtikzpicture}
		\caption{\textbf{UC3}: Gestione profilo}
		\label{UC3}
	\end{figure}
\begin{description}
\item[Attori:] Utente;
\item[Scopo e descrizione:] L'utente gestisce il proprio profilo potendo cambiare il suo nome, aggiungendo o rimuovendo le connessioni agli account dei provider 
      \item[Precondizione:] L'utente è autenticato nel sistema;

        \item[Flusso principale degli eventi:] \ 
 \begin{enumerate}
          \item L’utente visualizza le proprie informazioni personali;
          \item L'utente può modificare il proprio nome completo (\hyperlink{UC3.1}{UC3.1});
          \item L'utente può modificare il proprio username	 (\hyperlink{UC3.2}{UC3.2});
          \item L'utente può modificare la propria password (\hyperlink{UC3.3}{UC3.3});

      \end{enumerate}
    \item[Postcondizione:] Il sistema ha apportato le modifiche al profilo dell'utente.
  \end{description}
\hypertarget{UC3.1}{}
\subsection{Caso d'uso UC3.1: Modifica nome completo}
	\begin{figure}[H]
		\centering
		\begin{resizedtikzpicture}{\textwidth}
		\umlactor[x=0, y=-1]{Utente}
		\begin{umlsystem}[x=0, fill=lightgray!20]{Quizzipedia}
			\umlusecase[x=5, y=-1, fill=white, width=4cm, name=41]{\textbf{UC3.1:} Modifica nome completo}
			\umlassoc{Utente}{41}
			\umlusecase[x=15, y=-1, fill=white, width=4cm, name=197]{\textbf{UC3.5:} Errore nome completo troppo corto}
			\umlextend{197}{41}
		\end{umlsystem}
		\end{resizedtikzpicture}
		\caption{\textbf{UC3.1}: Modifica nome completo}
		\label{UC3.1}
	\end{figure}
\begin{description}
\item[Attori:] Utente;
\item[Scopo e descrizione:] L'utente modifica il proprio nome completo
      \item[Precondizione:] L'utente è autenticato nel sistema;

        \item[Flusso principale degli eventi:] \ 
 \begin{enumerate}
          \item L’utente modifica il proprio nome completo;

      \end{enumerate}
    \item[Estensioni:]
      \begin{enumerate}
          \item Se il nome completo è troppo corto viene visualizzato un errore (\hyperlink{UC3.5}{UC3.5});

      \end{enumerate}
    \item[Postcondizione:] L'utente ha modificato il proprio nome completo.
  \end{description}
\hypertarget{UC3.2}{}
\subsection{Caso d'uso UC3.2: Modifica username}
	\begin{figure}[H]
		\centering
		\begin{resizedtikzpicture}{\textwidth}
		\umlactor[x=0, y=-1]{Utente}
		\begin{umlsystem}[x=0, fill=lightgray!20]{Quizzipedia}
			\umlusecase[x=5, y=-1, fill=white, width=4cm, name=175]{\textbf{UC3.2:} Modifica username}
			\umlassoc{Utente}{175}
			\umlusecase[x=15, y=-3, fill=white, width=4cm, name=198]{\textbf{UC3.6:} Errore username già utilizzato}
			\umlextend{198}{175}
			\umlusecase[x=15, y=1, fill=white, width=4cm, name=196]{\textbf{UC3.4:} Errore username troppo corto}
			\umlextend{196}{175}
		\end{umlsystem}
		\end{resizedtikzpicture}
		\caption{\textbf{UC3.2}: Modifica username}
		\label{UC3.2}
	\end{figure}
\begin{description}
\item[Attori:] Utente;
\item[Scopo e descrizione:] L'utente modifica il proprio username che deve contenere almeno 6 caratteri
      \item[Precondizione:] L'utente è autenticato nel sistema;

        \item[Flusso principale degli eventi:] \ 
 \begin{enumerate}
          \item L’utente modifica il proprio username;

      \end{enumerate}
    \item[Estensioni:]
      \begin{enumerate}
          \item Se l'username è troppo corto viene visualizzato un errore (\hyperlink{UC3.4}{UC3.4});
          \item Se l'username è già stato utilizzato viene visualizzato un errore (\hyperlink{UC3.6}{UC3.6});

      \end{enumerate}
    \item[Postcondizione:] L'utente ha modificato il proprio username.
  \end{description}
\hypertarget{UC3.3}{}
\subsection{Caso d'uso UC3.3: Modifica password}
	\begin{figure}[H]
		\centering
		\begin{resizedtikzpicture}{\textwidth}
		\umlactor[x=0, y=-1]{Utente}
		\begin{umlsystem}[x=0, fill=lightgray!20]{Quizzipedia}
			\umlusecase[x=5, y=-1, fill=white, width=4cm, name=207]{\textbf{UC3.3.1:} Inserimento vecchia password}
			\umlassoc{Utente}{207}
		\end{umlsystem}
		\end{resizedtikzpicture}
		\caption{\textbf{UC3.3}: Modifica password}
		\label{UC3.3}
	\end{figure}
\begin{description}
\item[Attori:] Utente;
\item[Scopo e descrizione:] L'utente modifica la propria password che deve contenere almeno 8 caratteri
      \item[Precondizione:] L'utente è autenticato nel sistema
;

        \item[Flusso principale degli eventi:] \ 
 \begin{enumerate}
          \item L'utente inserisce la vecchia password (\hyperlink{UC3.3.1}{UC3.3.1});
          \item L'utente modifica la propria password che deve contenere almeno 8 caratteri;

      \end{enumerate}
    \item[Estensioni:]
      \begin{enumerate}
          \item Se la password è troppo corta viene visualizzato un errore	 (\hyperlink{UC3.7}{UC3.7});

      \end{enumerate}
    \item[Postcondizione:] L'utente ha modificato il proprio nome completo.
  \end{description}
\hypertarget{UC3.3.1}{}
\subsection{Caso d'uso UC3.3.1: Inserimento vecchia password}
	\begin{figure}[H]
		\centering
		\begin{resizedtikzpicture}{\textwidth}
		\umlactor[x=0, y=-1]{Utente}
		\begin{umlsystem}[x=0, fill=lightgray!20]{Quizzipedia}
			\umlusecase[x=5, y=-1, fill=white, width=4cm, name=207]{\textbf{UC3.3.1:} Inserimento vecchia password}
			\umlassoc{Utente}{207}
			\umlusecase[x=15, y=-1.25, fill=white, width=4cm, name=208]{\textbf{UC3.3.2:} Errore password non corrispondenti}
			\umlextend{208}{207}
		\end{umlsystem}
		\end{resizedtikzpicture}
		\caption{\textbf{UC3.3.1}: Inserimento vecchia password}
		\label{UC3.3.1}
	\end{figure}
\begin{description}
\item[Attori:] Utente;
\item[Scopo e descrizione:] L'utente inserisce la sua vecchia password
      \item[Precondizione:] L'utente è autenticato nel sistema;

        \item[Flusso principale degli eventi:] \ 
 \begin{enumerate}
          \item L'utente inserisce la sua vecchia password corretta;

      \end{enumerate}
    \item[Estensioni:]
      \begin{enumerate}
          \item Se alla richiesta di inserimento della vecchia password essa non corrisponde con quella presente nel sistema viene visualizzato un messaggio d'errore (\hyperlink{UC3.3.2}{UC3.3.2});

      \end{enumerate}
    \item[Postcondizione:] L'utente ha inserito la sua vecchia password corretta.
  \end{description}
\hypertarget{UC3.3.2}{}
\subsection{Caso d'uso UC3.3.2: Errore password non corrispondenti}
	\begin{figure}[H]
		\centering
		\begin{resizedtikzpicture}{\textwidth}
		\umlactor[x=0, y=-1]{Utente}
		\begin{umlsystem}[x=0, fill=lightgray!20]{Quizzipedia}
			\umlusecase[x=5, y=-1.25, fill=white, width=4cm, name=208]{\textbf{UC3.3.2:} Errore password non corrispondenti}
			\umlassoc{Utente}{208}
		\end{umlsystem}
		\end{resizedtikzpicture}
		\caption{\textbf{UC3.3.2}: Errore password non corrispondenti}
		\label{UC3.3.2}
	\end{figure}
\begin{description}
\item[Attori:] Utente;
\item[Scopo e descrizione:] La vecchia password inserita dall'utente non è corretta
      \item[Precondizione:] L'utente ha inserito la sua vecchia password ma è errata;

        \item[Flusso principale degli eventi:] \ 
 \begin{enumerate}
          \item La vecchia password inserita dall'utente non è corretta
;

      \end{enumerate}
    \item[Postcondizione:] Viene visualizzato un messaggio d'errore.
  \end{description}
\hypertarget{UC3.4}{}
\subsection{Caso d'uso UC3.4: Errore username troppo corto}
	\begin{figure}[H]
		\centering
		\begin{resizedtikzpicture}{\textwidth}
		\umlactor[x=0, y=-1]{Utente}
		\begin{umlsystem}[x=0, fill=lightgray!20]{Quizzipedia}
			\umlusecase[x=5, y=-1, fill=white, width=4cm, name=196]{\textbf{UC3.4:} Errore username troppo corto}
			\umlassoc{Utente}{196}
		\end{umlsystem}
		\end{resizedtikzpicture}
		\caption{\textbf{UC3.4}: Errore username troppo corto}
		\label{UC3.4}
	\end{figure}
\begin{description}
\item[Attori:] Utente;
\item[Scopo e descrizione:] Viene visualizzato un messaggio d'errore nel caso in cui l'utente inserisca uno username inferiore ai 6 caratteri
      \item[Precondizione:] L'username inserito dall'utente è troppo corto;

        \item[Flusso principale degli eventi:] \ 
 \begin{enumerate}
          \item Viene visualizzato un messaggio d'errore nel caso in cui l'utente inserisca un username inferiore ai 6 caratteri;

      \end{enumerate}
    \item[Postcondizione:] Viene visualizzato un messaggio d'errore.
  \end{description}
\hypertarget{UC3.5}{}
\subsection{Caso d'uso UC3.5: Errore nome completo troppo corto}
	\begin{figure}[H]
		\centering
		\begin{resizedtikzpicture}{\textwidth}
		\umlactor[x=0, y=-1]{Utente}
		\begin{umlsystem}[x=0, fill=lightgray!20]{Quizzipedia}
			\umlusecase[x=5, y=-1, fill=white, width=4cm, name=197]{\textbf{UC3.5:} Errore nome completo troppo corto}
			\umlassoc{Utente}{197}
		\end{umlsystem}
		\end{resizedtikzpicture}
		\caption{\textbf{UC3.5}: Errore nome completo troppo corto}
		\label{UC3.5}
	\end{figure}
\begin{description}
\item[Attori:] Utente;
\item[Scopo e descrizione:] Viene visualizzato un messaggio di errore nel caso il nome completo sia inferiore ai 2 caratteri

      \item[Precondizione:] Il nome completo inserito dall'ospite è troppo corto
;

        \item[Flusso principale degli eventi:] \ 
 \begin{enumerate}
          \item Viene visualizzato un messaggio di errore nel caso il nome completo sia inferiore ai 2 caratteri	;

      \end{enumerate}
    \item[Postcondizione:] Viene visualizzato un errore
.
  \end{description}
\hypertarget{UC3.6}{}
\subsection{Caso d'uso UC3.6: Errore username già utilizzato}
	\begin{figure}[H]
		\centering
		\begin{resizedtikzpicture}{\textwidth}
		\umlactor[x=0, y=-1]{Utente}
		\begin{umlsystem}[x=0, fill=lightgray!20]{Quizzipedia}
			\umlusecase[x=5, y=-1, fill=white, width=4cm, name=198]{\textbf{UC3.6:} Errore username già utilizzato}
			\umlassoc{Utente}{198}
		\end{umlsystem}
		\end{resizedtikzpicture}
		\caption{\textbf{UC3.6}: Errore username già utilizzato}
		\label{UC3.6}
	\end{figure}
\begin{description}
\item[Attori:] Utente;
\item[Scopo e descrizione:] Viene visualizzato un messaggio d'errore nel caso in cui l'utente inserisca un username già utilizzato da qualche altro utente
      \item[Precondizione:] L'utente ha inserito un username già utilizzato da qualche altro utente;

        \item[Flusso principale degli eventi:] \ 
 \begin{enumerate}
          \item Viene visualizzato un messaggio d'errore nel caso in cui l'utente inserisca un username già utilizzato da qualche altro utente;

      \end{enumerate}
    \item[Postcondizione:] Viene visualizzato un messaggio d'errore.
  \end{description}
\hypertarget{UC3.7}{}
\subsection{Caso d'uso UC3.7: Errore password troppo corta}
	\begin{figure}[H]
		\centering
		\begin{resizedtikzpicture}{\textwidth}
		\umlactor[x=0, y=-1]{Utente}
		\begin{umlsystem}[x=0, fill=lightgray!20]{Quizzipedia}
			\umlusecase[x=5, y=-1, fill=white, width=4cm, name=199]{\textbf{UC3.7:} Errore password troppo corta}
			\umlassoc{Utente}{199}
		\end{umlsystem}
		\end{resizedtikzpicture}
		\caption{\textbf{UC3.7}: Errore password troppo corta}
		\label{UC3.7}
	\end{figure}
\begin{description}
\item[Attori:] Utente;
\item[Scopo e descrizione:] Viene visualizzato un messaggio d'errore nel caso in cui l'ospite inserisca una password inferiore agli 8 caratteri

      \item[Precondizione:] La password inserita dall'ospite è troppo corta
;

        \item[Flusso principale degli eventi:] \ 
 \begin{enumerate}
          \item Viene visualizzato un messaggio d'errore nel caso in cui l'ospite inserisca una password inferiore agli 8 caratteri	;

      \end{enumerate}
    \item[Postcondizione:] Viene visualizzato un messaggio d'errore
.
  \end{description}
\hypertarget{UC4}{}
\subsection{Caso d'uso UC4: Logout}
	\begin{figure}[H]
		\centering
		\begin{resizedtikzpicture}{\textwidth}
		\umlactor[x=0, y=-1]{Utente}
		\begin{umlsystem}[x=0, fill=lightgray!20]{Quizzipedia}
			\umlusecase[x=5, y=-0.75, fill=white, width=4cm, name=5]{\textbf{UC4:} Logout}
			\umlassoc{Utente}{5}
		\end{umlsystem}
		\end{resizedtikzpicture}
		\caption{\textbf{UC4}: Logout}
		\label{UC4}
	\end{figure}
\begin{description}
\item[Attori:] Utente;
\item[Scopo e descrizione:] L'utente effettua il logout
      \item[Precondizione:] L'utente è autenticato nel sistema;

        \item[Flusso principale degli eventi:] \ 
 \begin{enumerate}
          \item L'utente seleziona la funzionalità di Logout;

      \end{enumerate}
    \item[Postcondizione:] L'utente non è più autenticato nel sistema.
  \end{description}
\hypertarget{UC5}{}
\subsection{Caso d'uso UC5: Gestione domande}
	\begin{figure}[H]
		\centering
		\begin{resizedtikzpicture}{\textwidth}
		\umlactor[x=0, y=-1]{Docente}
		\begin{umlsystem}[x=0, fill=lightgray!20]{Quizzipedia}
			\umlusecase[x=5, y=-5, fill=white, width=4cm, name=50]{\textbf{UC5.3:} Elimina domanda}
			\umlassoc{Docente}{50}
			\umlusecase[x=5, y=-1, fill=white, width=4cm, name=46]{\textbf{UC5.2:} Modifica domanda}
			\umlassoc{Docente}{46}
			\umlusecase[x=5, y=3, fill=white, width=4cm, name=43]{\textbf{UC5.1:} Inserimento domanda}
			\umlassoc{Docente}{43}
		\end{umlsystem}
		\end{resizedtikzpicture}
		\caption{\textbf{UC5}: Gestione domande}
		\label{UC5}
	\end{figure}
\begin{description}
\item[Attori:] Docente;
\item[Scopo e descrizione:] Il docente gestisce le proprie domande
      \item[Precondizione:] Il docente è autenticato nel sistema;

        \item[Flusso principale degli eventi:] \ 
 \begin{enumerate}
          \item Il docente può creare una nuova domanda (\hyperlink{UC5.1}{UC5.1});
          \item Il docente può modificare una domanda (\hyperlink{UC5.2}{UC5.2});
          \item Il docente può eliminare una domanda (\hyperlink{UC5.3}{UC5.3});

      \end{enumerate}
    \item[Postcondizione:] Il sistema ha ottenuto le informazioni sulle operazioni che il docente desidera eseguire sulle domande.
  \end{description}
\hypertarget{UC5.1}{}
\subsection{Caso d'uso UC5.1: Inserimento domanda}
	\begin{figure}[H]
		\centering
		\begin{resizedtikzpicture}{\textwidth}
		\umlactor[x=0, y=-1]{Docente}
		\begin{umlsystem}[x=0, fill=lightgray!20]{Quizzipedia}
			\umlusecase[x=5, y=-5.75, fill=white, width=4cm, name=162]{\textbf{UC5.1.3:} Scrittura nuova domanda da interfaccia grafica}
			\umlassoc{Docente}{162}
			\umlusecase[x=5, y=-1.25, fill=white, width=4cm, name=114]{\textbf{UC5.1.2:} Scrittura domanda in QML della nuova domanda}
			\umlassoc{Docente}{114}
			\umlusecase[x=5, y=3.25, fill=white, width=4cm, name=112]{\textbf{UC5.1.1:} Selezione argomenti nuova domanda}
			\umlassoc{Docente}{112}
		\end{umlsystem}
		\end{resizedtikzpicture}
		\caption{\textbf{UC5.1}: Inserimento domanda}
		\label{UC5.1}
	\end{figure}
\begin{description}
\item[Attori:] Docente;
\item[Scopo e descrizione:] Il docente compone una domanda in linguaggio QML, che verrà salvata nel sistema e potrà essere utilizzata nei questionari
      \item[Precondizione:] Il docente è autenticato nel sistema;

        \item[Flusso principale degli eventi:] \ 
 \begin{enumerate}
          \item Il docente seleziona gli argomenti della domanda (\hyperlink{UC5.1.1}{UC5.1.1});
          \item Il docente compone la domanda in QML  (\hyperlink{UC5.1.2}{UC5.1.2});
          \item Il docente compone la domanda attraverso l'interfaccia grafica (\hyperlink{UC5.1.3}{UC5.1.3});
          \item Il docente conferma la creazione della domanda;

      \end{enumerate}
    \item[Estensioni:]
      \begin{enumerate}
          \item Se il codice QML inserito non è valido viene visualizzato un messaggio di errore (\hyperlink{UC5.4}{UC5.4});
          \item Se non è stato selezionato nessun argomento viene visualizzato un messaggio di errore (\hyperlink{UC5.5}{UC5.5});

      \end{enumerate}
    \item[Postcondizione:] È stata creata una nuova domanda.
  \end{description}
\hypertarget{UC5.1.1}{}
\subsection{Caso d'uso UC5.1.1: Selezione argomenti nuova domanda}
	\begin{figure}[H]
		\centering
		\begin{resizedtikzpicture}{\textwidth}
		\umlactor[x=0, y=-1]{Docente}
		\begin{umlsystem}[x=0, fill=lightgray!20]{Quizzipedia}
			\umlusecase[x=5, y=-1.25, fill=white, width=4cm, name=112]{\textbf{UC5.1.1:} Selezione argomenti nuova domanda}
			\umlassoc{Docente}{112}
		\end{umlsystem}
		\end{resizedtikzpicture}
		\caption{\textbf{UC5.1.1}: Selezione argomenti nuova domanda}
		\label{UC5.1.1}
	\end{figure}
\begin{description}
\item[Attori:] Docente;
\item[Scopo e descrizione:] Il docente seleziona gli argomenti corrispondenti alla domanda selezionata
      \item[Precondizione:] Il docente sta creando una nuova domanda;

        \item[Flusso principale degli eventi:] \ 
 \begin{enumerate}
          \item Il docente seleziona gli argomenti della domanda;

      \end{enumerate}
    \item[Postcondizione:] Il docente ha definito gli argomenti per classificare la domanda
.
  \end{description}
\hypertarget{UC5.1.2}{}
\subsection{Caso d'uso UC5.1.2: Scrittura domanda in QML della nuova domanda}
	\begin{figure}[H]
		\centering
		\begin{resizedtikzpicture}{\textwidth}
		\umlactor[x=0, y=-1]{Docente}
		\begin{umlsystem}[x=0, fill=lightgray!20]{Quizzipedia}
			\umlusecase[x=5, y=-1.25, fill=white, width=4cm, name=114]{\textbf{UC5.1.2:} Scrittura domanda in QML della nuova domanda}
			\umlassoc{Docente}{114}
		\end{umlsystem}
		\end{resizedtikzpicture}
		\caption{\textbf{UC5.1.2}: Scrittura domanda in QML della nuova domanda}
		\label{UC5.1.2}
	\end{figure}
\begin{description}
\item[Attori:] Docente;
\item[Scopo e descrizione:] Il docente compone una domanda in linguaggio QML
      \item[Precondizione:] Il docente sta creando una nuova domanda;

        \item[Flusso principale degli eventi:] \ 
 \begin{enumerate}
          \item Il docente compone la domanda in QML;

      \end{enumerate}
    \item[Postcondizione:] La domanda è stata scritta in linguaggio QML.
  \end{description}
\hypertarget{UC5.1.3}{}
\subsection{Caso d'uso UC5.1.3: Scrittura nuova domanda da interfaccia grafica}
	\begin{figure}[H]
		\centering
		\begin{resizedtikzpicture}{\textwidth}
		\umlactor[x=0, y=-1]{Docente}
		\begin{umlsystem}[x=0, fill=lightgray!20]{Quizzipedia}
			\umlusecase[x=5, y=-1.25, fill=white, width=4cm, name=162]{\textbf{UC5.1.3:} Scrittura nuova domanda da interfaccia grafica}
			\umlassoc{Docente}{162}
		\end{umlsystem}
		\end{resizedtikzpicture}
		\caption{\textbf{UC5.1.3}: Scrittura nuova domanda da interfaccia grafica}
		\label{UC5.1.3}
	\end{figure}
\begin{description}
\item[Attori:] Docente;
\item[Scopo e descrizione:] Il docente compone la domanda attraverso un interfaccia grafica
      \item[Precondizione:] Il docente sta creando una nuova domanda;

        \item[Flusso principale degli eventi:] \ 
 \begin{enumerate}
          \item Il docente compone la domanda attraverso l'interfaccia grafica;

      \end{enumerate}
    \item[Postcondizione:] La domanda è stata composta attraverso l'interfaccia grafica.
  \end{description}
\hypertarget{UC5.2}{}
\subsection{Caso d'uso UC5.2: Modifica domanda}
	\begin{figure}[H]
		\centering
		\begin{resizedtikzpicture}{\textwidth}
		\umlactor[x=0, y=-1]{Docente}
		\begin{umlsystem}[x=0, fill=lightgray!20]{Quizzipedia}
			\umlusecase[x=5, y=-6, fill=white, width=4cm, name=165]{\textbf{UC5.2.3:} Modifica domanda da interfaccia grafica}
			\umlassoc{Docente}{165}
			\umlusecase[x=5, y=-1.5, fill=white, width=4cm, name=164]{\textbf{UC5.2.2:} Scrittura domanda in QML della domanda da modificare}
			\umlassoc{Docente}{164}
			\umlusecase[x=5, y=3.5, fill=white, width=4cm, name=163]{\textbf{UC5.2.1:} Selezione argomenti modifica domanda}
			\umlassoc{Docente}{163}
		\end{umlsystem}
		\end{resizedtikzpicture}
		\caption{\textbf{UC5.2}: Modifica domanda}
		\label{UC5.2}
	\end{figure}
\begin{description}
\item[Attori:] Docente;
\item[Scopo e descrizione:] Il docente effettua delle modifiche ad una domanda 
      \item[Precondizione:] Il docente è autenticato nel sistema;

        \item[Flusso principale degli eventi:] \ 
 \begin{enumerate}
          \item Ricerca della domanda da modificare (\hyperlink{UC22}{UC22});
          \item Selezione della domanda da modificare;
          \item Il docente seleziona gli argomenti della domanda	 (\hyperlink{UC5.2.1}{UC5.2.1});
          \item Il docente modifica la domanda in QML	 (\hyperlink{UC5.2.2}{UC5.2.2});
          \item Il docente compone la domanda attraverso l'interfaccia grafica (\hyperlink{UC5.2.3}{UC5.2.3});
          \item Conferma delle modifiche;

      \end{enumerate}
    \item[Estensioni:]
      \begin{enumerate}
          \item Se il codice QML inserito non è valido viene visualizzato un messaggio di errore (\hyperlink{UC5.4}{UC5.4});
          \item Se non è stato selezionato nessun argomento viene visualizzato un messaggio di errore (\hyperlink{UC5.5}{UC5.5});

      \end{enumerate}
    \item[Postcondizione:] La domanda è stata modificata.
  \end{description}
\hypertarget{UC5.2.1}{}
\subsection{Caso d'uso UC5.2.1: Selezione argomenti modifica domanda}
	\begin{figure}[H]
		\centering
		\begin{resizedtikzpicture}{\textwidth}
		\umlactor[x=0, y=-1]{Docente}
		\begin{umlsystem}[x=0, fill=lightgray!20]{Quizzipedia}
			\umlusecase[x=5, y=-1.25, fill=white, width=4cm, name=163]{\textbf{UC5.2.1:} Selezione argomenti modifica domanda}
			\umlassoc{Docente}{163}
		\end{umlsystem}
		\end{resizedtikzpicture}
		\caption{\textbf{UC5.2.1}: Selezione argomenti modifica domanda}
		\label{UC5.2.1}
	\end{figure}
\begin{description}
\item[Attori:] Docente;
\item[Scopo e descrizione:] Il docente seleziona gli argomenti corrispondenti alla domanda selezionata
      \item[Precondizione:] Il docente sta modificando una domanda
;

        \item[Flusso principale degli eventi:] \ 
 \begin{enumerate}
          \item Il docente seleziona gli argomenti della domanda;

      \end{enumerate}
    \item[Postcondizione:] Il docente ha definito gli argomenti per classificare la domanda.
  \end{description}
\hypertarget{UC5.2.2}{}
\subsection{Caso d'uso UC5.2.2: Scrittura domanda in QML della domanda da modificare}
	\begin{figure}[H]
		\centering
		\begin{resizedtikzpicture}{\textwidth}
		\umlactor[x=0, y=-1]{Docente}
		\begin{umlsystem}[x=0, fill=lightgray!20]{Quizzipedia}
			\umlusecase[x=5, y=-1.5, fill=white, width=4cm, name=164]{\textbf{UC5.2.2:} Scrittura domanda in QML della domanda da modificare}
			\umlassoc{Docente}{164}
		\end{umlsystem}
		\end{resizedtikzpicture}
		\caption{\textbf{UC5.2.2}: Scrittura domanda in QML della domanda da modificare}
		\label{UC5.2.2}
	\end{figure}
\begin{description}
\item[Attori:] Docente;
\item[Scopo e descrizione:] Il docente compone una domanda in linguaggio QML
      \item[Precondizione:] Il docente sta modificando una domanda;

        \item[Flusso principale degli eventi:] \ 
 \begin{enumerate}
          \item Il docente compone la domanda in QML;

      \end{enumerate}
    \item[Postcondizione:] La domanda è stata scritta in linguaggio QML.
  \end{description}
\hypertarget{UC5.2.3}{}
\subsection{Caso d'uso UC5.2.3: Modifica domanda da interfaccia grafica}
	\begin{figure}[H]
		\centering
		\begin{resizedtikzpicture}{\textwidth}
		\umlactor[x=0, y=-1]{Docente}
		\begin{umlsystem}[x=0, fill=lightgray!20]{Quizzipedia}
			\umlusecase[x=5, y=-1.25, fill=white, width=4cm, name=165]{\textbf{UC5.2.3:} Modifica domanda da interfaccia grafica}
			\umlassoc{Docente}{165}
		\end{umlsystem}
		\end{resizedtikzpicture}
		\caption{\textbf{UC5.2.3}: Modifica domanda da interfaccia grafica}
		\label{UC5.2.3}
	\end{figure}
\begin{description}
\item[Attori:] Docente;
\item[Scopo e descrizione:] Il docente modifica la domanda attraverso un interfaccia grafica

      \item[Precondizione:] Il docente sta modificando una domanda
;

        \item[Flusso principale degli eventi:] \ 
 \begin{enumerate}
          \item Il docente modifica la domanda attraverso l'interfaccia grafica;

      \end{enumerate}
    \item[Postcondizione:] La domanda è stata modificata attraverso l'interfaccia grafica.
  \end{description}
\hypertarget{UC5.3}{}
\subsection{Caso d'uso UC5.3: Elimina domanda}
	\begin{figure}[H]
		\centering
		\begin{resizedtikzpicture}{\textwidth}
		\umlactor[x=0, y=-1]{Docente}
		\begin{umlsystem}[x=0, fill=lightgray!20]{Quizzipedia}
			\umlusecase[x=5, y=-1, fill=white, width=4cm, name=50]{\textbf{UC5.3:} Elimina domanda}
			\umlassoc{Docente}{50}
			\umlusecase[x=15, y=-1.25, fill=white, width=4cm, name=206]{\textbf{UC5.13:} Errore eliminazione domanda utilizzata}
			\umlextend{206}{50}
		\end{umlsystem}
		\end{resizedtikzpicture}
		\caption{\textbf{UC5.3}: Elimina domanda}
		\label{UC5.3}
	\end{figure}
\begin{description}
\item[Attori:] Docente;
\item[Scopo e descrizione:] Il docente rimuove dal sistema una domanda da lui creata
      \item[Precondizione:] Il docente è autenticato nel sistema;

        \item[Flusso principale degli eventi:] \ 
 \begin{enumerate}
          \item Ricerca della domanda da eliminare (\hyperlink{UC22}{UC22});
          \item Selezione della domanda da eliminare;
          \item Conferma per l'eliminazione della domanda	;

      \end{enumerate}
    \item[Estensioni:]
      \begin{enumerate}
          \item Se il docente tenta di eliminare una domanda utilizzata da almeno un questionario viene mostrato un messaggio di errore (\hyperlink{UC5.13}{UC5.13});

      \end{enumerate}
    \item[Postcondizione:] La domanda è stata eliminata.
  \end{description}
\hypertarget{UC5.4}{}
\subsection{Caso d'uso UC5.4: Errore QML non valido}
	\begin{figure}[H]
		\centering
		\begin{resizedtikzpicture}{\textwidth}
		\umlactor[x=0, y=-1]{Docente}
		\begin{umlsystem}[x=0, fill=lightgray!20]{Quizzipedia}
			\umlusecase[x=5, y=-1, fill=white, width=4cm, name=142]{\textbf{UC5.4:} Errore QML non valido}
			\umlassoc{Docente}{142}
		\end{umlsystem}
		\end{resizedtikzpicture}
		\caption{\textbf{UC5.4}: Errore QML non valido}
		\label{UC5.4}
	\end{figure}
\begin{description}
\item[Attori:] Docente;
\item[Scopo e descrizione:] Il sistema avvisa il docente di uno o più errori nel codice QML inserito
      \item[Precondizione:] Il QML non è valido;

        \item[Flusso principale degli eventi:] \ 
 \begin{enumerate}
          \item Viene visualizzato un messaggio d'errore;

      \end{enumerate}
    \item[Postcondizione:] Non viene inserita la domanda.
  \end{description}
\hypertarget{UC5.5}{}
\subsection{Caso d'uso UC5.5: Errore argomento mancante}
	\begin{figure}[H]
		\centering
		\begin{resizedtikzpicture}{\textwidth}
		\umlactor[x=0, y=-1]{Docente}
		\begin{umlsystem}[x=0, fill=lightgray!20]{Quizzipedia}
			\umlusecase[x=5, y=-1, fill=white, width=4cm, name=143]{\textbf{UC5.5:} Errore argomento mancante}
			\umlassoc{Docente}{143}
		\end{umlsystem}
		\end{resizedtikzpicture}
		\caption{\textbf{UC5.5}: Errore argomento mancante}
		\label{UC5.5}
	\end{figure}
\begin{description}
\item[Attori:] Docente;
\item[Scopo e descrizione:] Il sistema avvisa il docente che deve essere inserito almeno un argomento
      \item[Precondizione:] Non è stato selezionato almeno un argomento;

        \item[Flusso principale degli eventi:] \ 
 \begin{enumerate}
          \item Viene visualizzato un messaggio d'errore;

      \end{enumerate}
    \item[Postcondizione:] Non viene inserita la domanda.
  \end{description}
\hypertarget{UC5.6}{}
\subsection{Caso d'uso UC5.6: Inserimento domanda di tipo vero/falso}
	\begin{figure}[H]
		\centering
		\begin{resizedtikzpicture}{\textwidth}
		\umlactor[x=0, y=-1]{Docente}
		\begin{umlsystem}[x=0, fill=lightgray!20]{Quizzipedia}
			\umlusecase[x=5, y=-1.25, fill=white, width=4cm, name=148]{\textbf{UC5.6:} Inserimento domanda di tipo vero/falso}
			\umlassoc{Docente}{148}
		\end{umlsystem}
		\end{resizedtikzpicture}
		\caption{\textbf{UC5.6}: Inserimento domanda di tipo vero/falso}
		\label{UC5.6}
	\end{figure}
\begin{description}
\item[Attori:] Docente;
\item[Scopo e descrizione:] Il docente inserisce una domanda di tipo vero/falso
      \item[Precondizione:] Il docente è autenticato presso il sistema;

        \item[Flusso principale degli eventi:] \ 
 \begin{enumerate}
          \item Il docente seleziona gli argomenti della domanda (\hyperlink{UC5.1.1}{UC5.1.1});
          \item Il docente compone la domanda vero/falso in QML specificando la risposta esatta (\hyperlink{UC5.1.2}{UC5.1.2});
          \item Il docente conferma la creazione della domanda;

      \end{enumerate}
    \item[Postcondizione:] È stata creata una nuova domanda di tipo vero/falso.
  \end{description}
\hypertarget{UC5.7}{}
\subsection{Caso d'uso UC5.7: Inserimento domanda a scelta multipla}
	\begin{figure}[H]
		\centering
		\begin{resizedtikzpicture}{\textwidth}
		\umlactor[x=0, y=-1]{Docente}
		\begin{umlsystem}[x=0, fill=lightgray!20]{Quizzipedia}
			\umlusecase[x=5, y=-1.25, fill=white, width=4cm, name=149]{\textbf{UC5.7:} Inserimento domanda a scelta multipla}
			\umlassoc{Docente}{149}
		\end{umlsystem}
		\end{resizedtikzpicture}
		\caption{\textbf{UC5.7}: Inserimento domanda a scelta multipla}
		\label{UC5.7}
	\end{figure}
\begin{description}
\item[Attori:] Docente;
\item[Scopo e descrizione:] Il docente inserisce una domanda a scelta multipla
      \item[Precondizione:] Il docente è autenticato presso il sistema;

        \item[Flusso principale degli eventi:] \ 
 \begin{enumerate}
          \item Il docente seleziona gli argomenti della domanda (\hyperlink{UC5.1.1}{UC5.1.1});
          \item Il docente compone la domanda a scelta multipla in QML specificando testo della domanda, risposte possibili e risposta corretta (\hyperlink{UC5.1.2}{UC5.1.2});
          \item Il docente conferma la creazione della domanda;

      \end{enumerate}
    \item[Postcondizione:] È stata creata una nuova domanda a scelta multipla.
  \end{description}
\hypertarget{UC5.8}{}
\subsection{Caso d'uso UC5.8: Inserimento domanda a risposta multipla}
	\begin{figure}[H]
		\centering
		\begin{resizedtikzpicture}{\textwidth}
		\umlactor[x=0, y=-1]{Docente}
		\begin{umlsystem}[x=0, fill=lightgray!20]{Quizzipedia}
			\umlusecase[x=5, y=-1.25, fill=white, width=4cm, name=150]{\textbf{UC5.8:} Inserimento domanda a risposta multipla}
			\umlassoc{Docente}{150}
		\end{umlsystem}
		\end{resizedtikzpicture}
		\caption{\textbf{UC5.8}: Inserimento domanda a risposta multipla}
		\label{UC5.8}
	\end{figure}
\begin{description}
\item[Attori:] Docente;
\item[Scopo e descrizione:] Il docente inserisce una domanda a risposta multipla
      \item[Precondizione:] Il docente è autenticato presso il sistema;

        \item[Flusso principale degli eventi:] \ 
 \begin{enumerate}
          \item Il docente seleziona gli argomenti della domanda (\hyperlink{UC5.1.1}{UC5.1.1});
          \item Il docente compone la domanda a risposta multipla in QML specificando il testo della domanda, risposte possibili e risposta/e esatte (\hyperlink{UC5.1.2}{UC5.1.2});
          \item Il docente conferma la creazione della domanda;

      \end{enumerate}
    \item[Postcondizione:] È stata creata una nuova domanda a risposta multipla.
  \end{description}
\hypertarget{UC5.9}{}
\subsection{Caso d'uso UC5.9: Inserimento domanda di tipo testo con parole omesse}
	\begin{figure}[H]
		\centering
		\begin{resizedtikzpicture}{\textwidth}
		\umlactor[x=0, y=-1]{Docente}
		\begin{umlsystem}[x=0, fill=lightgray!20]{Quizzipedia}
			\umlusecase[x=5, y=-1.25, fill=white, width=4cm, name=151]{\textbf{UC5.9:} Inserimento domanda di tipo testo con parole omesse}
			\umlassoc{Docente}{151}
		\end{umlsystem}
		\end{resizedtikzpicture}
		\caption{\textbf{UC5.9}: Inserimento domanda di tipo testo con parole omesse}
		\label{UC5.9}
	\end{figure}
\begin{description}
\item[Attori:] Docente;
\item[Scopo e descrizione:] Il docente inserisce una domanda di tipo testo con parole omesse
      \item[Precondizione:] Il docente è autenticato presso il sistema;

        \item[Flusso principale degli eventi:] \ 
 \begin{enumerate}
          \item Il docente seleziona gli argomenti della domanda (\hyperlink{UC5.1.1}{UC5.1.1});
          \item Il docente compone il testo della domanda in QML specificando quali parole possano essere scelte e dove vadano messe  (\hyperlink{UC5.1.2}{UC5.1.2});
          \item Il docente conferma la creazione della domanda;

      \end{enumerate}
    \item[Postcondizione:] È stata creata una domanda di tipo testo con parole omesse.
  \end{description}
\hypertarget{UC5.10}{}
\subsection{Caso d'uso UC5.10: Inserimento domanda con l'associazione di parole}
	\begin{figure}[H]
		\centering
		\begin{resizedtikzpicture}{\textwidth}
		\umlactor[x=0, y=-1]{Docente}
		\begin{umlsystem}[x=0, fill=lightgray!20]{Quizzipedia}
			\umlusecase[x=5, y=-1.25, fill=white, width=4cm, name=152]{\textbf{UC5.10:} Inserimento domanda con l'associazione di parole}
			\umlassoc{Docente}{152}
		\end{umlsystem}
		\end{resizedtikzpicture}
		\caption{\textbf{UC5.10}: Inserimento domanda con l'associazione di parole}
		\label{UC5.10}
	\end{figure}
\begin{description}
\item[Attori:] Docente;
\item[Scopo e descrizione:] Il docente inserisce una domanda con associazione di parole
      \item[Precondizione:] Il docente è autenticato presso il sistema;

        \item[Flusso principale degli eventi:] \ 
 \begin{enumerate}
          \item Il docente seleziona gli argomenti della domanda (\hyperlink{UC5.1.1}{UC5.1.1});
          \item Il docente compone la domanda in QML specificando le parole che possono essere combinate e le giuste combinazioni (\hyperlink{UC5.1.2}{UC5.1.2});
          \item Il docente conferma la creazione della domanda;

      \end{enumerate}
    \item[Postcondizione:] Stata inserita una domanda con associazione di parole.
  \end{description}
\hypertarget{UC5.11}{}
\subsection{Caso d'uso UC5.11: Inserimento domanda a risposta aperta}
	\begin{figure}[H]
		\centering
		\begin{resizedtikzpicture}{\textwidth}
		\umlactor[x=0, y=-1]{Docente}
		\begin{umlsystem}[x=0, fill=lightgray!20]{Quizzipedia}
			\umlusecase[x=5, y=-1.25, fill=white, width=4cm, name=153]{\textbf{UC5.11:} Inserimento domanda a risposta aperta}
			\umlassoc{Docente}{153}
		\end{umlsystem}
		\end{resizedtikzpicture}
		\caption{\textbf{UC5.11}: Inserimento domanda a risposta aperta}
		\label{UC5.11}
	\end{figure}
\begin{description}
\item[Attori:] Docente;
\item[Scopo e descrizione:] Il docente inserisce una domanda a risposta aperta
      \item[Precondizione:] Il docente è autenticato presso il sistema;

        \item[Flusso principale degli eventi:] \ 
 \begin{enumerate}
          \item Il docente seleziona gli argomenti della domanda (\hyperlink{UC5.1.1}{UC5.1.1});
          \item Il docente compone la domanda in QML specificando il testo della domanda (ma non la risposta) (\hyperlink{UC5.1.2}{UC5.1.2});
          \item Il docente conferma la creazione della domanda;

      \end{enumerate}
    \item[Postcondizione:] È stata inserita una domanda a risposta aperta.
  \end{description}
\hypertarget{UC5.12}{}
\subsection{Caso d'uso UC5.12: Visualizza domanda}
	\begin{figure}[H]
		\centering
		\begin{resizedtikzpicture}{\textwidth}
		\umlactor[x=0, y=-1]{Docente}
		\begin{umlsystem}[x=0, fill=lightgray!20]{Quizzipedia}
			\umlusecase[x=5, y=-1, fill=white, width=4cm, name=181]{\textbf{UC5.12:} Visualizza domanda}
			\umlassoc{Docente}{181}
		\end{umlsystem}
		\end{resizedtikzpicture}
		\caption{\textbf{UC5.12}: Visualizza domanda}
		\label{UC5.12}
	\end{figure}
\begin{description}
\item[Attori:] Docente;
\item[Scopo e descrizione:] Il docente visualizza una domanda
      \item[Precondizione:] Il docente è autenticato nel sistema
;

        \item[Flusso principale degli eventi:] \ 
 \begin{enumerate}
          \item Ricerca della domanda da visualizzare (\hyperlink{UC22}{UC22});
          \item Selezione della domanda da visualizzare;

      \end{enumerate}
    \item[Postcondizione:] Il docente visualizza una domanda.
  \end{description}
\hypertarget{UC5.13}{}
\subsection{Caso d'uso UC5.13: Errore eliminazione domanda utilizzata}
	\begin{figure}[H]
		\centering
		\begin{resizedtikzpicture}{\textwidth}
		\umlactor[x=0, y=-1]{Docente}
		\begin{umlsystem}[x=0, fill=lightgray!20]{Quizzipedia}
			\umlusecase[x=5, y=-1.25, fill=white, width=4cm, name=206]{\textbf{UC5.13:} Errore eliminazione domanda utilizzata}
			\umlassoc{Docente}{206}
		\end{umlsystem}
		\end{resizedtikzpicture}
		\caption{\textbf{UC5.13}: Errore eliminazione domanda utilizzata}
		\label{UC5.13}
	\end{figure}
\begin{description}
\item[Attori:] Docente;
\item[Scopo e descrizione:] Viene mostrato un errore nel caso in cui il docente tenti di eliminare una domanda utilizzata da almeno un questionario
      \item[Precondizione:] Il docente tenta di eliminare una domanda utilizzata da almeno un questionario;

        \item[Flusso principale degli eventi:] \ 
 \begin{enumerate}
          \item Viene mostrato un errore nel caso in cui il docente tenti di eliminare una domanda utilizzata da almeno un questionario;

      \end{enumerate}
    \item[Postcondizione:] Non viene eliminata la domanda e viene mostrato un messaggio di errore.
  \end{description}
\hypertarget{UC6}{}
\subsection{Caso d'uso UC6: Gestione questionari}
	\begin{figure}[H]
		\centering
		\begin{resizedtikzpicture}{\textwidth}
		\umlactor[x=0, y=-1]{Docente}
		\begin{umlsystem}[x=0, fill=lightgray!20]{Quizzipedia}
			\umlusecase[x=5, y=-5, fill=white, width=4cm, name=48]{\textbf{UC6.3:} Elimina questionario}
			\umlassoc{Docente}{48}
			\umlusecase[x=5, y=-1, fill=white, width=4cm, name=47]{\textbf{UC6.2:} Modifica questionario}
			\umlassoc{Docente}{47}
			\umlusecase[x=5, y=3, fill=white, width=4cm, name=42]{\textbf{UC6.1:} Inserisci questionario}
			\umlassoc{Docente}{42}
		\end{umlsystem}
		\end{resizedtikzpicture}
		\caption{\textbf{UC6}: Gestione questionari}
		\label{UC6}
	\end{figure}
\begin{description}
\item[Attori:] Docente;
\item[Scopo e descrizione:] Il docente gestisce i propri questionari
      \item[Precondizione:] Il docente è autenticato nel sistema;

        \item[Flusso principale degli eventi:] \ 
 \begin{enumerate}
          \item Il docente può creare un nuovo questionario (\hyperlink{UC6.1}{UC6.1});
          \item Il docente può modificare un questionario (\hyperlink{UC6.2}{UC6.2});
          \item Il docente può eliminare un questionario (\hyperlink{UC6.3}{UC6.3});

      \end{enumerate}
    \item[Postcondizione:] Il sistema ha ottenuto le informazioni sulle operazioni che il docente desidera eseguire su un questionario.
  \end{description}
\hypertarget{UC6.1}{}
\subsection{Caso d'uso UC6.1: Inserisci questionario}
	\begin{figure}[H]
		\centering
		\begin{resizedtikzpicture}{\textwidth}
		\umlactor[x=0, y=-1]{Docente}
		\begin{umlsystem}[x=0, fill=lightgray!20]{Quizzipedia}
			\umlusecase[x=5, y=-5.75, fill=white, width=4cm, name=118]{\textbf{UC6.1.3:} Seleziona argomenti del nuovo questionario}
			\umlassoc{Docente}{118}
			\umlusecase[x=5, y=-1.25, fill=white, width=4cm, name=93]{\textbf{UC6.1.2:} Elimina domanda da un nuovo questionario}
			\umlassoc{Docente}{93}
			\umlusecase[x=5, y=3.25, fill=white, width=4cm, name=91]{\textbf{UC6.1.1:} Aggiungi domanda in un nuovo questionario }
			\umlassoc{Docente}{91}
		\end{umlsystem}
		\end{resizedtikzpicture}
		\caption{\textbf{UC6.1}: Inserisci questionario}
		\label{UC6.1}
	\end{figure}
\begin{description}
\item[Attori:] Docente;
\item[Scopo e descrizione:] Il docente crea un questionario
      \item[Precondizione:] Il docente è autenticato nel sistema;

        \item[Flusso principale degli eventi:] \ 
 \begin{enumerate}
          \item Il docente può aggiungere domande al questionario (\hyperlink{UC6.1.1}{UC6.1.1});
          \item Il docente può togliere domande precedentemente aggiunte al questionario (\hyperlink{UC6.1.2}{UC6.1.2});
          \item Il docente seleziona gli argomenti del questionario (\hyperlink{UC6.1.3}{UC6.1.3});
          \item Il docente aggiunge il titolo del questionario;
          \item Il docente conferma la creazione del questionario;

      \end{enumerate}
    \item[Estensioni:]
      \begin{enumerate}
          \item Se il questionario non ha domande viene visualizzato un messaggio d'errore (\hyperlink{UC6.4}{UC6.4});

      \end{enumerate}
    \item[Postcondizione:] È stato creato un nuovo questionario.
  \end{description}
\hypertarget{UC6.1.1}{}
\subsection{Caso d'uso UC6.1.1: Aggiungi domanda in un nuovo questionario }
	\begin{figure}[H]
		\centering
		\begin{resizedtikzpicture}{\textwidth}
		\umlactor[x=0, y=-1]{Docente}
		\begin{umlsystem}[x=0, fill=lightgray!20]{Quizzipedia}
			\umlusecase[x=5, y=-1.25, fill=white, width=4cm, name=91]{\textbf{UC6.1.1:} Aggiungi domanda in un nuovo questionario }
			\umlassoc{Docente}{91}
		\end{umlsystem}
		\end{resizedtikzpicture}
		\caption{\textbf{UC6.1.1}: Aggiungi domanda in un nuovo questionario }
		\label{UC6.1.1}
	\end{figure}
\begin{description}
\item[Attori:] Docente;
\item[Scopo e descrizione:] Il docente ricerca e seleziona una domanda da inserire in un nuovo questionario
      \item[Precondizione:] Il docente sta creando un nuovo questionario;

        \item[Flusso principale degli eventi:] \ 
 \begin{enumerate}
          \item Viene ricercata un domanda (\hyperlink{UC22}{UC22});
          \item Selezione della domanda;
          \item Conferma inserimento domanda;

      \end{enumerate}
    \item[Postcondizione:] È stata aggiunta una domanda al nuovo questionario.
  \end{description}
\hypertarget{UC6.1.2}{}
\subsection{Caso d'uso UC6.1.2: Elimina domanda da un nuovo questionario}
	\begin{figure}[H]
		\centering
		\begin{resizedtikzpicture}{\textwidth}
		\umlactor[x=0, y=-1]{Docente}
		\begin{umlsystem}[x=0, fill=lightgray!20]{Quizzipedia}
			\umlusecase[x=5, y=-1.25, fill=white, width=4cm, name=93]{\textbf{UC6.1.2:} Elimina domanda da un nuovo questionario}
			\umlassoc{Docente}{93}
		\end{umlsystem}
		\end{resizedtikzpicture}
		\caption{\textbf{UC6.1.2}: Elimina domanda da un nuovo questionario}
		\label{UC6.1.2}
	\end{figure}
\begin{description}
\item[Attori:] Docente;
\item[Scopo e descrizione:] Il docente elimina una domanda da un nuovo questionario
      \item[Precondizione:] Il docente sta creando un nuovo questionario e ha selezionato una domanda da eliminare;

        \item[Flusso principale degli eventi:] \ 
 \begin{enumerate}
          \item Viene eliminata la domanda;

      \end{enumerate}
    \item[Postcondizione:] È stato eliminata la domanda dal nuovo questionario.
  \end{description}
\hypertarget{UC6.1.3}{}
\subsection{Caso d'uso UC6.1.3: Seleziona argomenti del nuovo questionario}
	\begin{figure}[H]
		\centering
		\begin{resizedtikzpicture}{\textwidth}
		\umlactor[x=0, y=-1]{Docente}
		\begin{umlsystem}[x=0, fill=lightgray!20]{Quizzipedia}
			\umlusecase[x=5, y=-1.25, fill=white, width=4cm, name=118]{\textbf{UC6.1.3:} Seleziona argomenti del nuovo questionario}
			\umlassoc{Docente}{118}
		\end{umlsystem}
		\end{resizedtikzpicture}
		\caption{\textbf{UC6.1.3}: Seleziona argomenti del nuovo questionario}
		\label{UC6.1.3}
	\end{figure}
\begin{description}
\item[Attori:] Docente;
\item[Scopo e descrizione:] Il docente seleziona gli argomenti relativi al nuovo questionario 
      \item[Precondizione:] Il docente sta creando un nuovo questionario;

        \item[Flusso principale degli eventi:] \ 
 \begin{enumerate}
          \item Il docente aggiunge un argomento al questionario;
          \item Il docente toglie un argomento al questionario;

      \end{enumerate}
    \item[Postcondizione:] Il docente ha selezionato gli argomenti del nuovo questionario.
  \end{description}
\hypertarget{UC6.2}{}
\subsection{Caso d'uso UC6.2: Modifica questionario}
	\begin{figure}[H]
		\centering
		\begin{resizedtikzpicture}{\textwidth}
		\umlactor[x=0, y=-1]{Docente}
		\begin{umlsystem}[x=0, fill=lightgray!20]{Quizzipedia}
			\umlusecase[x=5, y=-5.75, fill=white, width=4cm, name=168]{\textbf{UC6.2.3:} Selezione argomenti modifica questionario}
			\umlassoc{Docente}{168}
			\umlusecase[x=5, y=-1.25, fill=white, width=4cm, name=167]{\textbf{UC6.2.2:} Elimina domanda da un questionario da  modificare}
			\umlassoc{Docente}{167}
			\umlusecase[x=5, y=3.25, fill=white, width=4cm, name=166]{\textbf{UC6.2.1:} Aggiungi domanda in un questionario}
			\umlassoc{Docente}{166}
		\end{umlsystem}
		\end{resizedtikzpicture}
		\caption{\textbf{UC6.2}: Modifica questionario}
		\label{UC6.2}
	\end{figure}
\begin{description}
\item[Attori:] Docente;
\item[Scopo e descrizione:] Il docente modifica il questionario che ha selezionato potendo aggiungere e rimuovere domande e modificare gli argomenti del questionario
      \item[Precondizione:] Il docente è autenticato nel sistema;

        \item[Flusso principale degli eventi:] \ 
 \begin{enumerate}
          \item Ricerca del questionario da modificare (\hyperlink{UC16}{UC16});
          \item Selezione del questionario da modificare;
          \item Il docente può aggiungere domande al questionario (\hyperlink{UC6.2.1}{UC6.2.1});
          \item Il docente può togliere domande precedentemente aggiunte al questionario (\hyperlink{UC6.2.2}{UC6.2.2});
          \item Il docente può modificare gli argomenti del questionario (\hyperlink{UC6.2.3}{UC6.2.3});
          \item Il docente conferma la modifica del questionario;

      \end{enumerate}
    \item[Estensioni:]
      \begin{enumerate}
          \item Se il questionario non ha domande viene visualizzato un messaggio d'errore (\hyperlink{UC6.4}{UC6.4});

      \end{enumerate}
    \item[Postcondizione:] È stato modificato il questionario.
  \end{description}
\hypertarget{UC6.2.1}{}
\subsection{Caso d'uso UC6.2.1: Aggiungi domanda in un questionario}
	\begin{figure}[H]
		\centering
		\begin{resizedtikzpicture}{\textwidth}
		\umlactor[x=0, y=-1]{Docente}
		\begin{umlsystem}[x=0, fill=lightgray!20]{Quizzipedia}
			\umlusecase[x=5, y=-1.25, fill=white, width=4cm, name=166]{\textbf{UC6.2.1:} Aggiungi domanda in un questionario}
			\umlassoc{Docente}{166}
		\end{umlsystem}
		\end{resizedtikzpicture}
		\caption{\textbf{UC6.2.1}: Aggiungi domanda in un questionario}
		\label{UC6.2.1}
	\end{figure}
\begin{description}
\item[Attori:] Docente;
\item[Scopo e descrizione:] Il docente inserisce una domanda da un questionario da modificare
      \item[Precondizione:] Il docente ha selezionato un questionario da modificare e una domanda da inserire;

        \item[Flusso principale degli eventi:] \ 
 \begin{enumerate}
          \item Viene ricercata una domanda (\hyperlink{UC22}{UC22});
          \item Selezione della domanda	;
          \item Conferma inserimento domanda;

      \end{enumerate}
    \item[Postcondizione:] È stata aggiunta una domanda al questionario.
  \end{description}
\hypertarget{UC6.2.2}{}
\subsection{Caso d'uso UC6.2.2: Elimina domanda da un questionario da  modificare}
	\begin{figure}[H]
		\centering
		\begin{resizedtikzpicture}{\textwidth}
		\umlactor[x=0, y=-1]{Docente}
		\begin{umlsystem}[x=0, fill=lightgray!20]{Quizzipedia}
			\umlusecase[x=5, y=-1.25, fill=white, width=4cm, name=167]{\textbf{UC6.2.2:} Elimina domanda da un questionario da  modificare}
			\umlassoc{Docente}{167}
		\end{umlsystem}
		\end{resizedtikzpicture}
		\caption{\textbf{UC6.2.2}: Elimina domanda da un questionario da  modificare}
		\label{UC6.2.2}
	\end{figure}
\begin{description}
\item[Attori:] Docente;
\item[Scopo e descrizione:] Il docente elimina una domanda da un questionario da modificare
      \item[Precondizione:] Il docente ha selezionato un questionario da modificare e una domanda da eliminare;

        \item[Flusso principale degli eventi:] \ 
 \begin{enumerate}
          \item Viene eliminata la domanda;

      \end{enumerate}
    \item[Postcondizione:] È stato eliminata la domanda questionario da modificare.
  \end{description}
\hypertarget{UC6.2.3}{}
\subsection{Caso d'uso UC6.2.3: Selezione argomenti modifica questionario}
	\begin{figure}[H]
		\centering
		\begin{resizedtikzpicture}{\textwidth}
		\umlactor[x=0, y=-1]{Docente}
		\begin{umlsystem}[x=0, fill=lightgray!20]{Quizzipedia}
			\umlusecase[x=5, y=-1.25, fill=white, width=4cm, name=168]{\textbf{UC6.2.3:} Selezione argomenti modifica questionario}
			\umlassoc{Docente}{168}
		\end{umlsystem}
		\end{resizedtikzpicture}
		\caption{\textbf{UC6.2.3}: Selezione argomenti modifica questionario}
		\label{UC6.2.3}
	\end{figure}
\begin{description}
\item[Attori:] Docente;
\item[Scopo e descrizione:] Il docente seleziona gli argomenti corrispondenti al questionario selezionato
      \item[Precondizione:] Il docente sta modificando un questionario;

        \item[Flusso principale degli eventi:] \ 
 \begin{enumerate}
          \item Il docente aggiunge un argomento al questionario;
          \item Il docente toglie un argomento al questionario;

      \end{enumerate}
    \item[Postcondizione:] Il docente ha definito gli argomenti per classificare il questionario.
  \end{description}
\hypertarget{UC6.3}{}
\subsection{Caso d'uso UC6.3: Elimina questionario}
	\begin{figure}[H]
		\centering
		\begin{resizedtikzpicture}{\textwidth}
		\umlactor[x=0, y=-1]{Docente}
		\begin{umlsystem}[x=0, fill=lightgray!20]{Quizzipedia}
			\umlusecase[x=5, y=-1, fill=white, width=4cm, name=48]{\textbf{UC6.3:} Elimina questionario}
			\umlassoc{Docente}{48}
		\end{umlsystem}
		\end{resizedtikzpicture}
		\caption{\textbf{UC6.3}: Elimina questionario}
		\label{UC6.3}
	\end{figure}
\begin{description}
\item[Attori:] Docente;
\item[Scopo e descrizione:] Il docente rimuove un questionario dal sistema
      \item[Precondizione:] Il docente è autenticato nel sistema;

        \item[Flusso principale degli eventi:] \ 
 \begin{enumerate}
          \item Ricerca il questionario da eliminare (\hyperlink{UC16}{UC16});
          \item Selezione del questionario da eliminare;
          \item Il docente conferma l'eliminazione;

      \end{enumerate}
    \item[Postcondizione:] È stato eliminato il questionario.
  \end{description}
\hypertarget{UC6.4}{}
\subsection{Caso d'uso UC6.4: Errore questionario vuoto}
	\begin{figure}[H]
		\centering
		\begin{resizedtikzpicture}{\textwidth}
		\umlactor[x=0, y=-1]{Docente}
		\begin{umlsystem}[x=0, fill=lightgray!20]{Quizzipedia}
			\umlusecase[x=5, y=-1, fill=white, width=4cm, name=144]{\textbf{UC6.4:} Errore questionario vuoto}
			\umlassoc{Docente}{144}
		\end{umlsystem}
		\end{resizedtikzpicture}
		\caption{\textbf{UC6.4}: Errore questionario vuoto}
		\label{UC6.4}
	\end{figure}
\begin{description}
\item[Attori:] Docente;
\item[Scopo e descrizione:] Il sistema avvisa il docente che nel questionario deve essere presente almeno una domanda
      \item[Precondizione:] Il questionario selezionato non ha domande;

        \item[Flusso principale degli eventi:] \ 
 \begin{enumerate}
          \item Viene visualizzato un messaggio di errore;

      \end{enumerate}
    \item[Postcondizione:] Il questionario non viene inserito nel sistema.
  \end{description}
\hypertarget{UC6.5}{}
\subsection{Caso d'uso UC6.5: Visualizza questionario}
	\begin{figure}[H]
		\centering
		\begin{resizedtikzpicture}{\textwidth}
		\umlactor[x=0, y=-1]{Docente}
		\begin{umlsystem}[x=0, fill=lightgray!20]{Quizzipedia}
			\umlusecase[x=5, y=-1, fill=white, width=4cm, name=182]{\textbf{UC6.5:} Visualizza questionario}
			\umlassoc{Docente}{182}
		\end{umlsystem}
		\end{resizedtikzpicture}
		\caption{\textbf{UC6.5}: Visualizza questionario}
		\label{UC6.5}
	\end{figure}
\begin{description}
\item[Attori:] Docente;
\item[Scopo e descrizione:] Il docente visualizza un questionario
      \item[Precondizione:] Il docente è autenticato nel sistema
;

        \item[Flusso principale degli eventi:] \ 
 \begin{enumerate}
          \item Ricerca del questionario da visualizzare	 (\hyperlink{UC16}{UC16});
          \item Selezione del questionario da visualizzare	;

      \end{enumerate}
    \item[Postcondizione:] Il docente ha visualizzato un questionario.
  \end{description}
\hypertarget{UC7}{}
\subsection{Caso d'uso UC7: Gestione classi}
	\begin{figure}[H]
		\centering
		\begin{resizedtikzpicture}{\textwidth}
		\umlactor[x=0, y=-1]{Docente}
		\begin{umlsystem}[x=0, fill=lightgray!20]{Quizzipedia}
			\umlusecase[x=5, y=-5, fill=white, width=4cm, name=49]{\textbf{UC7.3:} Elimina classe}
			\umlassoc{Docente}{49}
			\umlusecase[x=5, y=-1, fill=white, width=4cm, name=45]{\textbf{UC7.2:} Modifica classe}
			\umlassoc{Docente}{45}
			\umlusecase[x=5, y=3, fill=white, width=4cm, name=44]{\textbf{UC7.1:} Inserisci classe}
			\umlassoc{Docente}{44}
		\end{umlsystem}
		\end{resizedtikzpicture}
		\caption{\textbf{UC7}: Gestione classi}
		\label{UC7}
	\end{figure}
\begin{description}
\item[Attori:] Docente;
\item[Scopo e descrizione:] Il docente gestisce le proprie classi
      \item[Precondizione:] Il docente è autenticato nel sistema;

        \item[Flusso principale degli eventi:] \ 
 \begin{enumerate}
          \item Il docente può creare una nuova classe (\hyperlink{UC7.1}{UC7.1});
          \item Il docente può modificare una classe (\hyperlink{UC7.2}{UC7.2});
          \item Il docente può eliminare una classe (\hyperlink{UC7.3}{UC7.3});

      \end{enumerate}
    \item[Postcondizione:] Il sistema ha ottenuto le informazioni sulle operazioni che il docente desidera eseguire sulla classe.
  \end{description}
\hypertarget{UC7.1}{}
\subsection{Caso d'uso UC7.1: Inserisci classe}
	\begin{figure}[H]
		\centering
		\begin{resizedtikzpicture}{\textwidth}
		\umlactor[x=0, y=-1]{Docente}
		\begin{umlsystem}[x=0, fill=lightgray!20]{Quizzipedia}
			\umlusecase[x=5, y=-5, fill=white, width=4cm, name=95]{\textbf{UC7.1.3:} Inserisci password classe}
			\umlassoc{Docente}{95}
			\umlusecase[x=5, y=-1, fill=white, width=4cm, name=94]{\textbf{UC7.1.2:} Inserisci argomenti classe}
			\umlassoc{Docente}{94}
			\umlusecase[x=5, y=3, fill=white, width=4cm, name=90]{\textbf{UC7.1.1:} Inserisci nome classe}
			\umlassoc{Docente}{90}
		\end{umlsystem}
		\end{resizedtikzpicture}
		\caption{\textbf{UC7.1}: Inserisci classe}
		\label{UC7.1}
	\end{figure}
\begin{description}
\item[Attori:] Docente;
\item[Scopo e descrizione:] Il docente crea una nuova classe alla quale gli studenti potranno iscriversi 
      \item[Precondizione:] Il docente è autenticato nel sistema;

        \item[Flusso principale degli eventi:] \ 
 \begin{enumerate}
          \item Viene inserito il nome della classe (\hyperlink{UC7.1.1}{UC7.1.1});
          \item Vengono inseriti gli argomenti della classe (\hyperlink{UC7.1.2}{UC7.1.2});
          \item Viene inserita la password della classe (\hyperlink{UC7.1.3}{UC7.1.3});
          \item Viene confermato l'inserimento della classe;

      \end{enumerate}
    \item[Postcondizione:] È stato creata una nuova classe.
  \end{description}
\hypertarget{UC7.1.1}{}
\subsection{Caso d'uso UC7.1.1: Inserisci nome classe}
	\begin{figure}[H]
		\centering
		\begin{resizedtikzpicture}{\textwidth}
		\umlactor[x=0, y=-1]{Docente}
		\begin{umlsystem}[x=0, fill=lightgray!20]{Quizzipedia}
			\umlusecase[x=5, y=-1, fill=white, width=4cm, name=90]{\textbf{UC7.1.1:} Inserisci nome classe}
			\umlassoc{Docente}{90}
			\umlusecase[x=15, y=-1, fill=white, width=4cm, name=92]{\textbf{UC7.4:} Errore nome classe già presente}
			\umlextend{92}{90}
		\end{umlsystem}
		\end{resizedtikzpicture}
		\caption{\textbf{UC7.1.1}: Inserisci nome classe}
		\label{UC7.1.1}
	\end{figure}
\begin{description}
\item[Attori:] Docente;
\item[Scopo e descrizione:] Il docente inserisce il nome scelto per la classe
      \item[Precondizione:] Il docente ha selezionato la classe;

        \item[Flusso principale degli eventi:] \ 
 \begin{enumerate}
          \item Il docente inserisce un nome per la classe non ancora presente nel sistema;

      \end{enumerate}
    \item[Estensioni:]
      \begin{enumerate}
          \item Se il nome della classe è già presente nel sistema viene visualizzato un messaggio di errore (\hyperlink{UC7.4}{UC7.4});

      \end{enumerate}
    \item[Postcondizione:] Il docente ha inserito il nome della classe.
  \end{description}
\hypertarget{UC7.1.2}{}
\subsection{Caso d'uso UC7.1.2: Inserisci argomenti classe}
	\begin{figure}[H]
		\centering
		\begin{resizedtikzpicture}{\textwidth}
		\umlactor[x=0, y=-1]{Docente}
		\begin{umlsystem}[x=0, fill=lightgray!20]{Quizzipedia}
			\umlusecase[x=5, y=-1, fill=white, width=4cm, name=94]{\textbf{UC7.1.2:} Inserisci argomenti classe}
			\umlassoc{Docente}{94}
		\end{umlsystem}
		\end{resizedtikzpicture}
		\caption{\textbf{UC7.1.2}: Inserisci argomenti classe}
		\label{UC7.1.2}
	\end{figure}
\begin{description}
\item[Attori:] Docente;
\item[Scopo e descrizione:] Il docente specifica di quali argomenti tratta la classe selezionata
      \item[Precondizione:] Il docente ha selezionato una classe;

        \item[Flusso principale degli eventi:] \ 
 \begin{enumerate}
          \item Il docente specifica di quali argomenti tratta la classe selezionata;

      \end{enumerate}
    \item[Postcondizione:] Il docente ha inserito gli argomenti della classe.
  \end{description}
\hypertarget{UC7.1.3}{}
\subsection{Caso d'uso UC7.1.3: Inserisci password classe}
	\begin{figure}[H]
		\centering
		\begin{resizedtikzpicture}{\textwidth}
		\umlactor[x=0, y=-1]{Docente}
		\begin{umlsystem}[x=0, fill=lightgray!20]{Quizzipedia}
			\umlusecase[x=5, y=-1, fill=white, width=4cm, name=95]{\textbf{UC7.1.3:} Inserisci password classe}
			\umlassoc{Docente}{95}
		\end{umlsystem}
		\end{resizedtikzpicture}
		\caption{\textbf{UC7.1.3}: Inserisci password classe}
		\label{UC7.1.3}
	\end{figure}
\begin{description}
\item[Attori:] Docente;
\item[Scopo e descrizione:] Il docente specifica la password che verrà fornita poi agli studenti in classe o per altre vie per poter accedere alla classe
      \item[Precondizione:] Il docente ha selezionato una classe;

        \item[Flusso principale degli eventi:] \ 
 \begin{enumerate}
          \item Il docente inserisce una password che servirà per accedere alla classe che verrà creata;

      \end{enumerate}
    \item[Postcondizione:] Il docente ha inserito la password per accedere alla classe.
  \end{description}
\hypertarget{UC7.2}{}
\subsection{Caso d'uso UC7.2: Modifica classe}
	\begin{figure}[H]
		\centering
		\begin{resizedtikzpicture}{\textwidth}
		\umlactor[x=0, y=-1]{Docente}
		\begin{umlsystem}[x=0, fill=lightgray!20]{Quizzipedia}
			\umlusecase[x=5, y=-1, fill=white, width=4cm, name=45]{\textbf{UC7.2:} Modifica classe}
			\umlassoc{Docente}{45}
		\end{umlsystem}
		\end{resizedtikzpicture}
		\caption{\textbf{UC7.2}: Modifica classe}
		\label{UC7.2}
	\end{figure}
\begin{description}
\item[Attori:] Docente;
\item[Scopo e descrizione:] Il docente modifica degli attributi o dei dati legati alla classe scelta
      \item[Precondizione:] Il docente è autenticato nel sistema, è presente nel sistema la classe da modificare;

        \item[Flusso principale degli eventi:] \ 
 \begin{enumerate}
          \item Viene selezionata una delle proprie classi da modificare;
          \item Può venire modificato il nome (\hyperlink{UC7.1.1}{UC7.1.1});
          \item Possono venire modificati gli argomenti (\hyperlink{UC7.1.2}{UC7.1.2});
          \item Può venire modificata la password (\hyperlink{UC7.1.3}{UC7.1.3});
          \item Viene confermata la modifica;

      \end{enumerate}
    \item[Postcondizione:] È stata modificata la classe.
  \end{description}
\hypertarget{UC7.3}{}
\subsection{Caso d'uso UC7.3: Elimina classe}
	\begin{figure}[H]
		\centering
		\begin{resizedtikzpicture}{\textwidth}
		\umlactor[x=0, y=-1]{Docente}
		\begin{umlsystem}[x=0, fill=lightgray!20]{Quizzipedia}
			\umlusecase[x=5, y=-1, fill=white, width=4cm, name=49]{\textbf{UC7.3:} Elimina classe}
			\umlassoc{Docente}{49}
		\end{umlsystem}
		\end{resizedtikzpicture}
		\caption{\textbf{UC7.3}: Elimina classe}
		\label{UC7.3}
	\end{figure}
\begin{description}
\item[Attori:] Docente;
\item[Scopo e descrizione:] Il docente rimuove dal sistema una classe da lui creata 
      \item[Precondizione:] Il docente è autenticato nel sistema, è presente nel sistema la classe da eliminare;

        \item[Flusso principale degli eventi:] \ 
 \begin{enumerate}
          \item Viene sezionata una delle proprie classi;
          \item Viene sezionata la funzionalità elimina;
          \item Viene confermata l'eliminazione;

      \end{enumerate}
    \item[Postcondizione:] È stata eliminata la classe.
  \end{description}
\hypertarget{UC7.4}{}
\subsection{Caso d'uso UC7.4: Errore nome classe già presente}
	\begin{figure}[H]
		\centering
		\begin{resizedtikzpicture}{\textwidth}
		\umlactor[x=0, y=-1]{Docente}
		\begin{umlsystem}[x=0, fill=lightgray!20]{Quizzipedia}
			\umlusecase[x=5, y=-1, fill=white, width=4cm, name=92]{\textbf{UC7.4:} Errore nome classe già presente}
			\umlassoc{Docente}{92}
		\end{umlsystem}
		\end{resizedtikzpicture}
		\caption{\textbf{UC7.4}: Errore nome classe già presente}
		\label{UC7.4}
	\end{figure}
\begin{description}
\item[Attori:] Docente;
\item[Scopo e descrizione:] Viene visualizzato un errore nel caso il nome della classe che si sta tentando di inserire esiste già
      \item[Precondizione:] Il nome della classe è già presente nel sistema;

        \item[Flusso principale degli eventi:] \ 
 \begin{enumerate}
          \item Viene visualizzato un messaggio di errore;

      \end{enumerate}
    \item[Postcondizione:] La classe non viene inserita nel sistema.
  \end{description}
\hypertarget{UC8}{}
\subsection{Caso d'uso UC8: Gestione argomenti}
	\begin{figure}[H]
		\centering
		\begin{resizedtikzpicture}{\textwidth}
		\umlactor[x=0, y=-1]{Docente}
		\begin{umlsystem}[x=0, fill=lightgray!20]{Quizzipedia}
			\umlusecase[x=5, y=-7, fill=white, width=4cm, name=146]{\textbf{UC8.5:} Eliminazione argomento}
			\umlassoc{Docente}{146}
			\umlusecase[x=5, y=-3, fill=white, width=4cm, name=141]{\textbf{UC8.4:} Modifica argomento}
			\umlassoc{Docente}{141}
			\umlusecase[x=5, y=1, fill=white, width=4cm, name=137]{\textbf{UC8.2:} Crea argomento}
			\umlassoc{Docente}{137}
			\umlusecase[x=5, y=5, fill=white, width=4cm, name=135]{\textbf{UC8.1:} Esplorazione argomenti}
			\umlassoc{Docente}{135}
		\end{umlsystem}
		\end{resizedtikzpicture}
		\caption{\textbf{UC8}: Gestione argomenti}
		\label{UC8}
	\end{figure}
\begin{description}
\item[Attori:] Docente;
\item[Scopo e descrizione:] Il docente gestisce gli argomenti
      \item[Precondizione:] Il docente è autenticato nel sistema;

        \item[Flusso principale degli eventi:] \ 
 \begin{enumerate}
          \item Il docente può esplorare gli argomenti (\hyperlink{UC8.1}{UC8.1});
          \item Il docente può creare un argomento (\hyperlink{UC8.2}{UC8.2});
          \item Il docente può modificare un argomento (\hyperlink{UC8.4}{UC8.4});
          \item Il docente può eliminare un argomento (\hyperlink{UC8.5}{UC8.5});

      \end{enumerate}
    \item[Postcondizione:] Il sistema ha ottenuto le informazioni sulle operazioni che il docente desidera eseguire su un argomento.
  \end{description}
\hypertarget{UC8.1}{}
\subsection{Caso d'uso UC8.1: Esplorazione argomenti}
	\begin{figure}[H]
		\centering
		\begin{resizedtikzpicture}{\textwidth}
		\umlactor[x=0, y=-1]{Docente}
		\begin{umlsystem}[x=0, fill=lightgray!20]{Quizzipedia}
			\umlusecase[x=5, y=-1, fill=white, width=4cm, name=135]{\textbf{UC8.1:} Esplorazione argomenti}
			\umlassoc{Docente}{135}
		\end{umlsystem}
		\end{resizedtikzpicture}
		\caption{\textbf{UC8.1}: Esplorazione argomenti}
		\label{UC8.1}
	\end{figure}
\begin{description}
\item[Attori:] Docente;
\item[Scopo e descrizione:] Il docente visualizza gli argomenti presenti nel sistema
      \item[Precondizione:] Il docente è autenticato nel sistema;

        \item[Flusso principale degli eventi:] \ 
 \begin{enumerate}
          \item Il docente visualizza l'albero degli argomenti;

      \end{enumerate}
    \item[Postcondizione:] Il docente ha visualizzato l'albero degli argomenti.
  \end{description}
\hypertarget{UC8.2}{}
\subsection{Caso d'uso UC8.2: Crea argomento}
	\begin{figure}[H]
		\centering
		\begin{resizedtikzpicture}{\textwidth}
		\umlactor[x=0, y=-1]{Docente}
		\begin{umlsystem}[x=0, fill=lightgray!20]{Quizzipedia}
			\umlusecase[x=5, y=-1, fill=white, width=4cm, name=137]{\textbf{UC8.2:} Crea argomento}
			\umlassoc{Docente}{137}
			\umlusecase[x=15, y=-1.25, fill=white, width=4cm, name=140]{\textbf{UC8.3:} Argomento già presente nel sistema}
			\umlextend{140}{137}
		\end{umlsystem}
		\end{resizedtikzpicture}
		\caption{\textbf{UC8.2}: Crea argomento}
		\label{UC8.2}
	\end{figure}
\begin{description}
\item[Attori:] Docente;
\item[Scopo e descrizione:] Il docente aggiunge un nuovo argomento nel sistema
      \item[Precondizione:] Il docente è autenticato nel sistema;

        \item[Flusso principale degli eventi:] \ 
 \begin{enumerate}
          \item Il docente inserisce il nome dell'argomento;
          \item Il docente seleziona l'argomento padre (\hyperlink{UC8.1}{UC8.1});
          \item Il docente conferma la creazione dell'argomento;

      \end{enumerate}
    \item[Estensioni:]
      \begin{enumerate}
          \item Se l'argomento già presente nel sistema viene visualizzato un messaggio di errore (\hyperlink{UC8.3}{UC8.3});

      \end{enumerate}
    \item[Postcondizione:] L'argomento inserito è ora presente tra gli argomenti del sistema.
  \end{description}
\hypertarget{UC8.3}{}
\subsection{Caso d'uso UC8.3: Argomento già presente nel sistema}
	\begin{figure}[H]
		\centering
		\begin{resizedtikzpicture}{\textwidth}
		\umlactor[x=0, y=-1]{Docente}
		\begin{umlsystem}[x=0, fill=lightgray!20]{Quizzipedia}
			\umlusecase[x=5, y=-1.25, fill=white, width=4cm, name=140]{\textbf{UC8.3:} Argomento già presente nel sistema}
			\umlassoc{Docente}{140}
		\end{umlsystem}
		\end{resizedtikzpicture}
		\caption{\textbf{UC8.3}: Argomento già presente nel sistema}
		\label{UC8.3}
	\end{figure}
\begin{description}
\item[Attori:] Docente;
\item[Scopo e descrizione:] Il sistema avvisa il docente che non è possibile inserire un argomento già presente
      \item[Precondizione:] L'argomento è già presente nel sistema;

        \item[Flusso principale degli eventi:] \ 
 \begin{enumerate}
          \item Viene visualizzato un messaggio d'errore;

      \end{enumerate}
    \item[Postcondizione:] L'argomento non viene inserito nel sistema.
  \end{description}
\hypertarget{UC8.4}{}
\subsection{Caso d'uso UC8.4: Modifica argomento}
	\begin{figure}[H]
		\centering
		\begin{resizedtikzpicture}{\textwidth}
		\umlactor[x=0, y=-1]{Docente}
		\begin{umlsystem}[x=0, fill=lightgray!20]{Quizzipedia}
			\umlusecase[x=5, y=-1, fill=white, width=4cm, name=141]{\textbf{UC8.4:} Modifica argomento}
			\umlassoc{Docente}{141}
			\umlusecase[x=15, y=-1.25, fill=white, width=4cm, name=140]{\textbf{UC8.3:} Argomento già presente nel sistema}
			\umlextend{140}{141}
		\end{umlsystem}
		\end{resizedtikzpicture}
		\caption{\textbf{UC8.4}: Modifica argomento}
		\label{UC8.4}
	\end{figure}
\begin{description}
\item[Attori:] Docente;
\item[Scopo e descrizione:] Il docente modifica il nome di un argomento
      \item[Precondizione:] Il docente è autenticato nel sistema;

        \item[Flusso principale degli eventi:] \ 
 \begin{enumerate}
          \item Il docente seleziona un argomento da modificare (\hyperlink{UC8.1}{UC8.1});
          \item Il docente può modificare il nome dell'argomento;
          \item Il docente può cambiare l'argomento padre (\hyperlink{UC8.1}{UC8.1});
          \item Il docente conferma la modifica dell'argomento;

      \end{enumerate}
    \item[Estensioni:]
      \begin{enumerate}
          \item Se l'argomento già presente nel sistema viene visualizzato un messaggio di errore (\hyperlink{UC8.3}{UC8.3});

      \end{enumerate}
    \item[Postcondizione:] È stato modificato un argomento.
  \end{description}
\hypertarget{UC8.5}{}
\subsection{Caso d'uso UC8.5: Eliminazione argomento}
	\begin{figure}[H]
		\centering
		\begin{resizedtikzpicture}{\textwidth}
		\umlactor[x=0, y=-1]{Docente}
		\begin{umlsystem}[x=0, fill=lightgray!20]{Quizzipedia}
			\umlusecase[x=5, y=-1, fill=white, width=4cm, name=146]{\textbf{UC8.5:} Eliminazione argomento}
			\umlassoc{Docente}{146}
			\umlusecase[x=15, y=-1.25, fill=white, width=4cm, name=147]{\textbf{UC8.6:} Errore l'argomento ha domande o questionari}
			\umlextend{147}{146}
		\end{umlsystem}
		\end{resizedtikzpicture}
		\caption{\textbf{UC8.5}: Eliminazione argomento}
		\label{UC8.5}
	\end{figure}
\begin{description}
\item[Attori:] Docente;
\item[Scopo e descrizione:] Il docente elimina un argomento già presente nel sistema
      \item[Precondizione:] Il docente è autenticato nel sistema;

        \item[Flusso principale degli eventi:] \ 
 \begin{enumerate}
          \item Il docente seleziona l'argomento da eliminare (\hyperlink{UC8.1}{UC8.1});
          \item Il docente conferma l'eliminazione;

      \end{enumerate}
    \item[Estensioni:]
      \begin{enumerate}
          \item Se l'argomento ha domande o questionari viene visualizzato un messaggio d'errore (\hyperlink{UC8.6}{UC8.6});

      \end{enumerate}
    \item[Postcondizione:] È stato eliminato un argomento.
  \end{description}
\hypertarget{UC8.6}{}
\subsection{Caso d'uso UC8.6: Errore l'argomento ha domande o questionari}
	\begin{figure}[H]
		\centering
		\begin{resizedtikzpicture}{\textwidth}
		\umlactor[x=0, y=-1]{Docente}
		\begin{umlsystem}[x=0, fill=lightgray!20]{Quizzipedia}
			\umlusecase[x=5, y=-1.25, fill=white, width=4cm, name=147]{\textbf{UC8.6:} Errore l'argomento ha domande o questionari}
			\umlassoc{Docente}{147}
		\end{umlsystem}
		\end{resizedtikzpicture}
		\caption{\textbf{UC8.6}: Errore l'argomento ha domande o questionari}
		\label{UC8.6}
	\end{figure}
\begin{description}
\item[Attori:] Docente;
\item[Scopo e descrizione:] Il sistema avvisa il docente che non può essere eliminato un argomento di cui esistano ancora domande o questionari
      \item[Precondizione:] L'argomento ha domande o questionari al suo interno;

        \item[Flusso principale degli eventi:] \ 
 \begin{enumerate}
          \item Viene visualizzato un messaggio d'errore;

      \end{enumerate}
    \item[Postcondizione:] L'argomento non viene eliminato.
  \end{description}
\hypertarget{UC9}{}
\subsection{Caso d'uso UC9: Visualizza statistiche}
	\begin{figure}[H]
		\centering
		\begin{resizedtikzpicture}{\textwidth}
		\umlactor[x=0, y=-1]{Docente}
		\begin{umlsystem}[x=0, fill=lightgray!20]{Quizzipedia}
			\umlusecase[x=5, y=-1, fill=white, width=4cm, name=10]{\textbf{UC9:} Visualizza statistiche}
			\umlassoc{Docente}{10}
			\umlusecase[x=15, y=-5.25, fill=white, width=4cm, name=13]{\textbf{UC12:} Visualizza statistiche classe}
			\umlinherit{13}{10}
			\umlusecase[x=15, y=-1.25, fill=white, width=4cm, name=12]{\textbf{UC11:} Visualizza statistiche questionario}
			\umlinherit{12}{10}
			\umlusecase[x=15, y=3.25, fill=white, width=4cm, name=11]{\textbf{UC10:} Visualizza statistiche domanda}
			\umlinherit{11}{10}
		\end{umlsystem}
		\end{resizedtikzpicture}
		\caption{\textbf{UC9}: Visualizza statistiche}
		\label{UC9}
	\end{figure}
\begin{description}
\item[Attori:] Docente;
\item[Scopo e descrizione:] Il docente visualizza le statistiche
      \item[Precondizione:] Il docente è autenticato presso il sistema;

        \item[Flusso principale degli eventi:] \ 
 \begin{enumerate}
          \item Il docente può visualizzare le statistiche;

      \end{enumerate}
    \item[Postcondizione:] Il docente ha visualizzato le statistiche a cui era interessato.
  \end{description}
\hypertarget{UC10}{}
\subsection{Caso d'uso UC10: Visualizza statistiche domanda}
	\begin{figure}[H]
		\centering
		\begin{resizedtikzpicture}{\textwidth}
		\umlactor[x=0, y=-1]{Docente}
		\begin{umlsystem}[x=0, fill=lightgray!20]{Quizzipedia}
			\umlusecase[x=5, y=-1, fill=white, width=4cm, name=11]{\textbf{UC10:} Visualizza statistiche domanda}
			\umlassoc{Docente}{11}
		\end{umlsystem}
		\end{resizedtikzpicture}
		\caption{\textbf{UC10}: Visualizza statistiche domanda}
		\label{UC10}
	\end{figure}
\begin{description}
\item[Attori:] Docente;
\item[Scopo e descrizione:] Il docente visualizza il numero di risposte totali, risposte corrette, risposte errate e la percentuale di risposte corrette sul totale
      \item[Precondizione:] Il docente è autenticato presso il sistema;

        \item[Flusso principale degli eventi:] \ 
 \begin{enumerate}
          \item Il docente ricerca una domanda (\hyperlink{UC22}{UC22});
          \item Il docente seleziona la domanda interessata;

      \end{enumerate}
    \item[Postcondizione:] Il docente ha visualizzato le statistiche relative alla domanda a cui era interessato.
  \end{description}
\hypertarget{UC11}{}
\subsection{Caso d'uso UC11: Visualizza statistiche questionario}
	\begin{figure}[H]
		\centering
		\begin{resizedtikzpicture}{\textwidth}
		\umlactor[x=0, y=-1]{Docente}
		\begin{umlsystem}[x=0, fill=lightgray!20]{Quizzipedia}
			\umlusecase[x=5, y=-1.25, fill=white, width=4cm, name=12]{\textbf{UC11:} Visualizza statistiche questionario}
			\umlassoc{Docente}{12}
		\end{umlsystem}
		\end{resizedtikzpicture}
		\caption{\textbf{UC11}: Visualizza statistiche questionario}
		\label{UC11}
	\end{figure}
\begin{description}
\item[Attori:] Docente;
\item[Scopo e descrizione:] Il docente visualizza per ogni punteggio la percentuale degli studenti che ottenuto tale punteggio e la media del punteggio
      \item[Precondizione:] Il docente è autenticato presso il sistema;

        \item[Flusso principale degli eventi:] \ 
 \begin{enumerate}
          \item Il docente ricerca un questionario (\hyperlink{UC16}{UC16});
          \item Il docente seleziona il questionario interessato;

      \end{enumerate}
    \item[Postcondizione:] Il docente ha visualizzato le statistiche relative al questionario a cui era interessato.
  \end{description}
\hypertarget{UC12}{}
\subsection{Caso d'uso UC12: Visualizza statistiche classe}
	\begin{figure}[H]
		\centering
		\begin{resizedtikzpicture}{\textwidth}
		\umlactor[x=0, y=-1]{Docente}
		\begin{umlsystem}[x=0, fill=lightgray!20]{Quizzipedia}
			\umlusecase[x=5, y=-8, fill=white, width=4cm, name=110]{\textbf{UC12.4:} Visualizza statistiche studente della classe}
			\umlassoc{Docente}{110}
			\umlusecase[x=5, y=-3.5, fill=white, width=4cm, name=109]{\textbf{UC12.3:} Visualizza sommario statistiche classe}
			\umlassoc{Docente}{109}
			\umlusecase[x=5, y=1, fill=white, width=4cm, name=108]{\textbf{UC12.2:} Visualizza risultati questionari della classe}
			\umlassoc{Docente}{108}
			\umlusecase[x=5, y=5.5, fill=white, width=4cm, name=107]{\textbf{UC12.1:} Visualizza risultati domande della classe}
			\umlassoc{Docente}{107}
		\end{umlsystem}
		\end{resizedtikzpicture}
		\caption{\textbf{UC12}: Visualizza statistiche classe}
		\label{UC12}
	\end{figure}
\begin{description}
\item[Attori:] Docente;
\item[Scopo e descrizione:] Il docente visualizza le statistiche di una sua classe
      \item[Precondizione:] Il docente è autenticato presso il sistema;

        \item[Flusso principale degli eventi:] \ 
 \begin{enumerate}
          \item Il docente seleziona la classe di cui vuole visualizzare le statistiche;
          \item Il docente può visualizzare i risultati delle sue domande della classe (\hyperlink{UC12.1}{UC12.1});
          \item Il docente può visualizzare i risultati dei suoi questionari della classe (\hyperlink{UC12.2}{UC12.2});
          \item Il docente può visualizzare un sommario delle statistiche della classe (\hyperlink{UC12.3}{UC12.3});
          \item Il docente può visualizzare i risultati dei test di uno studente della classe (\hyperlink{UC12.4}{UC12.4});

      \end{enumerate}
    \item[Postcondizione:] Il docente ha visualizzato le statistiche relative ad una delle sue classi.
  \end{description}
\hypertarget{UC12.1}{}
\subsection{Caso d'uso UC12.1: Visualizza risultati domande della classe}
	\begin{figure}[H]
		\centering
		\begin{resizedtikzpicture}{\textwidth}
		\umlactor[x=0, y=-1]{Docente}
		\begin{umlsystem}[x=0, fill=lightgray!20]{Quizzipedia}
			\umlusecase[x=5, y=-1.25, fill=white, width=4cm, name=107]{\textbf{UC12.1:} Visualizza risultati domande della classe}
			\umlassoc{Docente}{107}
		\end{umlsystem}
		\end{resizedtikzpicture}
		\caption{\textbf{UC12.1}: Visualizza risultati domande della classe}
		\label{UC12.1}
	\end{figure}
\begin{description}
\item[Attori:] Docente;
\item[Scopo e descrizione:] Il docente visualizza i risultati e le statistiche relative alle domande della classe selezionata
      \item[Precondizione:] Il docente ha selezionato una classe;

        \item[Flusso principale degli eventi:] \ 
 \begin{enumerate}
          \item Il docente cerca una domanda tra quelle della classe (\hyperlink{UC22}{UC22});
          \item Il docente seleziona la domanda per visualizzarne i risultati e le statistiche;

      \end{enumerate}
    \item[Postcondizione:] Il docente ha visualizzato i risultati e le statistiche relative alle domande desiderate.
  \end{description}
\hypertarget{UC12.2}{}
\subsection{Caso d'uso UC12.2: Visualizza risultati questionari della classe}
	\begin{figure}[H]
		\centering
		\begin{resizedtikzpicture}{\textwidth}
		\umlactor[x=0, y=-1]{Docente}
		\begin{umlsystem}[x=0, fill=lightgray!20]{Quizzipedia}
			\umlusecase[x=5, y=-1.25, fill=white, width=4cm, name=108]{\textbf{UC12.2:} Visualizza risultati questionari della classe}
			\umlassoc{Docente}{108}
		\end{umlsystem}
		\end{resizedtikzpicture}
		\caption{\textbf{UC12.2}: Visualizza risultati questionari della classe}
		\label{UC12.2}
	\end{figure}
\begin{description}
\item[Attori:] Docente;
\item[Scopo e descrizione:] Il docente visualizza i risultati e le statistiche relative ai questionari della classe
      \item[Precondizione:] Il docente ha selezionato una classe;

        \item[Flusso principale degli eventi:] \ 
 \begin{enumerate}
          \item Il docente cerca un questionario tra quelli della classe (\hyperlink{UC16}{UC16});
          \item Il docente seleziona il questionario per visualizzarne i risultati e le statistiche;

      \end{enumerate}
    \item[Postcondizione:] Il docente ha visualizzato i risultati e le statistiche relative ai questionari desiderate.
  \end{description}
\hypertarget{UC12.3}{}
\subsection{Caso d'uso UC12.3: Visualizza sommario statistiche classe}
	\begin{figure}[H]
		\centering
		\begin{resizedtikzpicture}{\textwidth}
		\umlactor[x=0, y=-1]{Docente}
		\begin{umlsystem}[x=0, fill=lightgray!20]{Quizzipedia}
			\umlusecase[x=5, y=-1.25, fill=white, width=4cm, name=109]{\textbf{UC12.3:} Visualizza sommario statistiche classe}
			\umlassoc{Docente}{109}
		\end{umlsystem}
		\end{resizedtikzpicture}
		\caption{\textbf{UC12.3}: Visualizza sommario statistiche classe}
		\label{UC12.3}
	\end{figure}
\begin{description}
\item[Attori:] Docente;
\item[Scopo e descrizione:] Il docente visualizza le statistiche generali relative alla classe selezionata
      \item[Precondizione:] Il docente ha selezionato una classe;

        \item[Flusso principale degli eventi:] \ 
 \begin{enumerate}
          \item Il docente seleziona la funzionalità per visualizzare il sommario delle statistiche della classe;

      \end{enumerate}
    \item[Postcondizione:] Il docente ha visualizzato le statistiche generali relative alla classe selezionata.
  \end{description}
\hypertarget{UC12.4}{}
\subsection{Caso d'uso UC12.4: Visualizza statistiche studente della classe}
	\begin{figure}[H]
		\centering
		\begin{resizedtikzpicture}{\textwidth}
		\umlactor[x=0, y=-1]{Docente}
		\begin{umlsystem}[x=0, fill=lightgray!20]{Quizzipedia}
			\umlusecase[x=5, y=-1.25, fill=white, width=4cm, name=111]{\textbf{UC12.4.1:} Visualizza risultati questionari dello studente}
			\umlassoc{Docente}{111}
		\end{umlsystem}
		\end{resizedtikzpicture}
		\caption{\textbf{UC12.4}: Visualizza statistiche studente della classe}
		\label{UC12.4}
	\end{figure}
\begin{description}
\item[Attori:] Docente;
\item[Scopo e descrizione:] Il docente visualizza i risultati e le statistiche relative alla classe selezionata
      \item[Precondizione:] Il docente ha selezionato una classe;

        \item[Flusso principale degli eventi:] \ 
 \begin{enumerate}
          \item Il docente seleziona lo studente di cui vuole vedere i risultati e le statistiche;
          \item Il docente visualizza i risultati e le statistiche di un questionario della classe eseguito dallo studente selezionato (\hyperlink{UC12.4.1}{UC12.4.1});

      \end{enumerate}
    \item[Postcondizione:] Il docente ha visualizzato i risultati e le statistiche relative allo studente desiderate.
  \end{description}
\hypertarget{UC12.4.1}{}
\subsection{Caso d'uso UC12.4.1: Visualizza risultati questionari dello studente}
	\begin{figure}[H]
		\centering
		\begin{resizedtikzpicture}{\textwidth}
		\umlactor[x=0, y=-1]{Docente}
		\begin{umlsystem}[x=0, fill=lightgray!20]{Quizzipedia}
			\umlusecase[x=5, y=-1.25, fill=white, width=4cm, name=111]{\textbf{UC12.4.1:} Visualizza risultati questionari dello studente}
			\umlassoc{Docente}{111}
		\end{umlsystem}
		\end{resizedtikzpicture}
		\caption{\textbf{UC12.4.1}: Visualizza risultati questionari dello studente}
		\label{UC12.4.1}
	\end{figure}
\begin{description}
\item[Attori:] Docente;
\item[Scopo e descrizione:] Il docente visualizza i risultati e le statistiche relative ai questionari dello studente selezionato
      \item[Precondizione:] Il docente ha selezionato uno studente della classe;

        \item[Flusso principale degli eventi:] \ 
 \begin{enumerate}
          \item Il docente cerca un questionario tra quelli dello studente nella classe (\hyperlink{UC16}{UC16});
          \item Il docente seleziona il questionario per visualizzarne i risultati e le statistiche;

      \end{enumerate}
    \item[Postcondizione:] Il docente ha visualizzato i risultati e le statistiche relative ai questionari dello studente selezionato.
  \end{description}
\hypertarget{UC13}{}
\subsection{Caso d'uso UC13: Esegui questionario}
	\begin{figure}[H]
		\centering
		\begin{resizedtikzpicture}{\textwidth}
		\umlactor[x=0, y=-1]{Studente}
		\begin{umlsystem}[x=0, fill=lightgray!20]{Quizzipedia}
			\umlusecase[x=5, y=-7.25, fill=white, width=4cm, name=132]{\textbf{UC13.3:} Conferma questionario}
			\umlassoc{Studente}{132}
			\umlusecase[x=5, y=-3.25, fill=white, width=4cm, name=205]{\textbf{UC13.11:} Visualizza valutazione questionario}
			\umlassoc{Studente}{205}
			\umlusecase[x=5, y=1.25, fill=white, width=4cm, name=160]{\textbf{UC13.10:} Feedback questionario}
			\umlassoc{Studente}{160}
			\umlusecase[x=5, y=5.25, fill=white, width=4cm, name=124]{\textbf{UC13.1:} Rispondi domanda}
			\umlassoc{Studente}{124}
		\end{umlsystem}
		\end{resizedtikzpicture}
		\caption{\textbf{UC13}: Esegui questionario}
		\label{UC13}
	\end{figure}
\begin{description}
\item[Attori:] Studente;
\item[Scopo e descrizione:] Lo studente effettua un questionario
      \item[Precondizione:] Lo studente è autenticato nel sistema;

        \item[Flusso principale degli eventi:] \ 
 \begin{enumerate}
          \item Lo studente risponde ad una domanda (\hyperlink{UC13.1}{UC13.1});
          \item Lo studente può andare alla domanda successiva;
          \item Lo studente può andare alla domanda precedente;
          \item Lo studente conferma il questionario (\hyperlink{UC13.3}{UC13.3});
          \item Lo studente visualizza la valutazione per il questionario (\hyperlink{UC13.11}{UC13.11});
          \item Lo studente può lasciare un feedback (\hyperlink{UC13.10}{UC13.10});

      \end{enumerate}
    \item[Postcondizione:] Il questionario è stato compilato e vengono visualizzati i risultati.
  \end{description}
\hypertarget{UC13.1}{}
\subsection{Caso d'uso UC13.1: Rispondi domanda}
	\begin{figure}[H]
		\centering
		\begin{resizedtikzpicture}{\textwidth}
		\umlactor[x=0, y=-1]{Studente}
		\begin{umlsystem}[x=0, fill=lightgray!20]{Quizzipedia}
			\umlusecase[x=5, y=-1, fill=white, width=4cm, name=124]{\textbf{UC13.1:} Rispondi domanda}
			\umlassoc{Studente}{124}
			\umlusecase[x=15, y=-12.25, fill=white, width=4cm, name=159]{\textbf{UC13.9:} Rispondi domanda a risposta aperta}
			\umlinherit{159}{124}
			\umlusecase[x=15, y=-7.75, fill=white, width=4cm, name=158]{\textbf{UC13.8:} Rispondi domanda con associazione di parole}
			\umlinherit{158}{124}
			\umlusecase[x=15, y=-3.25, fill=white, width=4cm, name=157]{\textbf{UC13.7:} Rispondi domanda di tipo testo con parole omesse}
			\umlinherit{157}{124}
			\umlusecase[x=15, y=1.25, fill=white, width=4cm, name=156]{\textbf{UC13.6:} Rispondi domanda a risposta multipla}
			\umlinherit{156}{124}
			\umlusecase[x=15, y=5.75, fill=white, width=4cm, name=155]{\textbf{UC13.5:} Rispondi domanda a scelta multipla}
			\umlinherit{155}{124}
			\umlusecase[x=15, y=10.25, fill=white, width=4cm, name=154]{\textbf{UC13.4:} Rispondi domanda vero/falso}
			\umlinherit{154}{124}
		\end{umlsystem}
		\end{resizedtikzpicture}
		\caption{\textbf{UC13.1}: Rispondi domanda}
		\label{UC13.1}
	\end{figure}
\begin{description}
\item[Attori:] Studente;
\item[Scopo e descrizione:] Lo studente riporta la risposta alla domanda corrente
      \item[Precondizione:] Lo studente ha iniziato l'esecuzione di un questionario;

        \item[Flusso principale degli eventi:] \ 
 \begin{enumerate}
          \item Lo studente seleziona la risposta che ritiene esatta con le modalità previste dal tipo di domanda;
          \item Lo studente può lasciare un feedback per dire che la domanda gli piace oppure per segnalare un errore (\hyperlink{UC13.10}{UC13.10});

      \end{enumerate}
    \item[Postcondizione:] Lo studente ha risposto alla domanda.
  \end{description}
\hypertarget{UC13.2}{}
\subsection{Caso d'uso UC13.2: Errore domanda non risposta}
	\begin{figure}[H]
		\centering
		\begin{resizedtikzpicture}{\textwidth}
		\umlactor[x=0, y=-1]{Studente}
		\begin{umlsystem}[x=0, fill=lightgray!20]{Quizzipedia}
			\umlusecase[x=5, y=-1, fill=white, width=4cm, name=125]{\textbf{UC13.2:} Errore domanda non risposta}
			\umlassoc{Studente}{125}
		\end{umlsystem}
		\end{resizedtikzpicture}
		\caption{\textbf{UC13.2}: Errore domanda non risposta}
		\label{UC13.2}
	\end{figure}
\begin{description}
\item[Attori:] Studente;
\item[Scopo e descrizione:] Il sistema non permette allo studente di terminare un questionario in cui non tutte le domande hanno una risposta.
      \item[Precondizione:] Lo studente non ha riposto ad alcune domande nel momento in cui conferma il questionario;

        \item[Flusso principale degli eventi:] \ 
 \begin{enumerate}
          \item Viene visualizzato un messaggio di errore;

      \end{enumerate}
    \item[Postcondizione:] Il questionario non viene terminato.
  \end{description}
\hypertarget{UC13.3}{}
\subsection{Caso d'uso UC13.3: Conferma questionario}
	\begin{figure}[H]
		\centering
		\begin{resizedtikzpicture}{\textwidth}
		\umlactor[x=0, y=-1]{Studente}
		\begin{umlsystem}[x=0, fill=lightgray!20]{Quizzipedia}
			\umlusecase[x=5, y=-1, fill=white, width=4cm, name=132]{\textbf{UC13.3:} Conferma questionario}
			\umlassoc{Studente}{132}
			\umlusecase[x=15, y=-1, fill=white, width=4cm, name=125]{\textbf{UC13.2:} Errore domanda non risposta}
			\umlextend{125}{132}
		\end{umlsystem}
		\end{resizedtikzpicture}
		\caption{\textbf{UC13.3}: Conferma questionario}
		\label{UC13.3}
	\end{figure}
\begin{description}
\item[Attori:] Studente;
\item[Scopo e descrizione:] Lo studente conferma il questionario eseguito
      \item[Precondizione:] Lo studente è autenticato nel sistema;

        \item[Flusso principale degli eventi:] \ 
 \begin{enumerate}
          \item Lo studente conferma il questionario;

      \end{enumerate}
    \item[Estensioni:]
      \begin{enumerate}
          \item Se quando lo studente conferma il questionario ci sono domande non risposte viene visualizzato un errore (\hyperlink{UC13.2}{UC13.2});

      \end{enumerate}
    \item[Postcondizione:] Il questionario viene confermato.
  \end{description}
\hypertarget{UC13.4}{}
\subsection{Caso d'uso UC13.4: Rispondi domanda vero/falso}
	\begin{figure}[H]
		\centering
		\begin{resizedtikzpicture}{\textwidth}
		\umlactor[x=0, y=-1]{Studente}
		\begin{umlsystem}[x=0, fill=lightgray!20]{Quizzipedia}
			\umlusecase[x=5, y=-1, fill=white, width=4cm, name=154]{\textbf{UC13.4:} Rispondi domanda vero/falso}
			\umlassoc{Studente}{154}
		\end{umlsystem}
		\end{resizedtikzpicture}
		\caption{\textbf{UC13.4}: Rispondi domanda vero/falso}
		\label{UC13.4}
	\end{figure}
\begin{description}
\item[Attori:] Studente;
\item[Scopo e descrizione:] Lo studente risponde ad una domanda di tipo vero/falso
      \item[Precondizione:] È stata iniziata l'esecuzione di un questionario;

        \item[Flusso principale degli eventi:] \ 
 \begin{enumerate}
          \item Lo studente seleziona vero o falso;

      \end{enumerate}
    \item[Postcondizione:] Lo studente ha risposto alla domanda di tipo vero/falso.
  \end{description}
\hypertarget{UC13.5}{}
\subsection{Caso d'uso UC13.5: Rispondi domanda a scelta multipla}
	\begin{figure}[H]
		\centering
		\begin{resizedtikzpicture}{\textwidth}
		\umlactor[x=0, y=-1]{Studente}
		\begin{umlsystem}[x=0, fill=lightgray!20]{Quizzipedia}
			\umlusecase[x=5, y=-1.25, fill=white, width=4cm, name=155]{\textbf{UC13.5:} Rispondi domanda a scelta multipla}
			\umlassoc{Studente}{155}
		\end{umlsystem}
		\end{resizedtikzpicture}
		\caption{\textbf{UC13.5}: Rispondi domanda a scelta multipla}
		\label{UC13.5}
	\end{figure}
\begin{description}
\item[Attori:] Studente;
\item[Scopo e descrizione:] Lo studente risponde ad una domanda a scelta multipla
      \item[Precondizione:] È stata iniziata l'esecuzione di un questionario;

        \item[Flusso principale degli eventi:] \ 
 \begin{enumerate}
          \item Lo studente seleziona la risposta che ritiene corretta;

      \end{enumerate}
    \item[Postcondizione:] Lo studente ha risposto alla domanda a scelta multipla.
  \end{description}
\hypertarget{UC13.6}{}
\subsection{Caso d'uso UC13.6: Rispondi domanda a risposta multipla}
	\begin{figure}[H]
		\centering
		\begin{resizedtikzpicture}{\textwidth}
		\umlactor[x=0, y=-1]{Studente}
		\begin{umlsystem}[x=0, fill=lightgray!20]{Quizzipedia}
			\umlusecase[x=5, y=-1.25, fill=white, width=4cm, name=156]{\textbf{UC13.6:} Rispondi domanda a risposta multipla}
			\umlassoc{Studente}{156}
		\end{umlsystem}
		\end{resizedtikzpicture}
		\caption{\textbf{UC13.6}: Rispondi domanda a risposta multipla}
		\label{UC13.6}
	\end{figure}
\begin{description}
\item[Attori:] Studente;
\item[Scopo e descrizione:] Lo studente risponde ad una domanda a risposta multipla
      \item[Precondizione:] È stata iniziata l'esecuzione di un questionario;

        \item[Flusso principale degli eventi:] \ 
 \begin{enumerate}
          \item Lo studente e seleziona la o le risposte che ritiene corrette;

      \end{enumerate}
    \item[Postcondizione:] Lo studente ha risposto alla domanda a risposta multipla.
  \end{description}
\hypertarget{UC13.7}{}
\subsection{Caso d'uso UC13.7: Rispondi domanda di tipo testo con parole omesse}
	\begin{figure}[H]
		\centering
		\begin{resizedtikzpicture}{\textwidth}
		\umlactor[x=0, y=-1]{Studente}
		\begin{umlsystem}[x=0, fill=lightgray!20]{Quizzipedia}
			\umlusecase[x=5, y=-1.25, fill=white, width=4cm, name=157]{\textbf{UC13.7:} Rispondi domanda di tipo testo con parole omesse}
			\umlassoc{Studente}{157}
		\end{umlsystem}
		\end{resizedtikzpicture}
		\caption{\textbf{UC13.7}: Rispondi domanda di tipo testo con parole omesse}
		\label{UC13.7}
	\end{figure}
\begin{description}
\item[Attori:] Studente;
\item[Scopo e descrizione:] Lo studente risponde ad una domanda di tipo testo con parole omesse
      \item[Precondizione:] È stata iniziata l'esecuzione di un questionario;

        \item[Flusso principale degli eventi:] \ 
 \begin{enumerate}
          \item Lo studente riempie il testo con parole omesse scegliendole da una lista;

      \end{enumerate}
    \item[Postcondizione:] Lo studente ha risposto alla domanda di tipo testo con parole omesse.
  \end{description}
\hypertarget{UC13.8}{}
\subsection{Caso d'uso UC13.8: Rispondi domanda con associazione di parole}
	\begin{figure}[H]
		\centering
		\begin{resizedtikzpicture}{\textwidth}
		\umlactor[x=0, y=-1]{Studente}
		\begin{umlsystem}[x=0, fill=lightgray!20]{Quizzipedia}
			\umlusecase[x=5, y=-1.25, fill=white, width=4cm, name=158]{\textbf{UC13.8:} Rispondi domanda con associazione di parole}
			\umlassoc{Studente}{158}
		\end{umlsystem}
		\end{resizedtikzpicture}
		\caption{\textbf{UC13.8}: Rispondi domanda con associazione di parole}
		\label{UC13.8}
	\end{figure}
\begin{description}
\item[Attori:] Studente;
\item[Scopo e descrizione:] Lo studente risponde ad una domanda con associazione di parole
      \item[Precondizione:] È stata iniziata l'esecuzione di un questionario;

        \item[Flusso principale degli eventi:] \ 
 \begin{enumerate}
          \item Lo studente e seleziona le associazioni di parole che ritiene corrette;

      \end{enumerate}
    \item[Postcondizione:] Lo studente ha risposto alla domanda con associazione di parole.
  \end{description}
\hypertarget{UC13.9}{}
\subsection{Caso d'uso UC13.9: Rispondi domanda a risposta aperta}
	\begin{figure}[H]
		\centering
		\begin{resizedtikzpicture}{\textwidth}
		\umlactor[x=0, y=-1]{Studente}
		\begin{umlsystem}[x=0, fill=lightgray!20]{Quizzipedia}
			\umlusecase[x=5, y=-1.25, fill=white, width=4cm, name=159]{\textbf{UC13.9:} Rispondi domanda a risposta aperta}
			\umlassoc{Studente}{159}
		\end{umlsystem}
		\end{resizedtikzpicture}
		\caption{\textbf{UC13.9}: Rispondi domanda a risposta aperta}
		\label{UC13.9}
	\end{figure}
\begin{description}
\item[Attori:] Studente;
\item[Scopo e descrizione:] Lo studente risponde ad una domanda a risposta aperta
      \item[Precondizione:] È stata iniziata l'esecuzione di un questionario;

        \item[Flusso principale degli eventi:] \ 
 \begin{enumerate}
          \item Lo studente inserisce la risposta come testo;

      \end{enumerate}
    \item[Postcondizione:] Lo studente ha risposto alla domanda a risposta aperta.
  \end{description}
\hypertarget{UC13.10}{}
\subsection{Caso d'uso UC13.10: Feedback questionario}
	\begin{figure}[H]
		\centering
		\begin{resizedtikzpicture}{\textwidth}
		\umlactor[x=0, y=-1]{Studente}
		\begin{umlsystem}[x=0, fill=lightgray!20]{Quizzipedia}
			\umlusecase[x=5, y=-1, fill=white, width=4cm, name=160]{\textbf{UC13.10:} Feedback questionario}
			\umlassoc{Studente}{160}
		\end{umlsystem}
		\end{resizedtikzpicture}
		\caption{\textbf{UC13.10}: Feedback questionario}
		\label{UC13.10}
	\end{figure}
\begin{description}
\item[Attori:] Studente;
\item[Scopo e descrizione:] Lo studente lascia un feedback positivo al questionario
      \item[Precondizione:] Lo studente ha completato il questionario e vede i suoi risultati;

        \item[Flusso principale degli eventi:] \ 
 \begin{enumerate}
          \item Lo studente può lasciare un feedback positivo al questionario;

      \end{enumerate}
    \item[Postcondizione:] Il feedback dello studente viene memorizzato per il questionario svolto .
  \end{description}
\hypertarget{UC13.11}{}
\subsection{Caso d'uso UC13.11: Visualizza valutazione questionario}
	\begin{figure}[H]
		\centering
		\begin{resizedtikzpicture}{\textwidth}
		\umlactor[x=0, y=-1]{Studente}
		\begin{umlsystem}[x=0, fill=lightgray!20]{Quizzipedia}
			\umlusecase[x=5, y=-1.25, fill=white, width=4cm, name=205]{\textbf{UC13.11:} Visualizza valutazione questionario}
			\umlassoc{Studente}{205}
		\end{umlsystem}
		\end{resizedtikzpicture}
		\caption{\textbf{UC13.11}: Visualizza valutazione questionario}
		\label{UC13.11}
	\end{figure}
\begin{description}
\item[Attori:] Studente;
\item[Scopo e descrizione:] Lo studente visualizza la valutazione del questionario eseguito
      \item[Precondizione:] Lo studente ha eseguito e confermato il questionario;

        \item[Flusso principale degli eventi:] \ 
 \begin{enumerate}
          \item Lo studente visualizza la valutazione del questionario eseguito;

      \end{enumerate}
    \item[Postcondizione:] Lo studente visualizza la valutazione del questionario eseguito.
  \end{description}
\hypertarget{UC14}{}
\subsection{Caso d'uso UC14: Iscrizione ad una classe}
	\begin{figure}[H]
		\centering
		\begin{resizedtikzpicture}{\textwidth}
		\umlactor[x=0, y=-1]{Studente}
		\begin{umlsystem}[x=0, fill=lightgray!20]{Quizzipedia}
			\umlusecase[x=5, y=-1, fill=white, width=4cm, name=97]{\textbf{UC14.1:} Inserisci password classe}
			\umlassoc{Studente}{97}
		\end{umlsystem}
		\end{resizedtikzpicture}
		\caption{\textbf{UC14}: Iscrizione ad una classe}
		\label{UC14}
	\end{figure}
\begin{description}
\item[Attori:] Studente;
\item[Scopo e descrizione:] Lo studente si iscrive ad una classe
      \item[Precondizione:] Lo studente è autenticato presso il sistema;

        \item[Flusso principale degli eventi:] \ 
 \begin{enumerate}
          \item Lo studente cerca una classe (\hyperlink{UC27}{UC27});
          \item Lo studente inserisce la password (che gli deve essere già stata consegnata personalmente dal docente della classe) per iscriversi alla classe (\hyperlink{UC14.1}{UC14.1});
          \item Lo studente conferma l'iscrizione alla classe;

      \end{enumerate}
    \item[Postcondizione:] Lo studente si è iscritto alla classe desiderata.
  \end{description}
\hypertarget{UC14.1}{}
\subsection{Caso d'uso UC14.1: Inserisci password classe}
	\begin{figure}[H]
		\centering
		\begin{resizedtikzpicture}{\textwidth}
		\umlactor[x=0, y=-1]{Studente}
		\begin{umlsystem}[x=0, fill=lightgray!20]{Quizzipedia}
			\umlusecase[x=5, y=-1, fill=white, width=4cm, name=97]{\textbf{UC14.1:} Inserisci password classe}
			\umlassoc{Studente}{97}
			\umlusecase[x=15, y=-1, fill=white, width=4cm, name=98]{\textbf{UC14.2:} Errore password classe}
			\umlextend{98}{97}
		\end{umlsystem}
		\end{resizedtikzpicture}
		\caption{\textbf{UC14.1}: Inserisci password classe}
		\label{UC14.1}
	\end{figure}
\begin{description}
\item[Attori:] Studente;
\item[Scopo e descrizione:] Lo studente inserisce la password per l'iscrizione alla classe
      \item[Precondizione:] Lo studente è autenticato presso il sistema;

        \item[Flusso principale degli eventi:] \ 
 \begin{enumerate}
          \item Lo studente inserisce la password corretta (che gli deve essere già stata consegnata personalmente dal docente della classe) per iscriversi alla classe;

      \end{enumerate}
    \item[Estensioni:]
      \begin{enumerate}
          \item Se la password inserita non è corretta viene visualizzato un messaggio di errore (\hyperlink{UC14.2}{UC14.2});

      \end{enumerate}
    \item[Postcondizione:] Lo studente ha inserito la password per iscriversi alla classe.
  \end{description}
\hypertarget{UC14.2}{}
\subsection{Caso d'uso UC14.2: Errore password classe}
	\begin{figure}[H]
		\centering
		\begin{resizedtikzpicture}{\textwidth}
		\umlactor[x=0, y=-1]{Studente}
		\begin{umlsystem}[x=0, fill=lightgray!20]{Quizzipedia}
			\umlusecase[x=5, y=-1, fill=white, width=4cm, name=98]{\textbf{UC14.2:} Errore password classe}
			\umlassoc{Studente}{98}
		\end{umlsystem}
		\end{resizedtikzpicture}
		\caption{\textbf{UC14.2}: Errore password classe}
		\label{UC14.2}
	\end{figure}
\begin{description}
\item[Attori:] Studente;
\item[Scopo e descrizione:] Il sistema avverte lo studente che ha inserito una password errata per l'iscrizione alla classe
      \item[Precondizione:] Lo studente ha inserito una password errata per iscriversi alla classe;

        \item[Flusso principale degli eventi:] \ 
 \begin{enumerate}
          \item Viene visualizzato un messaggio di errore;

      \end{enumerate}
    \item[Postcondizione:] Lo studente non viene iscritto alla classe.
  \end{description}
\hypertarget{UC15}{}
\subsection{Caso d'uso UC15: Visualizza storico studente}
	\begin{figure}[H]
		\centering
		\begin{resizedtikzpicture}{\textwidth}
		\umlactor[x=0, y=-1]{Studente}
		\begin{umlsystem}[x=0, fill=lightgray!20]{Quizzipedia}
			\umlusecase[x=5, y=-5.75, fill=white, width=4cm, name=139]{\textbf{UC15.3:} Visualizza sommario statistiche studente}
			\umlassoc{Studente}{139}
			\umlusecase[x=5, y=-1.25, fill=white, width=4cm, name=138]{\textbf{UC15.2:} Visualizza statistiche questionari studente}
			\umlassoc{Studente}{138}
			\umlusecase[x=5, y=3.25, fill=white, width=4cm, name=136]{\textbf{UC15.1:} Visualizza statistiche domande studente}
			\umlassoc{Studente}{136}
		\end{umlsystem}
		\end{resizedtikzpicture}
		\caption{\textbf{UC15}: Visualizza storico studente}
		\label{UC15}
	\end{figure}
\begin{description}
\item[Attori:] Studente;
\item[Scopo e descrizione:] Lo studente visualizza il proprio storico
      \item[Precondizione:] Lo studente è autenticato presso il sistema;

        \item[Flusso principale degli eventi:] \ 
 \begin{enumerate}
          \item Lo studente può visualizzare le sue risposte e le statistiche generali relative alle domande eseguite (\hyperlink{UC15.1}{UC15.1});
          \item Lo studente può visualizzare le sue risposte e le statistiche generali relative ai questionari eseguiti (\hyperlink{UC15.2}{UC15.2});
          \item Lo studente può visualizzare un sommario delle proprie statistiche (\hyperlink{UC15.3}{UC15.3});

      \end{enumerate}
    \item[Postcondizione:] Lo studente ha visualizzato le sue risposte e le statistiche generali relative ai test eseguiti.
  \end{description}
\hypertarget{UC15.1}{}
\subsection{Caso d'uso UC15.1: Visualizza statistiche domande studente}
	\begin{figure}[H]
		\centering
		\begin{resizedtikzpicture}{\textwidth}
		\umlactor[x=0, y=-1]{Studente}
		\begin{umlsystem}[x=0, fill=lightgray!20]{Quizzipedia}
			\umlusecase[x=5, y=-1.25, fill=white, width=4cm, name=136]{\textbf{UC15.1:} Visualizza statistiche domande studente}
			\umlassoc{Studente}{136}
		\end{umlsystem}
		\end{resizedtikzpicture}
		\caption{\textbf{UC15.1}: Visualizza statistiche domande studente}
		\label{UC15.1}
	\end{figure}
\begin{description}
\item[Attori:] Studente;
\item[Scopo e descrizione:] Lo studente visualizza la percentuale delle proprie risposte corrette e le statistiche generali della domanda
      \item[Precondizione:] Lo studente è autenticato presso il sistema;

        \item[Flusso principale degli eventi:] \ 
 \begin{enumerate}
          \item Lo studente cerca una domanda tra quelle che ha eseguito in passato (\hyperlink{UC22}{UC22});
          \item Lo studente seleziona la domanda per visualizzarne le risposte date e le statistiche generali;
          \item Lo studente visualizza le risposte già date in passato alla domanda selezionata;
          \item Lo studente visualizza la difficoltà della domanda calcolata come funzione delle risposte date da tutti gli utenti;

      \end{enumerate}
    \item[Postcondizione:] Lo studente ha visualizzato i risultati e le statistiche relative alle domande desiderate.
  \end{description}
\hypertarget{UC15.2}{}
\subsection{Caso d'uso UC15.2: Visualizza statistiche questionari studente}
	\begin{figure}[H]
		\centering
		\begin{resizedtikzpicture}{\textwidth}
		\umlactor[x=0, y=-1]{Studente}
		\begin{umlsystem}[x=0, fill=lightgray!20]{Quizzipedia}
			\umlusecase[x=5, y=-1.25, fill=white, width=4cm, name=138]{\textbf{UC15.2:} Visualizza statistiche questionari studente}
			\umlassoc{Studente}{138}
		\end{umlsystem}
		\end{resizedtikzpicture}
		\caption{\textbf{UC15.2}: Visualizza statistiche questionari studente}
		\label{UC15.2}
	\end{figure}
\begin{description}
\item[Attori:] Studente;
\item[Scopo e descrizione:] Lo studente visualizza i punteggi che ha ottenuto nel questionario e le statistiche generali dello stesso
      \item[Precondizione:] Lo studente è autenticato presso il sistema;

        \item[Flusso principale degli eventi:] \ 
 \begin{enumerate}
          \item Lo studente cerca un questionario tra quelli che ha eseguito in passato (\hyperlink{UC16}{UC16});
          \item Lo studente seleziona il questionario per visualizzarne le risposte date e le statistiche generali;
          \item Lo studente può vedere le statistiche di ogni singola domanda del questionario selezionato (\hyperlink{UC15.1}{UC15.1});
          \item Lo studente vede le proprie risposte al questionario e il punteggio totale;
          \item Lo studente vede la difficoltà del questionario calcolata come funzione delle risposte date da tutti gli utenti;

      \end{enumerate}
    \item[Postcondizione:] Lo studente ha visualizzato i risultati e le statistiche relative ai questionari desiderati.
  \end{description}
\hypertarget{UC15.3}{}
\subsection{Caso d'uso UC15.3: Visualizza sommario statistiche studente}
	\begin{figure}[H]
		\centering
		\begin{resizedtikzpicture}{\textwidth}
		\umlactor[x=0, y=-1]{Studente}
		\begin{umlsystem}[x=0, fill=lightgray!20]{Quizzipedia}
			\umlusecase[x=5, y=-1.25, fill=white, width=4cm, name=139]{\textbf{UC15.3:} Visualizza sommario statistiche studente}
			\umlassoc{Studente}{139}
		\end{umlsystem}
		\end{resizedtikzpicture}
		\caption{\textbf{UC15.3}: Visualizza sommario statistiche studente}
		\label{UC15.3}
	\end{figure}
\begin{description}
\item[Attori:] Studente;
\item[Scopo e descrizione:] Lo studente visualizza il totale delle risposte corrette sul totale delle risposte date e la media dei punteggi su tutti i questionari
      \item[Precondizione:] Lo studente è autenticato presso il sistema;

        \item[Flusso principale degli eventi:] \ 
 \begin{enumerate}
          \item Lo studente visualizza il totale delle domande eseguite e il numero di risposte corrette;
          \item Lo studente visualizza il totale dei questionari eseguiti e la media dei risultati ottenuti;
          \item Lo studente visualizza la media delle difficoltà delle domande a cui ha risposto correttamente;
          \item Lo studente visualizza la media delle difficoltà delle domande a cui non ha risposto correttamente;
          \item Per ogni argomento lo studente visualizza il totale delle domande eseguite, il numero di risposte corrette, la media delle difficoltà delle risposte corrette e la media delle difficoltà delle risposte non corrette;

      \end{enumerate}
    \item[Postcondizione:] Lo studente ha visualizzato i risultati e le statistiche relative alle domande desiderate.
  \end{description}
\hypertarget{UC16}{}
\subsection{Caso d'uso UC16: Ricerca questionario}
	\begin{figure}[H]
		\centering
		\begin{resizedtikzpicture}{\textwidth}
		\umlactor[x=0, y=-1]{Utente}
		\begin{umlsystem}[x=0, fill=lightgray!20]{Quizzipedia}
			\umlusecase[x=5, y=-1, fill=white, width=4cm, name=17]{\textbf{UC16:} Ricerca questionario}
			\umlassoc{Utente}{17}
			\umlusecase[x=15, y=-9.25, fill=white, width=4cm, name=22]{\textbf{UC21:} Ricerca questionario per difficoltà}
			\umlinherit{22}{17}
			\umlusecase[x=15, y=-4.75, fill=white, width=4cm, name=21]{\textbf{UC20:} Ricerca questionario per docente}
			\umlinherit{21}{17}
			\umlusecase[x=15, y=-0.75, fill=white, width=4cm, name=20]{\textbf{UC19:} Ricerca questionario per argomento}
			\umlinherit{20}{17}
			\umlusecase[x=15, y=3.25, fill=white, width=4cm, name=19]{\textbf{UC18:} Ricerca questionario per classe}
			\umlinherit{19}{17}
			\umlusecase[x=15, y=7.25, fill=white, width=4cm, name=18]{\textbf{UC17:} Ricerca questionario per titolo}
			\umlinherit{18}{17}
		\end{umlsystem}
		\end{resizedtikzpicture}
		\caption{\textbf{UC16}: Ricerca questionario}
		\label{UC16}
	\end{figure}
\begin{description}
\item[Attori:] Utente;
\item[Scopo e descrizione:] L'utente ricerca un questionario
      \item[Precondizione:] L'utente è autenticato presso il sistema;

        \item[Flusso principale degli eventi:] \ 
 \begin{enumerate}
          \item L'utente inserisce i dati per la ricerca;

      \end{enumerate}
    \item[Postcondizione:] Il sistema mostra la lista dei questionari che soddisfano la ricerca.
  \end{description}
\hypertarget{UC17}{}
\subsection{Caso d'uso UC17: Ricerca questionario per titolo}
	\begin{figure}[H]
		\centering
		\begin{resizedtikzpicture}{\textwidth}
		\umlactor[x=0, y=-1]{Utente}
		\begin{umlsystem}[x=0, fill=lightgray!20]{Quizzipedia}
			\umlusecase[x=5, y=-1, fill=white, width=4cm, name=18]{\textbf{UC17:} Ricerca questionario per titolo}
			\umlassoc{Utente}{18}
		\end{umlsystem}
		\end{resizedtikzpicture}
		\caption{\textbf{UC17}: Ricerca questionario per titolo}
		\label{UC17}
	\end{figure}
\begin{description}
\item[Attori:] Utente;
\item[Scopo e descrizione:] L'utente ricerca un questionario per titolo
      \item[Precondizione:] L'utente è autenticato presso il sistema;

        \item[Flusso principale degli eventi:] \ 
 \begin{enumerate}
          \item L'utente inserisce il titolo del questionario che vuole cercare;

      \end{enumerate}
    \item[Postcondizione:] Il sistema mostra la lista dei questionari il cui titolo corrisponde al titolo ricercato.
  \end{description}
\hypertarget{UC18}{}
\subsection{Caso d'uso UC18: Ricerca questionario per classe}
	\begin{figure}[H]
		\centering
		\begin{resizedtikzpicture}{\textwidth}
		\umlactor[x=0, y=-1]{Utente}
		\begin{umlsystem}[x=0, fill=lightgray!20]{Quizzipedia}
			\umlusecase[x=5, y=-1, fill=white, width=4cm, name=19]{\textbf{UC18:} Ricerca questionario per classe}
			\umlassoc{Utente}{19}
		\end{umlsystem}
		\end{resizedtikzpicture}
		\caption{\textbf{UC18}: Ricerca questionario per classe}
		\label{UC18}
	\end{figure}
\begin{description}
\item[Attori:] Utente;
\item[Scopo e descrizione:] L'utente ricerca il questionario per classe
      \item[Precondizione:] L'utente è autenticato presso il sistema;

        \item[Flusso principale degli eventi:] \ 
 \begin{enumerate}
          \item L'utente seleziona la classe di cui vuole visualizzare i questionari;

      \end{enumerate}
    \item[Postcondizione:] Il sistema mostra la lista dei questionari la cui classe corrisponde alla classe ricercata.
  \end{description}
\hypertarget{UC19}{}
\subsection{Caso d'uso UC19: Ricerca questionario per argomento}
	\begin{figure}[H]
		\centering
		\begin{resizedtikzpicture}{\textwidth}
		\umlactor[x=0, y=-1]{Utente}
		\begin{umlsystem}[x=0, fill=lightgray!20]{Quizzipedia}
			\umlusecase[x=5, y=-1, fill=white, width=4cm, name=20]{\textbf{UC19:} Ricerca questionario per argomento}
			\umlassoc{Utente}{20}
		\end{umlsystem}
		\end{resizedtikzpicture}
		\caption{\textbf{UC19}: Ricerca questionario per argomento}
		\label{UC19}
	\end{figure}
\begin{description}
\item[Attori:] Utente;
\item[Scopo e descrizione:] L'utente ricerca questionario per argomento
      \item[Precondizione:] L'utente è autenticato presso il sistema;

        \item[Flusso principale degli eventi:] \ 
 \begin{enumerate}
          \item L'utente seleziona gli argomento di cui vuole visualizzare i questionari;

      \end{enumerate}
    \item[Postcondizione:] Il sistema mostra la lista dei questionari i cui argomenti corrispondono agli argomenti ricercati.
  \end{description}
\hypertarget{UC20}{}
\subsection{Caso d'uso UC20: Ricerca questionario per docente}
	\begin{figure}[H]
		\centering
		\begin{resizedtikzpicture}{\textwidth}
		\umlactor[x=0, y=-1]{Utente}
		\begin{umlsystem}[x=0, fill=lightgray!20]{Quizzipedia}
			\umlusecase[x=5, y=-1, fill=white, width=4cm, name=21]{\textbf{UC20:} Ricerca questionario per docente}
			\umlassoc{Utente}{21}
		\end{umlsystem}
		\end{resizedtikzpicture}
		\caption{\textbf{UC20}: Ricerca questionario per docente}
		\label{UC20}
	\end{figure}
\begin{description}
\item[Attori:] Utente;
\item[Scopo e descrizione:] L'utente ricerca questionario per docente
      \item[Precondizione:] L'utente è autenticato presso il sistema;

        \item[Flusso principale degli eventi:] \ 
 \begin{enumerate}
          \item L'utente inserisce il nome del docente di cui vuole cercare i questionari;

      \end{enumerate}
    \item[Postcondizione:] Il sistema mostra la lista dei questionari il cui docente corrisponde al docente ricercato.
  \end{description}
\hypertarget{UC21}{}
\subsection{Caso d'uso UC21: Ricerca questionario per difficoltà}
	\begin{figure}[H]
		\centering
		\begin{resizedtikzpicture}{\textwidth}
		\umlactor[x=0, y=-1]{Utente}
		\begin{umlsystem}[x=0, fill=lightgray!20]{Quizzipedia}
			\umlusecase[x=5, y=-1.25, fill=white, width=4cm, name=22]{\textbf{UC21:} Ricerca questionario per difficoltà}
			\umlassoc{Utente}{22}
		\end{umlsystem}
		\end{resizedtikzpicture}
		\caption{\textbf{UC21}: Ricerca questionario per difficoltà}
		\label{UC21}
	\end{figure}
\begin{description}
\item[Attori:] Utente;
\item[Scopo e descrizione:] L'utente ricerca un questionario per difficoltà
      \item[Precondizione:] L'utente è autenticato presso il sistema;

        \item[Flusso principale degli eventi:] \ 
 \begin{enumerate}
          \item L'utente inserisce un limite inferiore e superiore di difficoltà di cui vuole cercare questionari;

      \end{enumerate}
    \item[Postcondizione:] Il sistema mostra la lista dei questionari la cui difficoltà corrisponde alla difficoltà ricercata.
  \end{description}
\hypertarget{UC22}{}
\subsection{Caso d'uso UC22: Ricerca domanda}
	\begin{figure}[H]
		\centering
		\begin{resizedtikzpicture}{\textwidth}
		\umlactor[x=0, y=-1]{Docente}
		\begin{umlsystem}[x=0, fill=lightgray!20]{Quizzipedia}
			\umlusecase[x=5, y=-1, fill=white, width=4cm, name=23]{\textbf{UC22:} Ricerca domanda}
			\umlassoc{Docente}{23}
			\umlusecase[x=15, y=-7, fill=white, width=4cm, name=27]{\textbf{UC26:} Ricerca domanda per docente}
			\umlinherit{27}{23}
			\umlusecase[x=15, y=-3, fill=white, width=4cm, name=26]{\textbf{UC25:} Ricerca domanda per difficoltà}
			\umlinherit{26}{23}
			\umlusecase[x=15, y=1, fill=white, width=4cm, name=25]{\textbf{UC24:} Ricerca domanda per argomento}
			\umlinherit{25}{23}
			\umlusecase[x=15, y=5, fill=white, width=4cm, name=24]{\textbf{UC23:} Ricerca domanda per keywords}
			\umlinherit{24}{23}
		\end{umlsystem}
		\end{resizedtikzpicture}
		\caption{\textbf{UC22}: Ricerca domanda}
		\label{UC22}
	\end{figure}
\begin{description}
\item[Attori:] Docente;
\item[Scopo e descrizione:] Il docente ricerca una domanda
      \item[Precondizione:] Il docente è autenticato presso il sistema;

        \item[Flusso principale degli eventi:] \ 
 \begin{enumerate}
          \item L'utente inserisce i dati per la ricerca	;

      \end{enumerate}
    \item[Postcondizione:] Il sistema mostra la lista delle domande che soddisfano la ricerca.
  \end{description}
\hypertarget{UC23}{}
\subsection{Caso d'uso UC23: Ricerca domanda per keywords}
	\begin{figure}[H]
		\centering
		\begin{resizedtikzpicture}{\textwidth}
		\umlactor[x=0, y=-1]{Docente}
		\begin{umlsystem}[x=0, fill=lightgray!20]{Quizzipedia}
			\umlusecase[x=5, y=-1, fill=white, width=4cm, name=24]{\textbf{UC23:} Ricerca domanda per keywords}
			\umlassoc{Docente}{24}
		\end{umlsystem}
		\end{resizedtikzpicture}
		\caption{\textbf{UC23}: Ricerca domanda per keywords}
		\label{UC23}
	\end{figure}
\begin{description}
\item[Attori:] Docente;
\item[Scopo e descrizione:] Il docente  ricerca una domanda per keywords

      \item[Precondizione:] Il docente è autenticato presso il sistema
;

        \item[Flusso principale degli eventi:] \ 
 \begin{enumerate}
          \item L'utente inserisce le keywords presenti nel testo delle domande che vuole cercare;

      \end{enumerate}
    \item[Postcondizione:] Il sistema mostra la lista delle domande che contengono nel titolo o nel corpo le parole chiave specificate.
  \end{description}
\hypertarget{UC24}{}
\subsection{Caso d'uso UC24: Ricerca domanda per argomento}
	\begin{figure}[H]
		\centering
		\begin{resizedtikzpicture}{\textwidth}
		\umlactor[x=0, y=-1]{Docente}
		\begin{umlsystem}[x=0, fill=lightgray!20]{Quizzipedia}
			\umlusecase[x=5, y=-1, fill=white, width=4cm, name=25]{\textbf{UC24:} Ricerca domanda per argomento}
			\umlassoc{Docente}{25}
		\end{umlsystem}
		\end{resizedtikzpicture}
		\caption{\textbf{UC24}: Ricerca domanda per argomento}
		\label{UC24}
	\end{figure}
\begin{description}
\item[Attori:] Docente;
\item[Scopo e descrizione:] Il docente ricerca una domanda per argomento
      \item[Precondizione:] Il docente è autenticato presso il sistema
;

        \item[Flusso principale degli eventi:] \ 
 \begin{enumerate}
          \item L'utente seleziona gli argomenti di cui vuole visualizzare le domande;

      \end{enumerate}
    \item[Postcondizione:] Il sistema mostra la lista delle domande che contengono gli argomenti selezionati.
  \end{description}
\hypertarget{UC25}{}
\subsection{Caso d'uso UC25: Ricerca domanda per difficoltà}
	\begin{figure}[H]
		\centering
		\begin{resizedtikzpicture}{\textwidth}
		\umlactor[x=0, y=-1]{Docente}
		\begin{umlsystem}[x=0, fill=lightgray!20]{Quizzipedia}
			\umlusecase[x=5, y=-1, fill=white, width=4cm, name=26]{\textbf{UC25:} Ricerca domanda per difficoltà}
			\umlassoc{Docente}{26}
		\end{umlsystem}
		\end{resizedtikzpicture}
		\caption{\textbf{UC25}: Ricerca domanda per difficoltà}
		\label{UC25}
	\end{figure}
\begin{description}
\item[Attori:] Docente;
\item[Scopo e descrizione:] Il docente ricerca una domanda per difficoltà

      \item[Precondizione:] Il docente è autenticato presso il sistema
;

        \item[Flusso principale degli eventi:] \ 
 \begin{enumerate}
          \item L'utente inserisce il limite minimo e massimo di difficoltà di cui vuole visualizzare le domande;

      \end{enumerate}
    \item[Postcondizione:] Il sistema mostra la lista delle domande con la difficoltà selezionata.
  \end{description}
\hypertarget{UC26}{}
\subsection{Caso d'uso UC26: Ricerca domanda per docente}
	\begin{figure}[H]
		\centering
		\begin{resizedtikzpicture}{\textwidth}
		\umlactor[x=0, y=-1]{Docente}
		\begin{umlsystem}[x=0, fill=lightgray!20]{Quizzipedia}
			\umlusecase[x=5, y=-1, fill=white, width=4cm, name=27]{\textbf{UC26:} Ricerca domanda per docente}
			\umlassoc{Docente}{27}
		\end{umlsystem}
		\end{resizedtikzpicture}
		\caption{\textbf{UC26}: Ricerca domanda per docente}
		\label{UC26}
	\end{figure}
\begin{description}
\item[Attori:] Docente;
\item[Scopo e descrizione:] Il docente ricerca una domanda per docente
      \item[Precondizione:] Il docente è autenticato presso il sistema
;

        \item[Flusso principale degli eventi:] \ 
 \begin{enumerate}
          \item L'utente inserisce il nome del docente di cui vuole visualizzare le domande;

      \end{enumerate}
    \item[Postcondizione:] Il sistema mostra la lista delle domande create dal docente selezionato.
  \end{description}
\hypertarget{UC27}{}
\subsection{Caso d'uso UC27: Ricerca classe}
	\begin{figure}[H]
		\centering
		\begin{resizedtikzpicture}{\textwidth}
		\umlactor[x=0, y=-1]{Studente}
		\begin{umlsystem}[x=0, fill=lightgray!20]{Quizzipedia}
			\umlusecase[x=5, y=-0.75, fill=white, width=4cm, name=28]{\textbf{UC27:} Ricerca classe}
			\umlassoc{Studente}{28}
			\umlusecase[x=15, y=-3, fill=white, width=4cm, name=30]{\textbf{UC29:} Ricerca classe per argomento}
			\umlinherit{30}{28}
			\umlusecase[x=15, y=1, fill=white, width=4cm, name=29]{\textbf{UC28:} Ricerca classe per docente}
			\umlinherit{29}{28}
		\end{umlsystem}
		\end{resizedtikzpicture}
		\caption{\textbf{UC27}: Ricerca classe}
		\label{UC27}
	\end{figure}
\begin{description}
\item[Attori:] Studente;
\item[Scopo e descrizione:] Lo studente ricerca una classe
      \item[Precondizione:] Lo studente è autenticato presso il sistema;

        \item[Flusso principale degli eventi:] \ 
 \begin{enumerate}
          \item Lo studente inserisce i dati per la ricerca;

      \end{enumerate}
    \item[Postcondizione:] Il sistema mostra la lista delle classi che soddisfano la ricerca.
  \end{description}
\hypertarget{UC28}{}
\subsection{Caso d'uso UC28: Ricerca classe per docente}
	\begin{figure}[H]
		\centering
		\begin{resizedtikzpicture}{\textwidth}
		\umlactor[x=0, y=-1]{Studente}
		\begin{umlsystem}[x=0, fill=lightgray!20]{Quizzipedia}
			\umlusecase[x=5, y=-1, fill=white, width=4cm, name=29]{\textbf{UC28:} Ricerca classe per docente}
			\umlassoc{Studente}{29}
		\end{umlsystem}
		\end{resizedtikzpicture}
		\caption{\textbf{UC28}: Ricerca classe per docente}
		\label{UC28}
	\end{figure}
\begin{description}
\item[Attori:] Studente;
\item[Scopo e descrizione:] Lo studente ricerca una classe per docente
      \item[Precondizione:] Lo studente è autenticato presso il sistema;

        \item[Flusso principale degli eventi:] \ 
 \begin{enumerate}
          \item Lo studente inserisce il nome del docente di cui le classi;

      \end{enumerate}
    \item[Postcondizione:] Il sistema mostra la lista delle classi che soddisfano la ricerca in base al docente selezionato.
  \end{description}
\hypertarget{UC29}{}
\subsection{Caso d'uso UC29: Ricerca classe per argomento}
	\begin{figure}[H]
		\centering
		\begin{resizedtikzpicture}{\textwidth}
		\umlactor[x=0, y=-1]{Studente}
		\begin{umlsystem}[x=0, fill=lightgray!20]{Quizzipedia}
			\umlusecase[x=5, y=-1, fill=white, width=4cm, name=30]{\textbf{UC29:} Ricerca classe per argomento}
			\umlassoc{Studente}{30}
		\end{umlsystem}
		\end{resizedtikzpicture}
		\caption{\textbf{UC29}: Ricerca classe per argomento}
		\label{UC29}
	\end{figure}
\begin{description}
\item[Attori:] Studente;
\item[Scopo e descrizione:] Lo studente ricerca una classe per argomento
      \item[Precondizione:] Lo studente è autenticato presso il sistema;

        \item[Flusso principale degli eventi:] \ 
 \begin{enumerate}
          \item Lo studente seleziona gli argomenti di cui vuole visualizzare le classi;

      \end{enumerate}
    \item[Postcondizione:] Il sistema mostra la lista delle classi che soddisfano la ricerca in base agli argomenti selezionati.
  \end{description}
\hypertarget{UC30}{}
\subsection{Caso d'uso UC30: Azioni Amministratore}
	\begin{figure}[H]
		\centering
		\begin{resizedtikzpicture}{\textwidth}
		\umlactor[x=0, y=-1]{Amministratore}
		\begin{umlsystem}[x=0, fill=lightgray!20]{Quizzipedia}
			\umlusecase[x=5, y=-2.75, fill=white, width=4cm, name=35]{\textbf{UC30.2:} Rimozione utente}
			\umlassoc{Amministratore}{35}
			\umlusecase[x=5, y=1.25, fill=white, width=4cm, name=34]{\textbf{UC30.1:} Cambia ruolo}
			\umlassoc{Amministratore}{34}
		\end{umlsystem}
		\end{resizedtikzpicture}
		\caption{\textbf{UC30}: Azioni Amministratore}
		\label{UC30}
	\end{figure}
\begin{description}
\item[Attori:] Amministratore;
\item[Scopo e descrizione:] Descrive le azione che può compiere un aministratore
      \item[Precondizione:] L'amministratore è autenticato nel sistema;

        \item[Flusso principale degli eventi:] \ 
 \begin{enumerate}
          \item L'amministratore può impostare il ruolo di un utente con ruolo inferiore al proprio (\hyperlink{UC30.1}{UC30.1});
          \item L'amministratore può rimuovere un utente con ruolo inferiore al proprio (\hyperlink{UC30.2}{UC30.2});

      \end{enumerate}
    \item[Postcondizione:] Il sistema ha ottenuto le informazioni sulle operazioni che l’amministratore desidera eseguire.
  \end{description}
\hypertarget{UC30.1}{}
\subsection{Caso d'uso UC30.1: Cambia ruolo}
	\begin{figure}[H]
		\centering
		\begin{resizedtikzpicture}{\textwidth}
		\umlactor[x=0, y=-1]{Amministratore}
		\begin{umlsystem}[x=0, fill=lightgray!20]{Quizzipedia}
			\umlusecase[x=5, y=-0.75, fill=white, width=4cm, name=34]{\textbf{UC30.1:} Cambia ruolo}
			\umlassoc{Amministratore}{34}
		\end{umlsystem}
		\end{resizedtikzpicture}
		\caption{\textbf{UC30.1}: Cambia ruolo}
		\label{UC30.1}
	\end{figure}
\begin{description}
\item[Attori:] Amministratore;
\item[Scopo e descrizione:] L'amministratore imposta il ruolo dello studente a docente o viceversa
      \item[Precondizione:] L'utente selezionato è uno studente o docente;

        \item[Flusso principale degli eventi:] \ 
 \begin{enumerate}
          \item L'amministratore seleziona un utente con ruolo inferiore al proprio;
          \item L'amministratore imposta il ruolo dell'utente selezionato ad uno inferiore al proprio;
          \item L'amministratore conferma l'operazione da eseguire;

      \end{enumerate}
    \item[Postcondizione:] Il ruolo dell'utente è diventato quello selezionato.
  \end{description}
\hypertarget{UC30.2}{}
\subsection{Caso d'uso UC30.2: Rimozione utente}
	\begin{figure}[H]
		\centering
		\begin{resizedtikzpicture}{\textwidth}
		\umlactor[x=0, y=-1]{Amministratore}
		\begin{umlsystem}[x=0, fill=lightgray!20]{Quizzipedia}
			\umlusecase[x=5, y=-1, fill=white, width=4cm, name=35]{\textbf{UC30.2:} Rimozione utente}
			\umlassoc{Amministratore}{35}
		\end{umlsystem}
		\end{resizedtikzpicture}
		\caption{\textbf{UC30.2}: Rimozione utente}
		\label{UC30.2}
	\end{figure}
\begin{description}
\item[Attori:] Amministratore;
\item[Scopo e descrizione:] L'amministratore rimuove un utente con ruolo inferiore al proprio
      \item[Precondizione:] L'utente selezionato esiste nel sistema ed ha un ruolo inferiore al proprio;

        \item[Flusso principale degli eventi:] \ 
 \begin{enumerate}
          \item Amministratore ricerca l'utente con ruolo inferiore al proprio;
          \item Amministratore seleziona l'utente con ruolo inferiore al proprio;
          \item Amministratore conferma la rimozione;

      \end{enumerate}
    \item[Postcondizione:] L'utente selezionato è stato rimosso dal sistema.
  \end{description}
\hypertarget{UC31}{}
\subsection{Caso d'uso UC31: Ricerca utente}
	\begin{figure}[H]
		\centering
		\begin{resizedtikzpicture}{\textwidth}
		\umlactor[x=0, y=-1]{Amministratore}
		\begin{umlsystem}[x=0, fill=lightgray!20]{Quizzipedia}
			\umlusecase[x=5, y=-0.75, fill=white, width=4cm, name=200]{\textbf{UC31:} Ricerca utente}
			\umlassoc{Amministratore}{200}
			\umlusecase[x=15, y=-3, fill=white, width=4cm, name=203]{\textbf{UC33:} Ricerca utente per username}
			\umlinherit{203}{200}
			\umlusecase[x=15, y=1, fill=white, width=4cm, name=201]{\textbf{UC32:} Ricerca utente per nome completo}
			\umlinherit{201}{200}
		\end{umlsystem}
		\end{resizedtikzpicture}
		\caption{\textbf{UC31}: Ricerca utente}
		\label{UC31}
	\end{figure}
\begin{description}
\item[Attori:] Amministratore;
\item[Scopo e descrizione:] L'amministratore ricerca un utente
      \item[Precondizione:] L'amministratore è autenticato presso il sistema;

        \item[Flusso principale degli eventi:] \ 
 \begin{enumerate}
          \item L'amministratore inserisce i dati per la ricerca;

      \end{enumerate}
    \item[Postcondizione:] Il sistema mostra la lista degli utenti che soddisfano i criteri di ricerca.
  \end{description}
\hypertarget{UC32}{}
\subsection{Caso d'uso UC32: Ricerca utente per nome completo}
	\begin{figure}[H]
		\centering
		\begin{resizedtikzpicture}{\textwidth}
		\umlactor[x=0, y=-1]{Amministratore}
		\begin{umlsystem}[x=0, fill=lightgray!20]{Quizzipedia}
			\umlusecase[x=5, y=-1, fill=white, width=4cm, name=201]{\textbf{UC32:} Ricerca utente per nome completo}
			\umlassoc{Amministratore}{201}
			\umlusecase[x=15, y=-1, fill=white, width=4cm, name=204]{\textbf{UC34:} Ricerca utente per ruolo}
			\umlinherit{204}{201}
		\end{umlsystem}
		\end{resizedtikzpicture}
		\caption{\textbf{UC32}: Ricerca utente per nome completo}
		\label{UC32}
	\end{figure}
\begin{description}
\item[Attori:] Amministratore;
\item[Scopo e descrizione:] L'amministratore ricerca un utente per nome completo
      \item[Precondizione:] L'amministratore è autenticato presso il sistema;

        \item[Flusso principale degli eventi:] \ 
 \begin{enumerate}
          \item L'amministratore ricerca un utente per nome completo;

      \end{enumerate}
    \item[Postcondizione:] Il sistema mostra la lista degli utenti il cui nome completo corrisponde al nome completo ricercato.
  \end{description}
\hypertarget{UC33}{}
\subsection{Caso d'uso UC33: Ricerca utente per username}
	\begin{figure}[H]
		\centering
		\begin{resizedtikzpicture}{\textwidth}
		\umlactor[x=0, y=-1]{Amministratore}
		\begin{umlsystem}[x=0, fill=lightgray!20]{Quizzipedia}
			\umlusecase[x=5, y=-1, fill=white, width=4cm, name=203]{\textbf{UC33:} Ricerca utente per username}
			\umlassoc{Amministratore}{203}
		\end{umlsystem}
		\end{resizedtikzpicture}
		\caption{\textbf{UC33}: Ricerca utente per username}
		\label{UC33}
	\end{figure}
\begin{description}
\item[Attori:] Amministratore;
\item[Scopo e descrizione:] L'amministratore ricerca un utente per username
      \item[Precondizione:] L'amministratore è autenticato presso il sistema;

        \item[Flusso principale degli eventi:] \ 
 \begin{enumerate}
          \item L'amministratore ricerca un utente per username;

      \end{enumerate}
    \item[Postcondizione:] Il sistema mostra la lista degli utenti il cui username corrisponde all'username ricercato.
  \end{description}
\hypertarget{UC34}{}
\subsection{Caso d'uso UC34: Ricerca utente per ruolo}
	\begin{figure}[H]
		\centering
		\begin{resizedtikzpicture}{\textwidth}
		\umlactor[x=0, y=-1]{Amministratore}
		\begin{umlsystem}[x=0, fill=lightgray!20]{Quizzipedia}
			\umlusecase[x=5, y=-1, fill=white, width=4cm, name=204]{\textbf{UC34:} Ricerca utente per ruolo}
			\umlassoc{Amministratore}{204}
		\end{umlsystem}
		\end{resizedtikzpicture}
		\caption{\textbf{UC34}: Ricerca utente per ruolo}
		\label{UC34}
	\end{figure}
\begin{description}
\item[Attori:] Amministratore;
\item[Scopo e descrizione:] L'amministratore ricerca un utente in base al ruolo 

      \item[Precondizione:] L'amministratore è autenticato presso il sistema
;

        \item[Flusso principale degli eventi:] \ 
 \begin{enumerate}
          \item L'amministratore ricerca un utente in base al ruolo 
;

      \end{enumerate}
    \item[Postcondizione:] Il sistema mostra la lista degli utenti il cui ruolo completo corrisponde al nome completo ricercato
.
  \end{description}
\hypertarget{UC35}{}
\subsection{Caso d'uso UC35: Recupero password}
	\begin{figure}[H]
		\centering
		\begin{resizedtikzpicture}{\textwidth}
		\umlactor[x=0, y=-1]{Utente}
		\begin{umlsystem}[x=0, fill=lightgray!20]{Quizzipedia}
			\umlusecase[x=5, y=-1, fill=white, width=4cm, name=209]{\textbf{UC35:} Recupero password}
			\umlassoc{Utente}{209}
			\umlusecase[x=15, y=-1.25, fill=white, width=4cm, name=210]{\textbf{UC35.1:} Errore mail per recupero password non presente}
			\umlextend{210}{209}
		\end{umlsystem}
		\end{resizedtikzpicture}
		\caption{\textbf{UC35}: Recupero password}
		\label{UC35}
	\end{figure}
\begin{description}
\item[Attori:] Utente;
\item[Scopo e descrizione:] L'ospite non ricorda la password di accesso al sistema e inizia la procedura di recupero password 
      \item[Precondizione:] L'ospite non ricorda la password;

        \item[Flusso principale degli eventi:] \ 
 \begin{enumerate}
          \item L'ospite inserisce la mail registrata nel sistema sulla quale ricevere la nuova password;
          \item L'ospite conferma la procedura;

      \end{enumerate}
    \item[Estensioni:]
      \begin{enumerate}
          \item Se la mail inserita dall'ospite non è presente nel sistema viene visualizzato un messaggio d'errore (\hyperlink{UC35.1}{UC35.1});

      \end{enumerate}
    \item[Postcondizione:] L'ospite ha ricevuto una mail con la nuova password.
  \end{description}
\hypertarget{UC35.1}{}
\subsection{Caso d'uso UC35.1: Errore mail per recupero password non presente}
	\begin{figure}[H]
		\centering
		\begin{resizedtikzpicture}{\textwidth}
		\umlactor[x=0, y=-1]{Ospite}
		\begin{umlsystem}[x=0, fill=lightgray!20]{Quizzipedia}
			\umlusecase[x=5, y=-1.25, fill=white, width=4cm, name=210]{\textbf{UC35.1:} Errore mail per recupero password non presente}
			\umlassoc{Ospite}{210}
		\end{umlsystem}
		\end{resizedtikzpicture}
		\caption{\textbf{UC35.1}: Errore mail per recupero password non presente}
		\label{UC35.1}
	\end{figure}
\begin{description}
\item[Attori:] Ospite;
\item[Scopo e descrizione:] L'ospite inserisce una mail per il recupero password che non è presente nel sistema
      \item[Precondizione:] L'ospite non ricorda la password;

        \item[Flusso principale degli eventi:] \ 
 \begin{enumerate}
          \item L'ospite inserisce una mail per il recupero password che non è presente nel sistema;

      \end{enumerate}
    \item[Postcondizione:] Viene visualizzato un messaggio di errore.
  \end{description}
 
