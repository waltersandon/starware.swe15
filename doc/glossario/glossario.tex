%pdflatex -synctex=1 -interaction=nonstopmode %.tex|makeglossaries %|pdflatex -synctex=1 -interaction=nonstopmode %.tex|pdflatex -synctex=1 -interaction=nonstopmode %.tex
\makeglossary

\newglossaryentry{responsabile} {
	name=responsabile,
	description={è il responsabile della gestione, pianificazione e realizzazione del progetto.},
	plural=Responsabili
}

\newglossaryentry{verificatore} {
	name=verificatore,
	description={è il responsabile dell'attività di verifica.},
	plural=Verificatori
}

\newglossaryentry{programmatore} {
	name=programmatore,
	description={è responsabile delle attività di codifica miranti alla realizzazione del prodotto e delle componenti di ausilio necessarie per l'esecuzione delle prove di verifica e validazione.},
	plural=programmatori
}

\newglossaryentry{progettista} {
	name=progettista,
	description={è responsabile delle attività di progettazione.},
	plural=Progettisti
}

\newglossaryentry{analista} {
	name=analista,
	description={è responsabile delle attività di analisi. },
	plural=Analisti
}

\newglossaryentry{amministratore} {
	name=amministratore,
	description={è responsabile dell'efficienza e dell'operatività dell'ambiente di sviluppo; si occupa della redazione e attuazione di piani e procedure di gestione della qualità; inoltre gestisce l'archivio della documentazione del progetto.},
	plural=Amministratori
}

\newglossaryentry{revisione dei requisiti} {
	name=revisione dei requisiti,
	description={è una revisione formale che determina l'accesso del gruppo al progetto didattico e la concordanza con il cliente di una visione condivisa del prodotto atteso.}
}

\newglossaryentry{revisione di accettazione} {
	name=revisione di accettazione,
	description={è una revisione formale per l'accertamento del soddisfacimento di tutti i requisiti e il completamento del progetto.}
}

\newglossaryentry{revisione di progettazione} {
	name=revisione di progettazione,
	description={è una revisione di progresso che accerta la realizzabilità del prodotto e informa il cliente sulle caratteristiche del prodotto.}
}

\newglossaryentry{revisione di qualifica} {
	name=revisione di qualifica,
	description={è una revisione di progresso che approva l'esito finale delle verifiche e attiva la fase di validazione.}
}

\newglossaryentry{analisi} {
	name=analisi,
	description={è il periodo di preparazione e produzioni di documenti che precede la Revisione dei requisiti.}
}

\newglossaryentry{progettazione architetturale e di dettaglio} {
	name=progettazione architetturale e di dettaglio,
	description={è il periodo che intercorre tra la Revisione dei requisiti e la Revisione di progettazione.}
}

\newglossaryentry{codifica} {
	name=codifica,
	description={è il periodo che intercorre tra la Revisione di progettazione e la Revisione di qualifica.}
}

\newglossaryentry{verifica e validazione} {
	name=verifica e validazione,
	description={è il periodo che intercorre tra la Revisione di qualifica e la Revisione di accettazione.}
}

\newglossaryentry{repository} {
	name=repository,
	description={è dove i file sono memorizzati, spesso su un server.}
}

\newglossaryentry{ruolo} {
	name=ruolo,
	description={una delle figure professionali che una persona fisica interpreta nel corso del progetto. I ruoli sono: responsabile, amministratore, analista, progettista, programmatore e verificatore.},
    plural=ruoli
}

\newglossaryentry{svg} {
	name=SVG,
	description={è un formato per la visualizzazione di oggetti in grafica vettoriale. Per maggiori informazioni si veda \href{https://it.wikipedia.org/wiki/Scalable_Vector_Graphics}{qui}.}
}

\newglossaryentry{png} {
	name=PNG,
	description={abbreviazione di Portable Network Graphics, è un formato di file per memorizzare immagini. Per ulteriori informazioni si veda \href{http://it.wikipedia.org/wiki/Portable_Network_Graphics}{qui}.}
}

\newglossaryentry{pdf} {
	name=PDF,
	description={è un formato di file basato su un linguaggio di descrizione di pagina sviluppato da Adobe Systems nel 1993 per rappresentare documenti in modo indipendente dall’hardware e dal software utilizzati per generarli o per visualizzarli. Per ulteriori informazioni si veda \href{http://it.wikipedia.org/wiki/Portable_Document_Format}{qui}.}
}

\newglossaryentry{uml} {
	name=UML,
	description={è un linguaggio di modellazione e specifica basato sul paradigma object-oriented. Per ulteriori informazioni si veda \href{http://it.wikipedia.org/wiki/Unified_Modeling_Language}{qui}.}
}

\newglossaryentry{walkthrough} {
	name=walkthrough,
    description={consiste nella lettura di un documento o codice cercando errori ed anomalie senza un'idea precisa di quali tipi di errori sarà possibile trovare.}
}

\newglossaryentry{lista di controllo} {
	name=lista di controllo,
	description={è un elenco di cose da fare per eseguire una determinata attività.}
}

\newglossaryentry{inspection} {
	name=inspection,
	description={è la lettura mirata di un documento o codice cercando errori specifici.}
}

\newglossaryentry{milestone} {
	name=milestone,
	description={momento saliente nello sviluppo di un prodotto software per la quale devono essere pronti documenti e/o funzionalità.}
}

\newglossaryentry{ticket} {
	name=ticket,
	description={rappresenta un compito nell'organizzazione e distribuzione del lavoro all'interno del progetto.},
	plural=tickets
}

\newglossaryentry{commit} {
	name=commit,
	description={è la copia di modifiche fatte su file locali verso la repository remota. Esso rappresenta anche un particolare stato della repository nel tempo.}
}

\newglossaryentry{versionamento} {
	name=versionamento,
	description={è la gestione di un versioni multiple di un insieme di informazioni. Per maggiori informazioni si veda \href{http://it. wikipedia.org/wiki/Controllo_versione}{qui}.}
}

\newglossaryentry{task} {
	name=task,
	description={è un compito secondo la definizione dello standard IEEE 12207.},
	plural=tasks
}

\newglossaryentry{attivita} {
	name=attivita,
	description={è un insieme di task.}
}

\newglossaryentry{redattore} {
	name=redattore,
	description={colui che redige un documento.},
	plural=redattori
}

\newglossaryentry{proponente} {
	name=proponente,
	description={colui che ha proposto al committente un capitolato d'appalto.}
}

\newglossaryentry{committente} {
	name=committente,
	description={colui che assegna un compito. In questo caso è il Professor Tullio Vardanega.}
}

\newglossaryentry{quality assurance} {
	name=quality assurance,
	description={è l'insieme delle attività volte a garantire il soddisfacimento degli obiettivi della qualità.}
}

\newglossaryentry{telegram} {
	name=telegram,
	description={è un servizio di messaggistica istantanea utilizzato dal gruppo per comunicazioni interne. Per maggiori informazioni si veda \href{https://it.wikipedia.org/wiki/Telegram_(software)}{qui}.}
}

\newglossaryentry{browser} {
	name=browser,
	description={è un'applicazione per il recupero, la presentazione e la navigazione di risorse web.}
}

\newglossaryentry{google drive} {
	name=Google Drive,
	description={è un servizio di memorizzazione e sincronizzazione online introdotto da Google il 24 aprile 2012. Per maggiori informazioni si veda \href{https://it.wikipedia.org/wiki/Google_Drive}{qui}.}
}

\newglossaryentry{skype} {
	name=Skype,
	description={è un software proprietario freeware di messaggistica istantanea e VoIP. Per maggiori informazioni si veda \href{https://it.wikipedia.org/wiki/Skype}{qui}.}
}

\newglossaryentry{gantt} {
	name=Gantt,
	description={è un diagramma di supporto alla gestione dei progetti.}
}

\newglossaryentry{projectlibre} {
	name=ProjectLibre,
	description={è un software di gestione progettuale.}
}

\newglossaryentry{pert} {
	name=PERT,
	description={è uno strumento volto alla programmazione delle attività che compongono il progetto e, più in generale, alla gestione degli aspetti temporali di quest'ultimo.}
}

\newglossaryentry{subtask} {
	name=subtask,
	description={è un task compreso all'interno di un altro task. La totalità di tutti i subtasks costituisce un intero task.}
	plural=subtasks
}

\newglossaryentry{ticketing} {
	name=ticketing,
	description={procedura con la quale il Responsabile assegna un task.}
}

\newglossaryentry{git} {
	name=git,
	description={è un sistema software di controllo di versione distribuito}
}

\newglossaryentry{quizzpedia} {
	name=Quizzpedia,
	description={è il nome del prodotto software richiesto dal capitolato d'appalto scelto.}
}

\newglossaryentry{schierabile} {
	name=schierabile,
	description={è la capacità di rilasciare al cliente, con relativa installazione e messa in funzione o esercizio, di una applicazione o di un sistema software tipicamente all'interno di un sistema informatico aziendale.}
}

\newglossaryentry{cross-platform} {
	name=cross-platform,
	description={può essere riferito ad un linguaggio di programmazione, ad un'applicazione software o ad un dispositivo hardware che funziona su più di un sistema.}
}

\newglossaryentry{qml} {
	name=QML,
	description={è un "Domain Specific Language" richiesto dal capitolato d'appalto per la definizione delle domande all'interno del sistema.}
}

\newglossaryentry{tomcat} {
	name=Tomcat,
	description={è un application server nella forma di contenitore servlet open source sviluppato dalla Apache Software Foundation. Per maggiori informazioni si veda \href{https://it.wikipedia.org/wiki/Apache_Tomcat}{qui}.}
}

\newglossaryentry{java} {
	name=Java,
	description={è un linguaggio di programmazione orientato agli oggetti, specificatamente progettato per essere il più possibile indipendente dalla piattaforma di esecuzione. Per maggiori informazioni si veda \href{https://it.wikipedia.org/wiki/Java_(linguaggio_di_programmazione)}{qui}.}
}

\newglossaryentry{Node.js} {
	name=Node.js,
	description={\TODO{}}
}

\newglossaryentry{JavaScript} {
	name=JavaScript,
	description={\TODO{}}
}

\newglossaryentry{PostgreSQL} {
	name=PostgreSQL,
	description={\TODO{}}
}

\newglossaryentry{MongoDB} {
	name=MongoDB,
	description={\TODO{}}
}

\newglossaryentry{HTML5} {
	name=HTML5,
	description={\TODO{}}
}

\newglossaryentry{CSS3} {
	name=CSS3,
	description={\TODO{}}
}

\newglossaryentry{XML} {
	name=XML,
	description={\TODO{}}
}

\newglossaryentry{NoSQL} {
	name=NoSQL,
	description={\TODO{}}
}

\newglossaryentry{Scala} {
	name=Scala,
	description={\TODO{}}
}

\newglossaryentry{Akka} {
	name=Akka,
	description={\TODO{}}
}

\newglossaryentry{BLE} {
	name=BLE,
	description={\TODO{}}
}

\newglossaryentry{MQTT} {
	name=MQTT,
	description={\TODO{}}
}

\newglossaryentry{AWS} {
	name=AWS,
	description={\TODO{}}
}

\newglossaryentry{Heroku} {
	name=Heroku,
	description={\TODO{}}
}

\newglossaryentry{Github} {
	name=Github,
	description={\TODO{}}
}

\newglossaryentry{template} {
	name=template,
	description={\TODO{}}
	plural=templates
}

\newglossaryentry{Dropbox} {
	name=Dropbox,
	description={\TODO{}}
}

\newglossaryentry{open-source} {
	name=open-source,
	description={\TODO{}}
}

\newglossaryentry{project management} {
	name=project management,
	description={\TODO{}}
}

\newglossaryentry{Linux} {
	name=Linux,
	description={\TODO{}}
}

\newglossaryentry{Windows} {
	name=Windows,
	description={\TODO{}}
}

\newglossaryentry{Mac OS} {
	name=Mac OS,
	description={\TODO{}}
}

\newglossaryentry{program evaluation and review techniques} {
	name=program evaluation and review techniques,
	description={\TODO{}}
}

\newglossaryentry{schedule variance} {
	name=schedule variance,
	description={\TODO{}}
}

\newglossaryentry{cost variance} {
	name=cost variance,
	description={\TODO{}}
}


\newglossaryentry{merge} {
	name=merge,
	description={\TODO{}}
}

\newglossaryentry{slack} {
	name=slack,
	description={\TODO{}}
}

\newglossaryentry{baseline} {
	name=baseline,
	description={\TODO{}}
}

\newglossaryentry{Asana} {
	name=Asana,
	description={\TODO{}}
}

\newglossaryentry{deadline} {
	name=deadline,
	description={\TODO{}}
}

\newglossaryentry{revert} {
	name=revert,
	description={\TODO{}}
}

\newglossaryentry{backup} {
	name=backup,
	description={\TODO{}}
}

\newglossaryentry{push} {
	name=push,
	description={\TODO{}}
}

\newglossaryentry{teamwork} {
	name=teamwork,
	description={\TODO{}}
}

\newglossaryentry{evento} {
	name=evento,
	description={\TODO{}}
}

\newglossaryentry{etichetta} {
	name=etichetta,
	description={\TODO{}}
	plural=etichette
}

\newglossaryentry{bug} {
	name=bug,
	description={\TODO{}}
	plural=bugs
}

\newglossaryentry{software} {
	name=software,
	description={\TODO{}}
}

\newglossaryentry{desktop} {
	name=desktop,
	description={\TODO{}}
}

\newglossaryentry{draw.io} {
	name=draw.io,
	description={\TODO{}}
}

\newglossaryentry{definizione di prodotto} {
	name=definizione di prodotto,
	description={\TODO{}}
}

\newglossaryentry{specifica tecnica} {
	name=specifica tecnica,
	description={\TODO{}}
}

\newglossaryentry{manuale utente} {
	name=manuale utente,
	description={\TODO{}}
}

\newglossaryentry{tracy} {
	name=tracy,
	description={\TODO{}}
}

\newglossaryentry{package} {
	name=package,
	description={\TODO{}}
}

\newglossaryentry{stakeholder} {
	name=stakeholder,
	description={\TODO{}}
}

\newglossaryentry{branch} {
	name=branch,
	description={\TODO{}}
	plural=branches
}

\newglossaryentry{pull} {
	name=pull,
	description={\TODO{}}
}

\newglossaryentry{underscore} {
	name=underscore,
	description={\TODO{}}
}

\newglossaryentry{spelling} {
	name=spelling,
	description={\TODO{}}
}

\newglossaryentry{cloud} {
	name=cloud,
	description={\TODO{}}
}

\newglossaryentry{smartphone} {
	name=smartphone,
	description={\TODO{}}
	plural=smartphones
}

\newglossaryentry{checkbox} {
	name=checkbox,
	description={\TODO{}}
}

\newglossaryentry{appointed} {
	name=appointed,
	description={\TODO{}}
}

\newglossaryentry{makefile} {
	name=makefile,
	description={\TODO{}}
}

\newglossaryentry{gulpease} {
	name=Gulpease,
	description={\TODO{}}
}
